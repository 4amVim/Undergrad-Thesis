\documentclass[20pt,a4paper]{extarticle} %big fontsize
\usepackage[margin=0.7in]{geometry}
\usepackage{amsmath,amsthm,amssymb}
\usepackage{enumitem,amsfonts,extarrows,xcolor,inputenc}

\theoremstyle{definition}
\newtheorem{example}{Example}[section]

\theoremstyle{definition}
\newtheorem{definition}{Definiton}[section]

\begin{document}
%\paragraph{Question 1}
%\paragraph{Question 2}
\paragraph{Question 4} %{{{ Q4 (b)
(b) $f(x,y):= \frac{2xy}{\sqrt{x^2+y^2}}$ \\
For partial derivative along $x$ at $(\delta,0),$
\begin{equation*}
	\begin{split}
		\frac{\partial f}{\partial x}(\delta,0) & =\lim_{h \to 0} \frac{f(\delta +h,0)-f(\delta,0)}{h}	\\
							&=\lim_{h \to 0} \frac{1}{h} \big( \frac{2(\delta+h) \times 0}
								{\sqrt{(\delta+h)^2+0}}
							- \frac{2\delta \times 0}{\sqrt{\delta^2+0}}\big)=0
	\end{split}
\end{equation*}
In particular, if $\delta=0$ then the second term is zero as $f(0,0):=0$. And, due to symmetry,
$f_y(0,\delta)=0$. As :
\begin{equation*}
	\begin{split}
		\frac{\partial f}{\partial y}(0,\delta) & =\lim_{h \to 0} \frac{f(0,\delta +h)-f(0,\delta)}{h}	\\
							&=\lim_{h \to 0} \frac{1}{h}\big(\frac{2 \times 0 \times (\delta+h)}
								{\sqrt{0+(\delta+h)^2}}
							- \frac{2\times 0 \times \delta}{\sqrt{0+\delta^2}}\big)=0
	\end{split}
\end{equation*}

So, not only are the partial derivatives $f_x \text{ and } f_y$ 0 at (0,0), but also along x and y
axes respectively.\\
To show the existence of second derivatives ,
\begin{equation*}
	f_{xx}(0,0) = \lim_{h \to 0} \frac{f_x(h,0)-f_x(0,0)}{h}=\lim_{h \to 0} \frac{0-0}{h}=0
\end{equation*}
Similarly, $f_{yy}(0,0)=0$

Now, if the derivative of f existed at (0,0), then, the following limit must be 0
\begin{equation*}
	\begin{split}
		\lim_{h^2+k^2 \to 0}&\frac{f(0+h,0+k)-f(0,0) -hf_x(0,0)-kf_y(0,0)
		}{\sqrt{h^2+k^2}} \\
		&=\lim_{h^2+k^2 \to 0} \frac{2hk}{h^2+k^2}
	\end{split}
\end{equation*}
But this limit is equal to $1$ along the line of slope 1, and thus f isn't differentiable.
\newpage
%}}}
%{{{ Q4 (a)
\[ \text{ (a) } f(x,y):=\frac{(x^2y+xy^2)sin(x-y)}{x^2+y^2}\]
For partial derivative along $x$ at $(\delta,0),$
\begin{equation*}
	\begin{split}
		&\frac{\partial f}{\partial x}(\delta,0) =\lim_{h \to 0} \frac{f(\delta +h,0)-f(\delta,0)}{h}	\\
	&=\lim_{h \to 0} \frac{1}{h} \big(
		\frac{\big( (\delta+h)^2 \times 0 + (\delta+h) \times 0 \big) sin((\delta+h))}
								{(\delta+h)^2+0}
							- 0\big)=0
	\end{split}
\end{equation*}
In particular, if $\delta=0$ then the second term is still zero as $f(0,0):=0$.
And, due to symmetry, $f_y(0,\delta)=0$.

So, not only are the partial derivatives $f_x \text{ and } f_y$ 0 at (0,0), but also along x and y
axes respectively.\\
To show the existence of second derivatives ,
\begin{equation*}
	f_{xx}(0,0) = \lim_{h \to 0} \frac{f_x(h,0)-f_x(0,0)}{h}=\lim_{h \to 0} \frac{0-0}{h}=0
\end{equation*}
Similarly, $f_{yy}(0,0)=0$

Now, to show the existence of derivative of f at (0,0), then, the following limit must be 0
\begin{equation*}
	\begin{split}
		\lim_{h^2+k^2 \to 0}&\frac{f(0+h,0+k)-f(0,0) -hf_x(0,0)-kf_y(0,0)
		}{\sqrt{h^2+k^2}} \\
		&=\lim_{h^2+k^2 \to 0} \frac{(h^2k+hk^2)(sin(h-k))}{(h^2+k^2)\sqrt{h^2+k^2)}} \\
	\end{split}
\end{equation*}
Now, as we want to evaluate differentiablity at (0,0), it's enough to consider all the lines of the
form $y=mx$ i.e. $(i,mi) \text{ with } i \in \mathbb{R}$. Also, for points along the x-axis, the limit is zero as,
\begin{equation*}
	\begin{split}
		&\lim_{h^2+k^2 \to 0} \frac{(0 \times k+0 \times k^2)(sin(0-k))}{(0+k^2)\sqrt{0+k^2)}}=0
	\end{split}
\end{equation*}
By symmetry, the limit is zero along y-axis as well, so, let m be some finite, non-zero number,
and consider the line $(i,mi)$,
\begin{equation*}
	\begin{split}
		&\lim_{i \to 0} \frac{(i^2 \times mi+i \times m^2i^2)
		(sin(i-mi))}{(i^2+m^2i^2)\sqrt{i^2+m^2i^2)}}\\
		&=\lim_{i \to 0} \frac{i^3(m+m^2)(sin(i(1-m))}{i^2(1+m^2) |i|\sqrt{1+m^2}}\\
		&=\lim_{i \to 0} \frac{m+m^2}{(1+m^2) \sqrt{1+m^2}} \times sgn(i) \times sin(i(1-m)) \\
	\end{split}
\end{equation*}
For any particular line, the first term is a constant, second one is $\pm 1$, while the third one goes
to $0$ as $i \to 0$. Thus, the limit is $0$ for every line going through the origin.

Thus, the differential exists at the origin.


\newpage
%}}}


%{{{ defended already
\paragraph{Question 5 (Marked)}
\[ f(x,y)=xy(1-x^2-y^2) \]
For each partial derivative to be zero,
\begin{align*}
	& f_x=y-3x^2y-y^3=0 \text{ and }f_y=x-x^3-3xy^2=0 \\
	\implies & f_x=y(1-3x^2-y^2)=0 \text{ and } f_y=x(1-x^2-3y^2)=0
\end{align*}
So, a critical point is $(0,0)$.\\

And,\\
If $x=0,y\neq0$ then $f_x=y(1-y^2)=0 \implies y= \pm 1$\\
gives two critical points: $(0,\pm 1)$\\

If $y=0,x\neq0$ then $0=f_y=y(1-x^2) \implies x= \pm 1$ \\
gives two critical points: $(\pm 1,0)$\\

If $x\neq 0 \neq y$, then \\
\begin{align*}
	f_x=y-3x^2y-y^3=0 & \implies y(1-3x^2)=y^3 &\\
			  & \implies 1-3x^2=y^2 &(equation A)\\
	\text{ And, also, } & &\\
	f_y=x-3y^2x-x^3=0 & \implies x(1-3y^2)=x^3 &\\
			  & \implies 1-3y^2=x^2 &(equation B)\\
\end{align*}
Thus,
\begin{align*}
	\text{ Substituting B into A, } & 1-3+9y^2=y^2 \implies y = \pm 0.5 \\
	\text{ And, substituting A into B, } & 1-3+9x^2=x^2 \implies x = \pm 0.5
\end{align*}

Thus, the four such possible points are also critical:
\[(0.5,0.5),(0.5,-0.5),(-0.5,0.5),(-0.5,-0.5)\]
Now, to classify these critical points, looking at $rt-s^2$
\[ r=f_{xx}=-6xy=f_{yy}=t \text{ and } s=f_{xy}=1-3x^2-3y^2 \]
\[ \text{ So, } rt=36x^2y^2 \text{ and } s^2= (1-3(x^2+y^2))^2 \]

At $(0,0), rt=0$ and $s=1$. Thus, $rt-s^2=0-1<0 $\\
Thus, $(0,0)$ is a saddle point.\\

At $(\pm 1,0) \text{ and } (0, \pm 1)$,\\
$rt=36\times 1 \times 0=36 \times 0 \times 1=0$ and $s^2=(1-3)^2=4$.\\
Thus, $rt-s^2=0-4<0$\\
Thus, $(\pm 1,0) , (0, \pm 1)$ are saddle points.\\

At $(0.5,0.5),(0.5,-0.5),(-0.5,0.5),(-0.5,-0.5)$,\\
$rt=36x^2y^2=36 \times 0.25 \times 0.25=\frac{9}{4}=2.25$ \\ and $s^2=(1-3(0.25+0.25))^2=(-0.5)^2=0.25$\\
Hence, $rt-s^2=2.25-0.25=2>0$ \\
Thus, $(0.5,-0.5),(-0.5,0.5)$ are minima(as $r=1.5$ ),\\ while $(0.5,0.5),(-0.5,-0.5)$ are maxima(as $r=1.5$ ).
\pagebreak


\paragraph{Question 3 (Marked)}
\textbf{(a)} $f(x,y):=\frac{x^2y^2}{x^2y^2+(x-y)^2}=\frac{1}{1+(\frac{x-y}{xy})^2}=\frac{1}{1+(\frac{1}{y}-\frac{1}{x})^2}$\\
\[ \lim_{x \to 0} f(x,y)=\lim_{x \to 0} \frac{1}{1+(\frac{1}{y}-\frac{1}{x})^2}=0
\implies \lim_{y \to 0}\lim_{x \to 0}f(x,y)=0 \]
And as the expression is symmetric in $x \text{ and } y$,
\[ \lim_{y \to 0}\lim_{x \to 0}f(x,y)=\lim_{x \to 0}\lim_{y \to 0}f(x,y)\]
But the simultaneous limit at$(0,0)$  along $T(t):=(t,t) $ is
\[ \lim_{t \to 0} f(T(t))= \lim_{t \to 0} \frac{1}{1+(\frac{1}{t}-\frac{1}{t})^2}=1 \]
If the simultaneous limit existed, all the iterated limits would be equal to it. So, there is a curve,
$S(t):=(\frac{1}{t},\frac{1}{t+1})$ with
\[ \lim_{t \to \infty} f(S(t)) = \lim_{t \to \infty} \frac{1}{1+(t+1-t)^2}=\frac{1}{2} \]
Thus, $f$ is discontinous at $(0,0)$

\textbf{(b)} $f(x,y):=\frac{e^{\frac{-1}{x^2}}y}{e^{\frac{-1}{x^2}}+y^2}=\frac{ye^{\frac{1}{x^2}}}
{ 1+(ye^{\frac{1}{x^2}})^2}=\frac{1}{ \frac{1}{ye^{1/x^2}}+ye^{1/x^2}} $\\
\[ \lim_{y \to 0} \lim_{x \to 0}\frac{1}{ \frac{1}{ye^{1/x^2}}+ye^{1/x^2}}= \lim_{y \to 0} 0=0  \]
\[ \lim_{x \to 0} \lim_{y \to 0}\frac{ye^{\frac{1}{x^2}}}{ 1+(ye^{\frac{1}{x^2}})^2}=\lim_{x \to 0} 0=0\]
To show the non-existence of simulatenous limit at $(0,0)$, consider the curve $T(t):=(t,e^{-1/t^2})$
\[\lim_{t\to 0}f(T(t))=\frac{e^{-1/t^2}\times e^{1/t^2}}{1+(e^{-1/t^2}\times e^{1/t^2})^2}=\frac{1}{2} \]
\newpage
%}}}
\end{document}

