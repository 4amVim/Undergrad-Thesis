\documentclass[20pt,a4paper]{extarticle} %big fontsize
\usepackage[margin=0.7in]{geometry}
\usepackage{amsmath,amsthm,amssymb}
\usepackage{enumitem,amsfonts,extarrows,xcolor}

\theoremstyle{definition}
\newtheorem{example}{Example}[section]

\theoremstyle{definition}
\newtheorem{definition}{Definiton}[section]

\begin{document}
\paragraph{Question 1}
For counter-example, consider
\newpage
\paragraph{Question 3}
\textbf{(a)} $f(x,y):=\frac{x^2y^2}{x^2y^2+(x-y)^2}=\frac{1}{1+(\frac{x-y}{xy})^2}=\frac{1}{1+(\frac{1}{y}-\frac{1}{x})^2}$\\
\[ \lim_{x \to 0} f(x,y)=\lim_{x \to 0} \frac{1}{1+(\frac{1}{y}-\frac{1}{x})^2}=0
\implies \lim_{y \to 0}\lim_{x \to 0}f(x,y)=0 \]
And as the expression is symmetric in $x \text{ and } y$,
\[ \lim_{y \to 0}\lim_{x \to 0}f(x,y)=\lim_{x \to 0}\lim_{y \to 0}f(x,y)\]
But the simultaneous limit at$(0,0)$  along $T(t):=(t,t) $ is
\[ \lim_{t \to 0} f(T(t))= \lim_{t \to 0} \frac{1}{1+(\frac{1}{t}-\frac{1}{t})^2}=1 \]
If the simultaneous limit existed, all the iterated limits would be equal to it. So, there is a curve,
$S(t):=(\frac{1}{t},\frac{1}{t+1})$ with
\[ \lim_{t \to \infty} f(S(t)) = \lim_{t \to \infty} \frac{1}{1+(t+1-t)^2}=\frac{1}{2} \]
Thus, $f$ is discontinous at $(0,0)$

\textbf{(b)} $f(x,y):=\frac{e^{\frac{-1}{x^2}}y}{e^{\frac{-1}{x^2}}+y^2}=\frac{ye^{\frac{1}{x^2}}}
{ 1+(ye^{\frac{1}{x^2}})^2}=\frac{1}{ \frac{1}{ye^{1/x^2}}+ye^{1/x^2}} $\\
\[ \lim_{y \to 0} \lim_{x \to 0}\frac{1}{ \frac{1}{ye^{1/x^2}}+ye^{1/x^2}}= \lim_{y \to 0} 0=0  \]
\[ \lim_{x \to 0} \lim_{y \to 0}\frac{ye^{\frac{1}{x^2}}}{ 1+(ye^{\frac{1}{x^2}})^2}=\lim_{x \to 0} 0=0\]
To show the non-existence of simulatenous limit at $(0,0)$, consider the curve $T(t):=(t,e^{-1/t^2})$
\[ \lim_{t \to 0} f(T(t)) = \frac{e^{-1/t^2}\times e^{1/t^2}}{1+(e^{-1/t^2}\times e^{1/t^2})^2}=\frac{1}{2} \]

\paragraph{Question 4}
\end{document}

