\documentclass[20pt,a4paper]{extarticle} %big fontsize
\usepackage[margin=0.7in]{geometry}
\usepackage{amsmath,amsthm,amssymb}
\usepackage{enumitem,amsfonts,extarrows,xcolor}

\theoremstyle{definition}
\newtheorem{example}{Example}[section]

\theoremstyle{definition}
\newtheorem{definition}{Definiton}[section]

\begin{document}
%\paragraph{Question 1}
%\paragraph{Question 2}
\paragraph{Question 3}
\textbf{(a)} $f(x,y):=\frac{x^2y^2}{x^2y^2+(x-y)^2}=\frac{1}{1+(\frac{x-y}{xy})^2}=\frac{1}{1+(\frac{1}{y}-\frac{1}{x})^2}$\\
\[ \lim_{x \to 0} f(x,y)=\lim_{x \to 0} \frac{1}{1+(\frac{1}{y}-\frac{1}{x})^2}=0
\implies \lim_{y \to 0}\lim_{x \to 0}f(x,y)=0 \]
And as the expression is symmetric in $x \text{ and } y$,
\[ \lim_{y \to 0}\lim_{x \to 0}f(x,y)=\lim_{x \to 0}\lim_{y \to 0}f(x,y)\]
But the simultaneous limit at$(0,0)$  along $T(t):=(t,t) $ is
\[ \lim_{t \to 0} f(T(t))= \lim_{t \to 0} \frac{1}{1+(\frac{1}{t}-\frac{1}{t})^2}=1 \]
If the simultaneous limit existed, all the iterated limits would be equal to it. So, there is a curve,
$S(t):=(\frac{1}{t},\frac{1}{t+1})$ with
\[ \lim_{t \to \infty} f(S(t)) = \lim_{t \to \infty} \frac{1}{1+(t+1-t)^2}=\frac{1}{2} \]
Thus, $f$ is discontinous at $(0,0)$

\textbf{(b)} $f(x,y):=\frac{e^{\frac{-1}{x^2}}y}{e^{\frac{-1}{x^2}}+y^2}=\frac{ye^{\frac{1}{x^2}}}
{ 1+(ye^{\frac{1}{x^2}})^2}=\frac{1}{ \frac{1}{ye^{1/x^2}}+ye^{1/x^2}} $\\
\[ \lim_{y \to 0} \lim_{x \to 0}\frac{1}{ \frac{1}{ye^{1/x^2}}+ye^{1/x^2}}= \lim_{y \to 0} 0=0  \]
\[ \lim_{x \to 0} \lim_{y \to 0}\frac{ye^{\frac{1}{x^2}}}{ 1+(ye^{\frac{1}{x^2}})^2}=\lim_{x \to 0} 0=0\]
To show the non-existence of simulatenous limit at $(0,0)$, consider the curve $T(t):=(t,e^{-1/t^2})$
\[\lim_{t\to 0}f(T(t))=\frac{e^{-1/t^2}\times e^{1/t^2}}{1+(e^{-1/t^2}\times e^{1/t^2})^2}=\frac{1}{2} \]

\newpage
\paragraph{Question 4}
(b) $f(x,y):= \frac{2xy}{\sqrt{x^2+y^2}}$ \\
\[ \frac{\partial f}{\partial x}=\lim_{h \to 0} \frac{f(h,0)-f(0,0)}{h}=
\lim_{h \to 0} \frac{1}{h} \big( \frac{2h \times 0}{\sqrt{h^2+0}} - 0 \big)=0\]
So, the partial derivative along $x$ exists at $(0,0)$. And by symmetry, it also exists along $y$.
And in a neighbourhood of $(0,0)$ ,
$\frac{\partial f}{\partial x}= \frac{2y\sqrt{x^2+y^2}-2xy(\frac{2x}{2\sqrt{x^2+y^2}})}{x^2+y^2}
=\frac{y\Big( 2\sqrt{x^2+y^2}-2x(\frac{2x}{2\sqrt{x^2+y^2}})\Big)}{x^2+y^2}$.
\[ \frac{\partial^2 f}{\partial x^2}=\lim_{h \to 0} \frac{f_x(h,0)-f_x(0,0)}{h}=
\lim_{h \to 0}\frac{0-0}{0}=0\]
Again, by symmetry, $\frac{\partial^2 f}{\partial y^2}(0,0)=0$. Thus, as second partial derivative
exists along $x \text{ and } y$, the first derivatives are continous at $(0,0)$ , and hence, f is
differentiable at $(0,0)$.\\

(a) $f(x,y):=\frac{xy(x+y)sin(x-y)}{x^2+y^2}$\\
The partial derivative along $x$,
\[ \frac{\partial f}{\partial x} =\lim_{h \to 0} \Big( \frac{f(h,0)-f(0,0)}{h} \Big) \]
\[=\lim_{h \to 0} \frac{(h\times 0(h+0)sin(h-0)-0)}{h^2+0}=0 \]
And in the neighbourhood of $(0,0)$,
\[ \frac{\partial f}{\partial x} = \frac{y\big( (-x^2y+2xy^2+y^3)sin(x-y)+(x^3+x^3y+x^2y^2+xy^3)cos(x-y) \big)]}{(x^2+y^2)^2} \]
Thus, for the second partial derivative along x,
\[ \frac{\partial^2 f}{\partial x^2}=\lim_{h \to 0} \frac{1}{h} \Big( \frac{\partial f (h,0)}{\partial x}
-\frac{\partial f (0,0)}{\partial x} \Big)= \lim_{h \to 0} \frac{(0\times (...))-0}{h}=0 \]
Similarly, along the y-direction, \\
\[ \frac{\partial f}{\partial x} = \lim_{h \to 0} \frac{f(0,h)-f(0,0)}{h}= \lim_{h \to 0}
\frac{0-0}{h}=0\]
And,
\[  \frac{\partial f}{\partial x} = \frac{-x}{(x^2+y^2)^2}[(-x^3-2x^2y+xy^2)sin(x-y)+...]\]
the second order derivative is,
\[ \frac{\partial^2 f}{\partial x^2} = \lim_{h \to 0} \frac{f_y(0,h)-f_y(0,0)}{h}=0 \]
Now, as the second partial derivative exists along x and y directions, the first partial derivatives are continous. Hence, $f$ is differentiable at (0,0).
\newpage
\paragraph{Question 5}
\[ f(x,y)=xy(1-x^2-y^2) \]
\[ f_x=y-3x^2y-y^3 \text{ and }f_y=x-x^3-3xy^2  \]
So, a critical point is $(0,0)$. And,\\
If $x=0,y\neq0$ then $0=f_x=y(1-y^2) \implies y= \pm 1$\\gives a critical point $(0,\pm 1)$\\
If $y=0,x\neq0$ then $0=f_y=y(1-x^2) \implies x= \pm 1$ \\gives a critical point $(\pm 1,0)$\\

If $x\neq 0 \neq y$, then \\
$f_y=y-3x^2y-y^3=0=x-x^3-3xy^2=f_y$ gives $(\pm 0.5, \pm 0.5)$ \\
Now, to classify these critical points, looking at $rt-s^2$
\[ r=f_{xx}=-6xy=f_{yy}=t \text{ and } s=f_{xy}=1-3x^2-3y^2 \]

At $(0,0), r=0=t$ and $s=1$. Thus, $rt-s^2=0-1<0$\\
Thus, $(0,0)$ is a maxima.\\

At $(\pm 1,0) \text{ and } (0, \pm 1)$,\\
$rt=36x^2y^2=0$ and $s=1-3$. Thus, $rt-s^2=0-4<0$\\
Thus, $(\pm 1,0) , (0, \pm 1)$ are maxima.\\

At $(\pm 0.5,\pm 0.5) \text{ and } (-0.5,-0.5)$,\\
$rt=36x^2y^2=\frac{9}{4}$ and $s=1-3(0.25+0.25)=-0.5$\\
Hence, $rt-s^2=2.25-0.25=2>0$ \\
Thus, all four points, $ (\pm 0.5,\pm 0.5) $ are minima.
\end{document}

