\documentclass[20pt,a4paper]{extarticle} %big fontsize
\usepackage[margin=0.7in]{geometry}
\usepackage{amsmath,amsthm,amssymb}
\usepackage{enumitem,amsfonts,extarrows,xcolor,inputenc}

\theoremstyle{definition}
\newtheorem{example}{Example}[section]

\theoremstyle{definition}
\newtheorem{definition}{Definiton}[section]

\begin{document}
%\paragraph{Question 1}
%\paragraph{Question 2}
\paragraph{Question 5}
\[ f(x,y)=xy(1-x^2-y^2) \]
For each partial derivative to be zero,
\begin{align*}
	& f_x=y-3x^2y-y^3=0 \text{ and }f_y=x-x^3-3xy^2=0 \\
	\implies & f_x=y(1-3x^2-y^2)=0 \text{ and } f_y=x(1-x^2-3y^2)=0
\end{align*}
So, a critical point is $(0,0)$.\\

And,\\
If $x=0,y\neq0$ then $f_x=y(1-y^2)=0 \implies y= \pm 1$\\
gives two critical points: $(0,\pm 1)$\\

If $y=0,x\neq0$ then $0=f_y=y(1-x^2) \implies x= \pm 1$ \\
gives two critical points: $(\pm 1,0)$\\

If $x\neq 0 \neq y$, then \\
\begin{align*}
	f_x=y-3x^2y-y^3=0 & \implies y(1-3x^2)=y^3 &\\
			  & \implies 1-3x^2=y^2 &(equation A)\\
	\text{ And, also, } & &\\
	f_y=x-3y^2x-x^3=0 & \implies x(1-3y^2)=x^3 &\\
			  & \implies 1-3y^2=x^2 &(equation B)\\
\end{align*}
Thus,
\begin{align*}
	\text{ Substituting B into A, } & 1-3+9y^2=y^2 \implies y = \pm 0.5 \\
	\text{ And, substituting A into B, } & 1-3+9x^2=x^2 \implies x = \pm 0.5
\end{align*}

Thus, the four such possible points are also critical:
\[(0.5,0.5),(0.5,-0.5),(-0.5,0.5),(-0.5,-0.5)\]
Now, to classify these critical points, looking at $rt-s^2$
\[ r=f_{xx}=-6xy=f_{yy}=t \text{ and } s=f_{xy}=1-3x^2-3y^2 \]
\[ \text{ So, } rt=36x^2y^2 \text{ and } s^2= (1-3(x^2+y^2))^2 \]

At $(0,0), rt=0$ and $s=1$. Thus, $rt-s^2=0-1<0 $\\
Thus, $(0,0)$ is a saddle point.\\

At $(\pm 1,0) \text{ and } (0, \pm 1)$,\\
$rt=36\times 1 \times 0=36 \times 0 \times 1=0$ and $s^2=(1-3)^2=4$.\\
Thus, $rt-s^2=0-4<0$\\
Thus, $(\pm 1,0) , (0, \pm 1)$ are saddle points.\\

At $(0.5,0.5),(0.5,-0.5),(-0.5,0.5),(-0.5,-0.5)$,\\
$rt=36x^2y^2=36 \times 0.25 \times 0.25=\frac{9}{4}=2.25$ \\ and $s^2=(1-3(0.25+0.25))^2=(-0.5)^2=0.25$\\
Hence, $rt-s^2=2.25-0.25=2>0$ \\
Thus, $(0.5,-0.5),(-0.5,0.5)$ are minima(as $r=1.5$ ),\\ while $(0.5,0.5),(-0.5,-0.5)$ are maxima(as $r=1.5$ ).
\pagebreak

\paragraph{Question 4}
(b) $f(x,y):= \frac{2xy}{\sqrt{x^2+y^2}}$ \\
\[ \frac{\partial f}{\partial x}=\lim_{h \to 0} \frac{f(h,0)-f(0,0)}{h}=
\lim_{h \to 0} \frac{1}{h} \big( \frac{2h \times 0}{\sqrt{h^2+0}} - 0 \big)=0\]
So, the partial derivative along $x$ exists at $(0,0)$. And by symmetry, it also exists along $y$.\\
And at a point, $(x,y)$ in a strict neighbourhood of $(0,0)$ ,
\begin{align*}
	\frac{\partial f}{\partial y}(x,y)&=\lim_{k \to 0} \frac{1}{k} \Big(
	\frac{2x(y+k)}{\sqrt{x^2+(y+k)^2}}- \frac{2xy}{\sqrt{x^2+y^2}}  \Big) \\
					  & =\lim_{k \to 0} \frac{1}{k} \Big(
						  \frac{2x(y+k) \sqrt{x^2+y^2} - 2xy(\sqrt{x^2+(y+k)^2}}
						  {\sqrt{x^2+y^2}\sqrt{x^2+(y+k)^2}} \Big) \\
					  & =\lim_{k \to 0} \frac{2x}{k} \Big(
						  \frac{(y+k)\sqrt{x^2+y^2} - y\sqrt{x^2+(y+k)^2}}
						  {\sqrt{x^2+y^2}\sqrt{x^2+(y+k)^2}} \Big)\\
\end{align*}
If the above derivative is continous,then $ \lim_{(j,0) \to (0,0)}\frac{\partial f}{\partial x} = 0$ \\
\begin{align*}
	\label{eq:}
\frac{\partial f}{\partial y}(j,0) 				  & =\lim_{k \to 0} \frac{2j}{k} \Big(
						  \frac{(0+k)\sqrt{j^2+0^2} - 0\sqrt{j^2+(0+k)^2}}
						  {\sqrt{j^2+0^2}\sqrt{j^2+(0+k)^2}} \Big) \\
						  & =\lim_{k \to 0} \frac{2j}{k} \Big(
							  \frac{k|j|}{|j|\sqrt{j^2+k^2}} \Big)
							  =\lim_{k \to 0} \frac{2j}{\sqrt{j^2+k^2}}=2
\end{align*}
Thus, $\frac{\partial f}{\partial y}$ is discontinous at (0,0) and by symmetry, so is
$\frac{\partial f}{\partial x}$. \newpage
Hence $\frac{\partial^2 f}{\partial x^2} \text{ and }
\frac{\partial^2 f}{\partial x^2}$ don't exist at $(0,0)$.
For differentiability at (0,0), consider the following limit along line $i=j$,
\begin{align*}
&\lim_{h^2+k^2 \to 0} \frac{f(0+h,0+k)- f(0,0)-h\frac{\partial f}{\partial x}(0,0)
-k\frac{\partial f}{\partial y}(0,0)}{\sqrt{h^2+k^2 }} \\
&=\lim_{h^2+k^2 \to 0} \frac{\frac{2hk}{\sqrt{h^2+k^2}}- 0-h\times 0-k \times 0)}{\sqrt{h^2+k^2 }}
=\lim_{j=i \to 0} \frac{1}{\sqrt{2j^2}} \frac{2j^2}{\sqrt{2j^2}}=1
\end{align*}
Hence, as aforementioned limit is non-zero for some path through $(0,0)$, the function isn't
differentiable at $(0,0)$. \\

(a) $f(x,y):=\frac{xy(x+y)sin(x-y)}{x^2+y^2}$\\
The partial derivative along $x \text{ at } (0,0) $,
\[ \frac{\partial f}{\partial x} =\lim_{h \to 0} \Big( \frac{f(h,0)-f(0,0)}{h} \Big) \]
\[=\lim_{h \to 0} \frac{(h\times 0(h+0)sin(h-0)-0)}{h^2+0}=0 \]
And at a point, $(a,b)$  in the neighbourhood of $(0,0)$ partial derivative along x is,
\begin{align*}
	& \lim_{h \to 0} \frac{f(a+h,b)-f(a,b)}{h} = \\ & \lim_{h \to 0} \frac{1}{h} \Big(
	\frac{(x+h)y(x+h+y)sin(x+h-y)}{(x+h)^2+y^2}- \frac{xy(x+y)sin(x-y)}{x^2+y^2} \Big)
\end{align*}

(WIP)
\[ \frac{\partial f}{\partial x} = \frac{y\big( (-x^2y+2xy^2+y^3)sin(x-y)+(x^3+x^3y+x^2y^2+xy^3)cos(x-y) \big)]}{(x^2+y^2)^2} \]
Thus, for the second partial derivative along x,
\[ \frac{\partial^2 f}{\partial x^2}=\lim_{h \to 0} \frac{1}{h} \Big( \frac{\partial f (h,0)}{\partial x}
-\frac{\partial f (0,0)}{\partial x} \Big)= \lim_{h \to 0} \frac{(0\times (...))-0}{h}=0 \]
Similarly, along the y-direction, \\
\[ \frac{\partial f}{\partial x} = \lim_{h \to 0} \frac{f(0,h)-f(0,0)}{h}= \lim_{h \to 0}
\frac{0-0}{h}=0\]
And,
\[  \frac{\partial f}{\partial x} = \frac{-x}{(x^2+y^2)^2}[(-x^3-2x^2y+xy^2)sin(x-y)+...]\]
the second order derivative is,
\[ \frac{\partial^2 f}{\partial x^2} = \lim_{h \to 0} \frac{f_y(0,h)-f_y(0,0)}{h}=0 \]
Now, as the second partial derivative exists along x and y directions, the first partial derivatives are continous. Hence, $f$ is differentiable at (0,0).
\newpage



\paragraph{Question 3 (Marked)}
\textbf{(a)} $f(x,y):=\frac{x^2y^2}{x^2y^2+(x-y)^2}=\frac{1}{1+(\frac{x-y}{xy})^2}=\frac{1}{1+(\frac{1}{y}-\frac{1}{x})^2}$\\
\[ \lim_{x \to 0} f(x,y)=\lim_{x \to 0} \frac{1}{1+(\frac{1}{y}-\frac{1}{x})^2}=0
\implies \lim_{y \to 0}\lim_{x \to 0}f(x,y)=0 \]
And as the expression is symmetric in $x \text{ and } y$,
\[ \lim_{y \to 0}\lim_{x \to 0}f(x,y)=\lim_{x \to 0}\lim_{y \to 0}f(x,y)\]
But the simultaneous limit at$(0,0)$  along $T(t):=(t,t) $ is
\[ \lim_{t \to 0} f(T(t))= \lim_{t \to 0} \frac{1}{1+(\frac{1}{t}-\frac{1}{t})^2}=1 \]
If the simultaneous limit existed, all the iterated limits would be equal to it. So, there is a curve,
$S(t):=(\frac{1}{t},\frac{1}{t+1})$ with
\[ \lim_{t \to \infty} f(S(t)) = \lim_{t \to \infty} \frac{1}{1+(t+1-t)^2}=\frac{1}{2} \]
Thus, $f$ is discontinous at $(0,0)$

\textbf{(b)} $f(x,y):=\frac{e^{\frac{-1}{x^2}}y}{e^{\frac{-1}{x^2}}+y^2}=\frac{ye^{\frac{1}{x^2}}}
{ 1+(ye^{\frac{1}{x^2}})^2}=\frac{1}{ \frac{1}{ye^{1/x^2}}+ye^{1/x^2}} $\\
\[ \lim_{y \to 0} \lim_{x \to 0}\frac{1}{ \frac{1}{ye^{1/x^2}}+ye^{1/x^2}}= \lim_{y \to 0} 0=0  \]
\[ \lim_{x \to 0} \lim_{y \to 0}\frac{ye^{\frac{1}{x^2}}}{ 1+(ye^{\frac{1}{x^2}})^2}=\lim_{x \to 0} 0=0\]
To show the non-existence of simulatenous limit at $(0,0)$, consider the curve $T(t):=(t,e^{-1/t^2})$
\[\lim_{t\to 0}f(T(t))=\frac{e^{-1/t^2}\times e^{1/t^2}}{1+(e^{-1/t^2}\times e^{1/t^2})^2}=\frac{1}{2} \]

\newpage
\end{document}

