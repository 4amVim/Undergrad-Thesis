\documentclass[20pt]{extarticle} %big fontsize
\usepackage[margin=1in]{geometry}
\usepackage{amsmath,amsthm,amssymb}
\usepackage{enumitem}
%force margins to 1inch
%\addtolength{\oddsidemargin}{-.875in}
%\addtolength{\evensidemargin}{-.875in}
%\addtolength{\textwidth}{1.75in}
%\addtolength{\topmargin}{-.875in}
%\addtolength{\textheight}{1.75in}

\newcommand{\N}{\mathbb{N}}

\newenvironment{solution}{\begin{proof}[Solution]}{\end{proof}}

\begin{document}

\paragraph{Question 1}
\[ <a_n>\text{ is a real sequence; }\sigma_n := \frac{a_1+a_2+...+a_n}{n} \]
Going to show that
\begin{enumerate}[label=\Roman*]
	\item $\text{lim inf } a_n \leq \text{lim inf }\sigma_n  $ \\
		If $a_n$ is unbounded below, then \text{lim inf }$a_n = -\infty \leq \text{lim inf }\sigma_n$
		So, let $a_n$ be bounded below, thus LHS is a real number,
		\[ m:= \text{lim inf }a_n \]
		Now, if $a_n$ is constant, and equal to a,
		\[ \forall n \in \mathbb{N} \text{ , inf} \{ a_i | i\geq n \} =a = \frac{na}{n} =\sigma_n\]
		Otherwise, if $a_n$ is not constant, then,
		\[ \exists i,j \in \mathbb{N} \text { such that(wlog) } a_i < a_j \]
		Suppose if possible, lim inf $a_n=m> \text{ lim inf } \sigma_n$
		But, \[ \forall n \in \mathbb{N} \text{ such that } n \geq i,j, \]
		\begin{equation*}
			\begin{split}
				\sigma_n= \frac{a_1 + ... a_i + a_j + ... + a_n}{n} & \geq{\frac{(n-2)m+a_i+a_j}{n}}\\
										    & \geq \frac{(n-1)m+a_j}{n}\\
										    & >m \text{ [ $\because m \leq a_i < a_j $ ] }
			\end{split}
		\end{equation*}
		\[	\text{Hence, inf } \sigma_n \geq \frac{(n-1)m+a_j}{n} \]
		\[		\implies \text{lim inf } \sigma_n \geq \lim_{n \to \infty} \frac{(n-1)m+a_j}{n}=m\]
		\text{But this contradicts the initial assumption.}

	\item \text{lim sup }$ a_n \geq \text{lim sup }\sigma_n$
		If $a_n$ is unbounded above, then \text{lim sup }$a_n =  \infty \geq \text{lim sup }\sigma_n$ .\\

		So, let $a_n$ be bounded above, thus LHS is a real number,
		\[ M:= \text{lim sup }a_n \]
		Now, if $a_n$ is constant, and equal to a,
		\[ \forall n \in \mathbb{N} \text{ , sup} \{ a_i | i\geq n \} =a = \frac{na}{n} =\sigma_n\]
		Otherwise, if $a_n$ is not constant, then,
		\[ \exists i,j \in \mathbb{N} \text { such that(wlog) } a_i < a_j \]
		Suppose if possible, lim sup $a_n=M< \text{ lim sup } \sigma_n$
		But, \[ \forall n \in \mathbb{N} \text{ such that } n \geq i,j, \]
		\begin{equation*}
			\begin{split}
				\sigma_n= \frac{a_1 + ... a_i + a_j + ... + a_n}{n} & \leq{\frac{(n-2)M+a_i+a_j}{n}}\\
										    & \leq \frac{(n-1)M+a_j}{n}\\
										    & <M \text{ [ $\because M \geq a_j > a_i $ ] }
			\end{split}
		\end{equation*}
		\[	\text{Hence, inf } \sigma_n \leq \frac{(n-1)M+a_j}{n} \]
		\[		\implies \text{lim inf } \sigma_n \leq \lim_{n \to \infty} \frac{(n-1)M+a_j}{n}=M\]
		\text{But this contradicts the initial assumption.}
\end{enumerate}

\paragraph{Question 2}
\[ \text{lim inf } \frac{a_{n+1}}{a_n} \leq \text{lim inf }(a_n)^\frac{1}{n} \leq \text{lim sup }(a_n)^\frac{1}{n} \leq \text{lim sup }\frac{a_{n+1}}{a_n} \]

\begin{enumerate}[label=\Roman*]
	\item Showing that $\text{lim inf } \frac{a_{n+1}}{a_n} \leq \text{lim inf }(a_n)^\frac{1}{n}$\\
		Case 1: \text{lim inf } $ \frac{a_{n+1}}{a_n}=0 $
		\[ \forall n \in \mathbb{N}, a_n > 0 \implies (a_n)^ \frac{1}{n}> 0 \]
		\[ \text{ Thus } , \text{lim inf }(a_n)^ \frac{1}{n} \geq 0 = \text{lim inf } \frac{a_{n+1}}{a_n} 	 \]

		Case 2: \text{lim inf } $ \frac{a_{n+1}}{a_n}= \infty $\\
		So, for any $ a \in \mathbb{N},$
		\[ \exists M_a \in \mathbb{N}: n\geq	M_a \implies \frac{a_{n+1}}{a_n}> a \]
		Fix any a, and choose $n>M_a$. For any such n,
		\[  \frac{a_{n+1}}{a_n}>a \implies a_{n+1} >aa_n  \]
		\[ \implies a_n>aa_{n-1}> a^2a_{n-2}>...>a^{n-M}a_M \]
		\[ \implies  (a_n)^ \frac{1}{n} > a( \frac{a_M}{a^M} )^	\frac{1}{n} \]
		Now as for fixed $a$, $ \frac{a_M}{a^M} $ is constant,
		\[ \lim_{n \to \infty} (\frac{a_M}{a^M})^{ \frac{1}{n} }=1 \implies \lim_{n \to \infty} a(\frac{a_M}{a^M})^{ \frac{1}{n} }=a\]
		Thus, \[\forall \epsilon >0, \exists K \in \mathbb{N}: n>K \implies a(\frac{a_M}{a^M})^{ \frac{1}{n} }>a- \epsilon \]
		And hence, in particular, $a(\frac{a_M}{a^M})^{ \frac{1}{n} }>a-1$ for any natural number $a$.
		\[ (a_n)^ \frac{1}{n}  > a(\frac{a_M}{a^M})^{ \frac{1}{n} }> a-1 \implies \lim_{n \to \infty} (a_n)^ \frac{1}{n}= \infty  \]
		\[ \therefore  \text{lim inf } \frac{a_{n+1}}{a_n} = \infty=\text{lim inf }(a_n)^\frac{1}{n}\]
		Case 3: \text{lim inf } $ \frac{a_{n+1}}{a_n}= a \in \mathbb{R} $\\
		So, $ \forall \epsilon > 0,  \exists M \in \mathbb{N}: n > M \implies \frac{a_{n+1}}{a_n} > a-\epsilon$
		\[ a_n>(a-\epsilon) a_{n-1}>(a-\epsilon)^2 a_{n-2}>...>(a-\epsilon)^{n-M} a_M \]
		\[ \implies (a_n)^ \frac{1}{n}>(a- \epsilon) (\frac{a_M}{a^M})^{ \frac{1}{n} }\]
	But, as $\lim_{n \to \infty} (\frac{a_M}{a^M})^ \frac{1}{n}=1 $,\\
	\[ (a_n)^ \frac{1}{n}>(a- \epsilon) (\frac{a_M}{a^M})^{ \frac{1}{n} }>(a-\epsilon)(1-\epsilon)=a-(1+a)\epsilon+ \epsilon^2 \]
	\[ \implies  (a_n)^ \frac{1}{n}>a-(1+a)\epsilon \]
		But as this holds for every $\epsilon >0$,
		\[ \text{lim inf }(a_n)^ \frac{1}{n} \geq a-0= \text{lim inf } \frac{a_{n+1}}{a_n}  \]

	\item $\text{lim sup }(a_n)^\frac{1}{n} \leq \text{lim sup }\frac{a_{n+1}}{a_n}$\\
		Case 1: \text{lim sup } $ \frac{a_{n+1}}{a_n}=\infty $\\
		As \text{lim sup }$(a_n)^ \frac{1}{n} \leq \infty = \text{lim sup }\frac{a_{n+1}}{a_n}$,done\\

		Case 2: \text{lim sup } $ \frac{a_{n+1}}{a_n}= -\infty $\\
		\[ \text{lim sup }  \frac{a_{n+1}}{a_n}= -\infty \implies \lim_{n \to \infty} \frac{a_{n+1}}{a_n} = -\infty\]
		But, as all $a_n$ are positive, so is their raito, and hence it cant be unbounded below. \\

		Case 3: \text{lim sup } $ \frac{a_{n+1}}{a_n}= a \in \mathbb{R} $\\
		So, $ \forall \epsilon > 0,  \exists M \in \mathbb{N}: n > M \implies \frac{a_{n+1}}{a_n} > a-\epsilon$
		\[ a_n>(a-\epsilon) a_{n-1}>(a-\epsilon)^2 a_{n-2}>...>(a-\epsilon)^{n-M} a_M \]
		\[ \implies (a_n)^ \frac{1}{n}>(a- \epsilon) (\frac{a_M}{a^M})^{ \frac{1}{n} } \implies \lim_{n \to \infty} (a_n)^ \frac{1}{n} > (a-\epsilon) \]
		Thus, $\text{lim inf }(a_n)^ \frac{1}{n} \geq a- \epsilon $, but as this holds for every $\epsilon$,
		\[ \text{lim inf }(a_n)^ \frac{1}{n} \geq a= \text{lim inf } \frac{a_{n+1}}{a_n}  \]

\end{enumerate}

\paragraph{Question 3}
\begin{enumerate}[label=\Roman*]
	\item
\end{enumerate}

\paragraph{Question 4}
\begin{enumerate}
	\item
	\item
	\item
	\item Some convergant sequences are:
		\begin{enumerate}
			\item	$a_n:=1$
		\end{enumerate}
	\item Some divergent sequences are: \begin{enumerate}
		\item	$a_n := n$
			\[\frac{  a_{n+1} }{  a_n  }= \frac{n+1}{n}=1+ \frac{1}{n} \]
			This tends to 1 [ for $ \epsilon= \frac{1}{n} , \text{ take } \delta = \frac{1}{n+1}   $]\\
			But the sequence diverges [ to $\infty$  ].
	\end{enumerate}
\end{enumerate}

\paragraph{Appendix}
\begin{enumerate}
	\item \text{also, make sure to show liminf $\leq$ limsup}
\end{enumerate}

\end{document}
