\documentclass[20pt,a4paper]{extarticle} %big fontsize
\usepackage[margin=0.7in]{geometry}
\usepackage{amsmath,amsthm,amssymb}
\usepackage{enumitem,amsfonts,extarrows,xcolor}

\begin{document}

\paragraph{Question 1}
\begin{eqnarray*}
	C \subseteq D \subseteq \mathbb{R};\\
	\big(f _n\big)_{n \in \mathbb{N}} \text{ is uniformly convergent on } C ;\\
	\forall i \in \mathbb{N} , \; f_i : D \xrightarrow{} \mathbb{R} \text{ is continous }\\
	\textbf{Show } \exists f \text{ such that } f_n \xlongrightarrow [ \text{ uniformly }]{ \overline{C} \cap D } f
	\text{ and } f \text{ is continous. }
\end{eqnarray*}
\begin{proof}

	Fix any $\epsilon > 0$. Need to show that
	\[ \exists K >0 \text{ s.t. } k,l\geq K \implies \forall x \in \overline{C} \cap D, \;
	|f_k(x) - f_l(x) |< \epsilon\]
	So, fix any $c \in \overline{C} \cap D$.

	As each $f_i$ is given continous on $D$, $\exists \delta(i,c) >0 $ such that
	\[ \forall y \in D, |c-y| \leq \delta \implies |f_i(c) - f_i(y)| < \epsilon/2 \]
	So, in particular, for any sequence in $C$, $(c_n) \rightarrow c$,
	\begin{align*}
		\exists N((c_n),\delta) \text{ such that }
		n\geq N & \implies |c_n - c| < \delta \\
		& \implies |f_i(c_n) - f_i(c)| < \epsilon/2
	\end{align*}
	Now, as $f_i \overset{C}{\rightrightarrows} f$, $\exists \beta$ such that
	for any $c \in C$,
	\begin{equation*}
		k \geq \beta \implies |f_k(c) - f(c)| < \epsilon/2
	\end{equation*}
	By triangle inequality,
	\[ |f_i(c) - f_j(c)| \leq |f_i(c) - f_i(c_\alpha)| + |f_i(c_\alpha) - f_j(c_\alpha))|+|f_j(c_\alpha) - f_j(c)|\]
	Thus,for $L > max\{ \beta , N \}$, both (1) and (2) will hold:
	\begin{align*}
		i \geq L \implies \epsilon/2 +\epsilon/2 &> |f_i(c_i) - f_i(c)|+ |f_i(c_i) - f(c)| \\
							 &\geq |f_i(c) - f(c) |
	\end{align*}
	Hence, the sequence uniformly converges to $f$ on $\overline{C}\cap D$.\\
	And as $(f_i)_{i \in \mathbb{N}}$ is a sequence of continous functions on $D$, that uniformly converges to $f$,
	$f$ is continous on $\overline{C}\cap D$.
\end{proof}
\textcolor{yellow}{Note: In case $D$ were bounded, $\overline{C} \cap D $ would be compact. Then, Heine-Cantor Theorem would give the existence of $f$ , and Uniform Limit Theorem would give it's continuity.   }

\paragraph{Question 2}
Prove that $\Sigma x^n(1-x)$ converges pointwise on $[0,1]$ but not uniformly.
While $\Sigma (-1)^nx^n(1-x) $ converges uniformly on $[0,1]$.
\begin{proof}
	As $x^n(1-x) = x^n - x^{n+1}$, the first sum telescopes:
	\[\sum_{i=1}^k x^n(1-x) = (x - x^2) + (x^2 -x^3) +... + (x^k -x^{k+1}) = x- x^{k+1} \]
	So, for $x=1$, every partial sum is 0, and for $0\leq x <1$,
	\[ \lim_{k \to \infty } \sum_{i=1}^k x^i(1-x) = \lim_{k \to \infty }(x - x^{k+1})=x\]
	Thus, the series converges pointwise on $[0,1]$. Suppose it also converges uniformly to $f$ .
	Then, as the $k^{th}$ partial sum is $x-x^{k+1}$, a polynomial, and hence continous on $[0,1]$,
	it's limit function, $f$ must be continous on $[0,1]$. But, $f$ is discontinous at 1 as
	\[ \lim_{x \to 1} f(x) = \lim_{x \to 1} x = 1 \neq 0 = f(1) \]
	\pagebreak
	The partial sums for the second series of functions,
	\[\sum_{i=1}^k (-x)^n(1-x) = -x +2[(-x)^2 + (-x)^3 + ... + (-x)^k] + (-x)^{k+1})\]
	So, for $x=1$, every partial sum is 0, and for $0\leq x <1$,
	\begin{align*}
		\lim_{k \to \infty } \sum_{i=1}^k (-x)^i(1-x) &= x + 2\lim_{k \to \infty }((-x)^{k+1}+\sum_{i=1}^k (-x)^i ) \\
							      &=x + 2 \lim_{k \to \infty }\frac{-x(1-(-x)^k)}{1+x}\\
							      &=x+\frac{-2x}{1+x}
	\end{align*}
	To show uniform convergence, going to use the Drichilet test:
	\begin{enumerate}[label=\Roman*]
		\item Take $b_n(x): = \frac{x^n}{2} = (\frac{x}{\sqrt[n]{2}})^n$ .Going to show that
			\begin{enumerate}[label=(\roman*)]
				\item $\forall x \in$[0,1], $b_n(x) \geq b_{n+1}(x)$ : \\
					As $b_{n+1}(x) - b_{n}(x) = \frac{x^{n+1}}{\sqrt[n+1]{2}} -
					\frac{x^n}{\sqrt[n]{2}}= \frac{x^n}{\sqrt[n]{2}}(\frac{x}{\sqrt{2}}-1) \leq 0  $
				\item $b_n \rightrightarrows 0(C)$ \\
					\begin{align*}
						\text{ Take } \delta > \frac{ln2}{ln(\epsilon +1)} &\implies ln(\epsilon+1) >
					\frac{1}{\delta}ln(2) \\
					& \implies \epsilon> \sqrt[\delta]{2}
					\end{align*}
					\begin{align*}
						|b_n(x) - b_m(x)| = |\frac{x}{\sqrt[n]{2}}- \frac{x}{\sqrt[m]{2}}|
						&=x(\frac{1}{\sqrt[n]{2}}- \frac{1}{\sqrt[m]{2}}) \\
						\leq (\frac{1}{\sqrt[n]{2}}- \frac{1}{\sqrt[m]{2}})
						= \frac {\sqrt[m]{2}-\sqrt[n]{2}}{\sqrt[n]{2}\sqrt[m]{2}}
						&\leq \sqrt[m]{2}-\sqrt[n]{2}\\
						& \leq \sqrt[\delta]{2} -1 \leq \epsilon
					\end{align*}
			\end{enumerate}
		\item Take $a_n(x):=2(-1)^n(1-x)$. So, \\
				$|(A_n(x))|=|\sum_{i=1}^n a_i(x)| = 2(1-x)|\sum_{i=1}^n (-1)^n|$ \\
				= $\begin{cases}
	0 & n \text{ is even } \\
	2(1-x) &  n \text{ is odd }
\end{cases}$ \\
$\therefore $ for any n,x $|A_n(x)| \leq 2$, $a_n$ is uniformly bounded.
\item So, Drichilet's test is applicable and
	\begin{align*}
		\Sigma b_n a_n &= \Sigma \frac{x^n}{2} \times 2 (-1)^n(1-x) \\
			       &= \Sigma (-1)^nx^n(1-x) \\
			       &\text{ uniformly converges on [0,1] }
	\end{align*}
	\end{enumerate}

\end{proof}

\pagebreak

\paragraph{Question 3}
\begin{eqnarray*}
	A \text{ is closed and bounded }; \\
	(f_n) \text{ is a sequence of continous functions on }A; \\
	(f_n) \xrightarrow{p.w.} f, \text{ with f continous on }A; \\
	\forall x \in A, f_n(x) \geq f_{n+1}(x), \text{ with } n \in \mathbb{N};
\end{eqnarray*}
\textbf{Prove} that $f_n \rightrightarrows f(A)$ \\

\paragraph{Question 4}
Construct a sequence of functions, $(f_n)$  on [0,1] such that
	\begin{enumerate}[label=(\alph*)]
		\item each $f_i$ is discontinous at every point of [0,1]; \\
			and
		\item $\exists f$, a continous function on [0,1] such that $f_n \rightrightarrows f$
	\end{enumerate}
	\begin{proof}
		Define $f_n(x)=
		\begin{cases}
			\frac{1}{n} & x \in \mathbb{Q} \\
			0 & x \not \in \mathbb{Q} \\
		\end{cases}$
			\begin{enumerate}[label=(\alph*)]
				\item This sequence is discontinous on [0,1] as :\begin{enumerate}[label=(\roman*)]
					\item For $q\in \mathbb{Q}$, take any sequence of irrational numbers,
						$(p_n)\rightarrow q$. So,
						\[ \lim_{k \to \infty} f_n(p_k)=0\neq \frac{1}{n} = f_n(q) \]
					\item For $p \not \in \mathbb{Q}$, take any sequence of rational numbers,
						$(q_n)\rightarrow p$. So,
						\[ \lim_{k \to \infty} f_n(q_k)=\frac{1}{n}\neq 0= f_n(q) \]
				\end{enumerate}
			\item The defined sequence uniformly converges to 0 as:\\
				Fix $\epsilon = \frac{1}{n}$, and choose  $\delta > n $,
				\[ m \geq \delta \implies |f_m(x)|=\frac{1}{m} \leq \frac{1}{\delta} < \frac{1}{n} = \epsilon \]
			\end{enumerate}

	\end{proof}

\paragraph{Question 5}
\textbf{Prove} If $\Sigma a_n$ is absolutely convergent then $\Sigma \frac{a_n x^n}{1+x^{2n}}$ converges uniformly on $\mathbb{R}.$

\end{document}
