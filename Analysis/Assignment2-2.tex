\documentclass[20pt,a4paper]{extarticle} %big fontsize
\usepackage[margin=0.7in]{geometry}
\usepackage{amsmath,amsthm,amssymb}
\usepackage{enumitem,amsfonts,extarrows}

\begin{document}

\paragraph{Question 1} %{{{ Question 1
\begin{eqnarray*}
	C \subseteq D \subseteq \mathbb{R};\\
	\big(f _n\big)_{n \in \mathbb{N}} \text{ is uniformly convergent on } C ;\\
	\forall i \in \mathbb{N} , \; f_i : D \xrightarrow{} \mathbb{R} \text{ is continous }\\
	\textbf{Show } \exists f \text{ such that } f_n \xlongrightarrow [ \text{ uniformly }]{ \overline{C} \cap D } f
	\text{ and } f \text{ is continous. }
\end{eqnarray*}
\begin{proof}
	Fix any $\epsilon > 0$. Need to show that
	\[ \exists K >0 \text{ s.t. } k\geq K \implies \forall x \in \overline{C} \cap D, \;
	|f_k(x) - f(x) |< \epsilon\]
	So, fix any $x \in \overline{C} \cap D$.

	As each $f_i$ is given continous on $D$, $\exists \delta >0 $ such that
	\[ \forall y \in D, |x-y| \leq \delta \implies |f_i(x) - f_i(y)| < \epsilon/2 \]
	So, in particular, for any sequence in $C$, $(c_n) \rightarrow x$,
	\begin{align*}\label{eq:1}
		\exists N \text{ such that } n\geq N & \implies |c_n - x| < \delta \\
						     & \implies |f_i(c_n) - f_i(x)| < \epsilon/2
	\end{align*}
	Now, as $f_i \overset{C}{\rightrightarrows} f$, $\exists \beta$ such that for any $c \in C$,
	\begin{equation*}\label{eq:2}
		k \geq \beta \implies |f_k(c) - f(c)| < \epsilon/2
	\end{equation*}
	By triangle inequality,
	\[ |f_i(x) - f(x)| \leq |f_i(x) - f_i(c_i)| + |f_i(c_i) - f(x)|\]
	Thus,for $L > max\{ \beta , N \}$, both (1) and (2) will hold:
	\begin{align*}
		i \geq L \implies \epsilon/2 +\epsilon/2 &> |f_i(c_i) - f_i(x)|+ |f_i(c_i) - f(x)| \\
							 &\geq |f_i(x) - f(x) |
	\end{align*}
	Hence, the sequence uniformly converges to $f$ on $\overline{C}\cap D$.\\
	And as $(f_i)_{i \in \mathbb{N}}$ is a sequence of continous functions on $D$, that uniformly converges to $f$,
	$f$ is continous on $\overline{C}\cap D$.
\end{proof}

\paragraph{Question 2}
Prove that $\Sigma x^n(1-x)$ converges pointwise on $[0,1]$ but not uniformly.
While $\Sigma (-1)^nx^n(1-x) $ converges uniformly on $[0,1]$.
\begin{proof}
	As $x^n(1-x) = x^n - x^{n+1}$, the first sum telescopes:
	\[\sum_{i=1}^k x^n(1-x) = (x - x^2) + (x^2 -x^3) +... + (x^k -x^{k+1}) = x- x^{k+1} \]
	So, for $x=1$, every partial sum is 0, and for $0\leq x <1$,
	\[ \lim_{k \to \infty } \sum_{i=1}^k x^i(1-x) = \lim_{k \to \infty }(x - x^{k+1})=x\]
	Thus, the series converges pointwise on $[0,1]$. Suppose it also converges uniformly to $f$ .
	Then, as the $k^{th}$ partial sum is $x-x^{k+1}$, a polynomial, and hence continous on $[0,1]$,
	it's limit function, $f$ must be continous on $[0,1]$. But, $f$ is discontinous at 1 as
	\[ \lim_{x \to 1} f(x) = \lim_{x \to 1} x = 1 \neq 0 = f(1) \]
	\pagebreak
	The partial sums for the second series of functions,
	\[\sum_{i=1}^k (-x)^n(1-x) = -x +2[(-x)^2 + (-x)^3 + ... + (-x)^k] + (-x)^{k+1})\]
	So, for $x=1$, every partial sum is 0, and for $0\leq x <1$,
	\begin{align*}
		\lim_{k \to \infty } \sum_{i=1}^k (-x)^i(1-x) &= x + 2\lim_{k \to \infty }((-x)^{k+1}+\sum_{i=1}^k (-x)^i ) \\
							      &=x + 2 \lim_{k \to \infty }\frac{-x(1-(-x)^k)}{1+x}\\
							      &=x+\frac{-2x}{1+x}
	\end{align*}
	To show uniform convergence, try M-test?

\end{proof}

\pagebreak

\paragraph{Question 3}
\[\text{ $a_n$ \& $b_n$ are bounded, non-negative sequences; $a_n \rightarrow a >0 $   } \]
As $a_n, b_n $ are bounded sequences, so is $a_n b_n$.\\
Hence, L:=lim sup $a_n b_n $ and R:=lim sup $b_n$ are real numbers



\end{document}
