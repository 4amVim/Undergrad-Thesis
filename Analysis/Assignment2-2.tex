\documentclass[20pt,a4paper]{extarticle} %big fontsize
\usepackage[margin=0.7in]{geometry}
\usepackage{amsmath,amsthm,amssymb}
\usepackage{enumitem,amsfonts,extarrows,xcolor}

\begin{document}

\paragraph{Question 1}
\begin{eqnarray*}
	C \subseteq D \subseteq \mathbb{R};\\
	\big(f _n\big)_{n \in \mathbb{N}} \text{ is uniformly convergent on } C ;\\
	\forall i \in \mathbb{N} , \; f_i : D \xrightarrow{} \mathbb{R} \text{ is continous }\\
	\textbf{Show } \exists f \text{ such that } f_n \xlongrightarrow [ \text{ uniformly }]{ \overline{C} \cap D } f
	\text{ and } f \text{ is continous. }
\end{eqnarray*}
\begin{proof}
	Fix an $\epsilon > 0$ so, by uniform continuity of $f$,
	\[ \exists M(\epsilon): p,m \geq M \implies \forall x \in C, |f_n(x) - f_m(x)| < \frac{\epsilon}{3} \]
	Now, let $k \in \overline{C}$ so, $\exists (x_n) \subseteq C $ such that $x_n \rightarrow x$. \\
	By continuity of each $f_i$,
	\[ \exists \delta(i,\epsilon) \text{ s.t. } |x-y|\leq \delta \implies |f_i(x)-f_i(y)|<\frac{\epsilon}{3} \]
	So, there's $K(p,m,\epsilon,(x_n))$ such that
	\[ i\geq K \implies |f_p(x_i)-f_p(k)|< \frac{\epsilon}{3} \text{ and } |f_m(x_i)-f_m(k)|< \frac{\epsilon}{3} \]
	Thus, by triangle inequality,
	\begin{align*}
		2 \times \frac{\epsilon}{3} = \epsilon & > |f_p(x_i)-f_p(k)| + |f_m(x_i)-f_m(k)| \\
					    & \geq |f_p(x_i)-f_m(x_i)+f_p(k)-f_m(k)| \\
					    & \geq ||f_p(x_i)-f_m(x_i)|-|f_m(k)-f_p(k)||
	\end{align*}
	Now, as $|f_p(x_i)-f_m(x_i)|< \frac{\epsilon}{3}$, \\
	If $|f_m(k)-f_p(k)|< \frac{\epsilon}{3}$ then it's shown that $M$ works for any $k$. Else, \\
	\begin{align*}
	|f_m(k)-f_p(k)| \geq \frac{\epsilon}{2} & \implies \frac{2\epsilon}{3} > |f_m(k)-f_p(k)|-\frac{\epsilon}{3} \\
						& \implies \epsilon > |f_m(k)-f_p(k)|
\end{align*}
\end{proof}

	\pagebreak
\paragraph{Question 3}
\begin{eqnarray*}
	A \text{ is closed and bounded }; \\
	(f_n) \text{ is a sequence of continous functions on }A; \\
	(f_n) \xrightarrow{p.w.} f, \text{ with f continous on }A; \\
	\forall x \in A, f_n(x) \geq f_{n+1}(x), \text{ with } n \in \mathbb{N};
\end{eqnarray*}
\textbf{Prove} that $f_n \rightrightarrows f(A)$ \\

\paragraph{Question 5}
\textbf{Prove} If $\Sigma a_n$ is absolutely convergent then $\Sigma \frac{a_n x^n}{1+x^{2n}}$ converges uniformly on $\mathbb{R}.$

\end{document}
