\documentclass[18pt,a4paper]{article}
\usepackage[margin=0.7in]{geometry}
\usepackage{amsmath,amsthm,amssymb,tikz-cd}
\usepackage{enumitem,amsfonts,extarrows,xcolor}

\newtheorem{theorem}{Theorem}[section]

\theoremstyle{definition}
\newtheorem{definition}{Definiton}[section]
\newtheorem{lemma}{Lemma}[definition]
\newtheorem{prop}{Proposition}[definition]
\newtheorem{proop}{Proposition}[section]
\newtheorem{lmma}{Lemma}[section]

\tikzset{commutative diagrams/.cd,mysymbol/.style={start anchor=center,end anchor=center,draw=none}}
\newcommand\cen[2][\leq]{\arrow[mysymbol]{#2}[description]{#1}}

\makeatletter
\newcommand{\carrow}{}% just in case
\DeclareRobustCommand{\carrow}{%
	\mathrel{\vphantom{\rightarrow}\mathpalette\circle@arrow\relax}%
}
\newcommand{\circle@arrow}[2]{%
	\m@th
	\ooalign{%
		\hidewidth$#1\circ\mkern1mu$\hidewidth\cr
	$#1\longrightarrow$\cr}%
}

\begin{document}
\section{Yoneda Lemma} %{{{ Yoneda Lemma		Section

\begin{lmma} %{{{ H_A	Definition
	For a locally small category $\mathcal{A}$, fixing an object $A \in \mathcal{A} $ gives
	a functor, $H_A: \mathcal{A} ^{op} \rightarrow Set$ defined as:
	\begin{enumerate}[label=(\roman*)]
		\item For any object $B \in \mathcal{A} $ , $H_A(B):=Hom_{Set}(B,A)$
		\item For any morphism, $g : X \rightarrow Y $ in $\mathcal{A}$,
			\[H_A(g): \mathcal{A} (Y,A) \rightarrow \mathcal{A}(X,A)
				\text{ given by } p \mapsto p \circ g\]
	\end{enumerate}
\end{lmma}
\begin{lmma}
	For a natural transformation $\alpha$, it's a natural isomorphism iff
	each of its components is an isomorphism.
\end{lmma}
\begin{lmma}
	naturality in two variables simultaneously is equivalent to
	naturalityin each variable independently(1.3.29 pg 39)
\end{lmma}
\begin{definition} %{{{Category A op		Definition
For any category $\mathcal{A} $, it's opposite category is a category having the same objects as
$\mathcal{A}$, with a morphism from object $A$ to $B$, that is $f \in \mathcal{A}^{op} (A,B)$
only if there is a morphism $g \in \mathcal{A}(B,A)$.
\end{definition}
\begin{definition} %{{{Presheaf Category 	Definition
	The category of presheaves on $\mathcal{A} $, denoted by $[\mathcal{A} ^{op},Set]$
	is defined to have functors from $\mathcal{A} ^{op}$ to $Set$ as objects,
	and natural transformations between them as morphisms.
\end{definition}
\begin{theorem}{\textbf{Yoneda Lemma}} %{{{ Yoneda		Theorem
	If $\mathcal{A} $ is a locally small category then, for any object $A \in \mathcal{A} $
	and $X \in [ \mathcal{A}^{op},Set]$,\\ there's exists an isomorphism,
	\[ [ \mathcal{A} ^{op},Set ](H_A,X)\cong X(A) \text{such that it is natural in } A \text{ and } X.\]
\end{theorem}
\textbf{Explaination:} \\
First , fix any category, $\mathcal{A} $ . Now ,choose two things (independent of each other):
\begin{enumerate}[label=\roman*]
	\item an object, $A$ from the category $ \mathcal{A}= \mathcal{A} ^{op} $
	\item an object, $X \in [ \mathcal{A}^{op},Set] $, the presheaf category \\
		i.e. a functor $X: A^{op} \rightarrow Set$
\end{enumerate}
Here, $[ \mathcal{A}^{op},Set](	H_A,X) $ denotes morphisms
$H_A \rightarrow X$ in $[ \mathcal{A}^{op},Set] $,
i.e. natural transformations, $\alpha :$
\begin{tikzcd}[row sep=large, column sep=huge]
	A^{op} \arrow[r,bend left=50, "H_A"{name=U, below}]
	\arrow[r, bend right = 50, "X"{name=D, above}]
& Set
\arrow[Rightarrow, from=U , to=D, "\alpha"]
\end{tikzcd}

The key idea is that the choice of $A$ and $X$ completely determines all possible maps
(i.e. natural transformations) from functor $H_A \text{ to functor} X$. Moreover, that this isomorphism
is \textit{natural} in A and X. \\
Meaning that $[ \mathcal{A}^{op},Set](H_A,X) \text{ and } X(A)$ are
\textit{functorial} in \textit{both} $A \text{ and } X$

\paragraph{Notation:} \begin{itemize}
	\item We denote the category of presheaves on $\mathcal{A}$ by $\mathcal{C}$.
	\item For the map $\string ^ $, that is $\hat a = b$, we use $a \xrightarrow{ \string ^ } b$
	\item For the map $\string ~ $, that is $\tilde a = b$ we use $a \xrightarrow{\string ~} b$
\end{itemize}

To prove the theorem, first, we show that $[ \mathcal{A} ^{op}, Set](H_A,X)$ is isomorphic to $X(A)$.
And then that this isomorphism is natural in $X$ and $A$.

\begin{proof}Let a locally small category $\mathcal{A} $ be given.
	Fix an object $A \in \mathcal{A} $ and a presheaf $X$.

	\paragraph{Showing isomorphism between $[\mathcal{A}^{op},Set](H_A,X)$ and $X(A)$}\mbox{} \\

	Define $\string ^ :\mathcal{C}(H_A,X) \to X(A) $
	for any $\alpha:H_A \to X,$ as  $\hat{\alpha}:= \alpha_A(1_A)$.\\

	Define $\string ~ : X(A) \to [\mathcal{A}^{op}, Set](H_A,X)$
	for any $ x \in X(A)$ as the natural transformation $\tilde{x} : H_A \to X$ whose
	K-component is the function mapping each morphsim $p \in \mathcal{A}(K,A)$
	to $\Big(X(p)\Big)(x)$. That is, $\tilde{x}_K (p):=\Big(X(p)\Big)(x)$.

	We are going to show that $\tilde{x}$ is a natural transformation.
	Fix objects $K,L \in \mathcal{A} $ and morphism $q \in \mathcal{A}^{op}(K,L)$.
	$\text{Need to show that the square }
	\begin{tikzcd}
		H_A(K) \arrow[swap]{d}{\tilde{x}_K} \arrow{r}{H_A(q)}
		& H_A(L) \arrow{d}{\tilde{x}_L}\\
		X(K) \arrow[swap]{r}{X(q)}
		& X(L)
	\end{tikzcd} \text{, that is }
	\begin{tikzcd}
		\mathcal{A}(K,A) \arrow[swap]{d}{\tilde{x}_K} \arrow{r}{ - \circ q}
		& \mathcal{A} (L,A) \arrow{d}{\tilde{x}_L}\\
		X(K) \arrow[swap]{r}{X(q)}
		& X(L)
	\end{tikzcd}\text{ commutes }$.\\

	So, for any $f:K\to A$, need that $\tilde{x}_L(f \circ q)= X(q) \circ \tilde{x}_K(f)$. Using
	the definition of $\tilde{x} $ gives the following.
	\begin{align*}
	LHS &=\tilde{x}_L(f \circ q ) =\Big( X(f \circ q)\Big)(x) \\
	RHS &=X(q) \circ \tilde{x}_K(f) =\Big(X(q)\Big) \big(X(f)(x)\big)=\Big(X(q) \circ X(f)\Big) (x)
	\end{align*}
	And as $X$ is a contravariant functor, $X(f \circ q)= X(q) \circ X(f)$, giving that LHS=RHS.

	Now going to show that $\string ^$ and $\string ~$ define an isomorphism.
	Need to show that $\string ^ $ and $\string ~ $ are mutually inverse.
	\begin{enumerate}[label=(\roman*)]
		\item  For any $x \in X(A)$,
		$\hat{\tilde{x}}=\tilde{x}_A (1_A)=\Big(X(1_A) \Big) (x)=1_{X(A)}(x)=x$.
	\item For any $\alpha \in \mathcal{C} (H_A,X)$ , need to show that $\tilde{\hat{\alpha}}=\alpha$.
		So, it's required that each of their component are equal.
		As both $\tilde{\hat{\alpha}}$ and $\alpha$ are natural transformations
	between functors that go to the category \textit{Set}, each of the components is a function.
	So, need to show that for any $f \in \mathcal{A} (K,A)=H_A(K)$,
	$\Big(\tilde{\hat{\alpha}}\Big)_K(f)$=$\alpha_K(f)$.
	Using first the definition of $ \string ~$ and then that of $\hat{\alpha}$ gives:
	\begin{align}
	LHS=\tilde{\hat{\alpha}}_B(f)=\Big(X(f)\Big)(\hat{\alpha})=\Big(X(f)\Big)(\alpha_A(1_A))
	\end{align}
	And as $f:k \to A$, we also have the following.
	\begin{equation}
		RHS=\alpha_K(f)= \alpha_K(1_A \circ f)
	\end{equation}
	Because $\alpha$ is a natural transformation, the square following square commutes for $1_A$.
	\[\begin{tikzcd}
		\mathcal{A} (A,A) \arrow[swap]{d}{\alpha_A} \arrow{r}{- \circ f}
	& \mathcal{A} (K,A) \arrow{d}{\alpha_K}\\
	X(A) \arrow[swap]{r}{X(f)}
	& X(K)
	\end{tikzcd}\]
	This gives that $\alpha_K(1_A \circ f)=\Big(X(f) \Big) \big( \alpha_A (1_A) \big)$.
	Hence, we have from (1) and (2), we get that $RHS=LHS$.
	\end{enumerate}
	\paragraph{Showing naturality of this isomorphism }\mbox{} \\



\end{proof}
%}}}
\section{Cayley's Theorem}%
Informally, Yoneda Lemma gives us a stable proxy to study any presheaf on a locally small category.
In group theory, Cayley's theorem is a result that similarly allows us to study any group by instead
studying a subgroup of some symmetric group.

\begin{definition}[Symmetric group on a set] %{{{ Symmetric group on a set	Definition
\end{definition}
\begin{theorem}{\textbf{Cayley's Theorem}} %{{{ Cayley's Theorem		Theorem
	Every group, (G,.) is isomorphic to a subgroup of symmetric group on G.
\end{theorem}

\section{Embedding of a category in Presheaf category}%{{{ Embedding		Section

\begin{definition}[Embedding of a category] %{{{ Embedding of a category		Definition
	A category, $\mathcal{A}$  is said to be embedded in a category, $\mathcal{B}$ if there exists a
	functor $F: \mathcal{A} \to \mathcal{B} $ such that $F$ is full, faithful and injective (on objects).
\end{definition} %}}}





\end{document}
