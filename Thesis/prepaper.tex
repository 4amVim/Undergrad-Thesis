\documentclass[18pt,a4paper]{extarticle}
\usepackage[margin=0.7in]{geometry}
\usepackage{amsmath,amsthm,amssymb,tikz-cd}
\usepackage{enumitem,amsfonts,extarrows,xcolor}

\theoremstyle{theorem}
\newtheorem{theorem}{Theorem}[section]

\theoremstyle{definition}
\newtheorem{definition}{Definiton}[section]

\theoremstyle{lemma}
\newtheorem{lemma}{Lemma}[section]

\begin{document}
\section{Yoneda Lemma} %{{{ Yoneda Lemma		Section
\begin{lemma}[$H_A$ or $\mathcal{A} (\_,A)$ ] %{{{ H_A	Definition
	For any category $\mathcal{A}$, fixing an object, $A \in \mathcal{A} $,\\
	there's a functor, $H_A: \mathcal{A} ^{op} \rightarrow Set$ defined as:
	\begin{enumerate}[label=\roman*]
		\item For object $B \in \mathcal{A} $ , $F(B):=Hom(B,A)$
		\item For any morphism in $\mathcal{A} $ , $g : X \rightarrow Y $,\\
			 \[H_A(g): \mathcal{A} (Y,A) \rightarrow \mathcal{A}(X,A)
			 \text{ , as, } \forall p \in \mathcal{A}(Y,A) \text{ , }
	 p \mapsto p \circ g \text{ i.e. } \Big( H_A(g) \Big) (p) := p \circ g \]
	\end{enumerate}
\end{lemma}
\begin{theorem}{\textbf{Yoneda}} %{{{ Yoneda		Theorem
	If $\mathcal{A} $ is a locally small category, for any object $A \in \mathcal{A} $
	and $X \in [ \mathcal{A}^{op},Set]$,\\ there's exists a natural isomorphism:
	\[ [ \mathcal{A} ^{op},Set ](H_A,X) \cong X(A) \text{ naturally in } A \in \mathcal{A} \]
\end{theorem}
\textbf{Explaination:} \\
First , fix any category, $\mathcal{A} $ . Now ,choose two things (independent of each other):
\begin{enumerate}[label=\roman*]
	\item an object, $A$ from the category $ \mathcal{A}= \mathcal{A} ^{op} $
	\item an object, $X \in [ \mathcal{A}^{op},Set] $, the presheaf category \\
		i.e. a functor $X: A^{op} \rightarrow Set$
\end{enumerate}
Here, $[ \mathcal{A}^{op},Set](	H_A,X) $ denotes morphisms
$H_A \rightarrow X$ in $[ \mathcal{A}^{op},Set] $,
i.e. natural transformations, $\alpha :$
\begin{tikzcd}[row sep=large, column sep=huge]
	A^{op} \arrow[r,bend left=50, "H_A"{name=U, below}]
	\arrow[r, bend right = 50, "X"{name=D, above}]
& Set
\arrow[Rightarrow, from=U , to=D, "\alpha"]
\end{tikzcd}
Each of these natural transformations is a collection of, morphisms in $Set$,
hence each of their components is exactly a function. i.e.
$\forall \alpha \in [ \mathcal{A}^{op},Set](H_A,X), \forall K \in \mathcal{A}, \alpha_K \text{ is a function :} H_A(K) \to X(K)$\\

$X(A)$ is precisely a set, because $X(A)$ is the image of (our chosen object,) $A$, under (our chosen functor,) X. \\

The key idea is that the choice of $A$ and $X$ completely determines all possible maps
(i.e. natural transformations) from $H_A \text{ to } X$.

Moreover, this isomorphism is \textit{natural} in A and X. \\
Meaning that $[ \mathcal{A}^{op},Set](H_A,X) \text{ and } X(A)$ are
\textit{functorial} in \textit{both} $A \text{ and } X$

\paragraph{Notation:} \begin{itemize}
\item Denoting the category of all presheaves on $\mathcal{A} $ by $\mathcal{C}$,
	i.e. $\mathcal{C}:=[ \mathcal{A} ^{op}, Set] $
\item using $\string ^ $ as a map i.e. $\hat a = b$ stands for $a \xrightarrow{\string ^} b$
\item using $\string ~ $ as a map i.e. $\hat a = b$ stands for $a \xrightarrow{\string ~} b$
\end{itemize}

To prove the theorem, first,
going to show that $[ \mathcal{A} ^{op}, Set](H_A,X)$ is isomorphic to $X(A)$. And then that
this isomorphism is natural.

\begin{proof}Let a locally small category,  $\mathcal{A} $ be given. \\
	Let $A \in \mathcal{A} $ and $X \in [ \mathcal{A} ^{op}, Set]$\\

	\paragraph{1. Defining $\phi$ and $\psi$}
	Define $\phi$ (on natural transformations) as the A-component (of that natural transformation)
	at the identity of A. i.e. for $\alpha \in \mathcal{C}(H_A,X) ,
	\phi(\alpha):= \alpha_A(1_A) $ \\

	Define $\psi$ on an object, $ x \in X(A), $
	by defining it's $K$-component for any $K \in \mathcal{A} $:
	\[({\psi(x)})_K : H_A(K) \to X(K)
	\text{ as, for each } p\in Hom_{\mathcal{A} }(K,A), p \mapsto \Big(X(p)\Big)(x) \]
	That is to say that the K-component maps any arrow $p:K \to A$
	to the image of x under the map X(p).

\end{proof}
%}}}

\section{Cayley's Theorem}%

\section{Embedding of a category in Presheaf category}%{{{ Embedding		Section

\begin{definition}[Embedding of a category] %{{{ Embedding of a category		Definition
A category, $\mathcal{A}$  is said to be embedded in a category, $\mathcal{B}$ if there exists a
functor $F: \mathcal{A} \to \mathcal{B} $ such that $F$ is full, faithful and injective (on objects).
\end{definition} %}}}

\begin{proof}
	Prove that $H_A$ is indeed a functor

\end{proof}




\section{Quasi-Paper}%
\label{sec:quasi_paper}



\end{document}
