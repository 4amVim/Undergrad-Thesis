\documentclass[18pt,a4paper]{extarticle}
\usepackage[margin=0.7in]{geometry}
\usepackage{amsmath,amsthm,amssymb,tikz-cd}
\usepackage{enumitem,amsfonts,extarrows,xcolor}

\theoremstyle{definition}
\newtheorem{theorem}{Theorem}[section]

\theoremstyle{definition}
\newtheorem{definition}{Definiton}[section]

\begin{document}
\begin{theorem}{\textbf{Yoneda}} %{{{ Yoneda		Theorem
	If $\mathcal{A} $ be a locally small category, then, \\
	\[ [ \mathcal{A} ^{op},Set ](H_A,X) \cong X(A) \text{ naturally in } A \in \mathcal{A} \text{ and }
	X \in [ \mathcal{A}^{op},Set]\]
\end{theorem}
\textbf{Explaination:} \\
First of all, we fix any category, $\mathcal{A} $ . Now we choose two things (independent of each other):
\begin{enumerate}[label=\roman*]
	\item an object, $A$ from the category $ \mathcal{A}= \mathcal{A} ^{op} $
	\item an object, $X$ of the category $[ \mathcal{A}^{op},Set] $
		which is precisely a functor $X: A^{op} \rightarrow Set$
\end{enumerate}
Here, $[ \mathcal{A}^{op},Set](	H_A,X) $ denotes arrows $H_A \rightarrow X$ in $[ \mathcal{A}^{op},Set] $
i.e. natural transformations, $\alpha :$
\begin{tikzcd}[row sep=large, column sep=huge]
	A^{op} \arrow[r,bend left=50, "H_A"{name=U, below}]
	\arrow[r, bend right = 50, "X"{name=D, above}]
& Set
\arrow[Rightarrow, from=U , to=D, "\alpha"]
\end{tikzcd}
Each of these natural transformations is a collection of maps in $Set$, hence each of their
components is exactly a function. i.e.
$\forall \alpha \in [ \mathcal{A}^{op},Set](H_A,X), \forall K \in \mathcal{A}, \alpha_K \text{ is a function :} H_A(K) \to X(K)$\\

$X(A)$ is precisely a set, because $X(A)$ is the image of (our chosen object,) $A$, under (our chosen functor,) X. \\

So, the idea is that our choice of $A \text{ and } X$ completely determines all possible maps
(i.e. natural transformations) from $H_A \text{ to } X$. This answers out big question of "what are all the maps
$H_A  \rightarrow X$ " or "how does $H_A$ see other presheaves on $\mathcal{A} $ ".\\

\textit{The theorem says not just that the two are isomorphic, but that they're \textbf{naturally} isomorphic.}

This means that $[ \mathcal{A}^{op},Set](H_A,X) \text{ and } X(A)$ are
\textit{functorial} in \textit{both} $A \text{ and } X$\\

Also,
So, the aforementioned collection of natural transformations also must be a set:
As $\mathcal{A} $ is locally small, for each choice of $K \in \mathcal{A} $ , $H_A(K)=Hom_{\mathcal{A}}(K,A)$ is a set.
Thus, as $\alpha_K $ is a function,and hence a relation,\\ it's a subset of $Hom_{\mathcal{A} }(K,A)\times X(A)$, which is a set as it's a cartesian product of sets.\\
Thus, $K$-component of every natural transformation is a set. i.e. $\forall \alpha \in [ \mathcal{A}^{op},Set](H_A,X), \forall K \in \mathcal{A}, \alpha_K \text{ is a set}$ \\

Now, the question is wether the set of all natural transformations is a set

\begin{proof}Let a locally small category,  $\mathcal{A} $ be given. \\

	Denoting the category of all presheaves on $\mathcal{A} $ by $\mathcal{C}$,
	i.e. $\mathcal{C}:= [ \mathcal{A} ^{op}, Set] $  \\

	Now, fix any object, $A \in \mathcal{A} $, and any object, $X \in \mathcal{C}$.   \\
	Need to show that $\mathcal{C}(H_A,X) $ is naturally isomorphic to $X(A)$.\\
	Thus, need two mutually inverse natural transformations, such that :
	\begin{tikzcd}[row sep=3em, column sep=3em]
		\mathcal{C}(H_A,X) \arrow[r,"\psi"]
		& \arrow[l, shift right,"\phi"]  X(A)
	\end{tikzcd}\\
	\textit{\textbf{Need to show that the RHS is a set, and then,} } As both the RHS and LHS above
	are sets, the natural transformations between them are maps in $Set$ i.e. functions.


	Going to show the following in order,
	\begin{enumerate}
		\item Define $\phi \text{ and } \psi$
			\item Show that $\phi \text{ and } \psi$ are mutually inverse
		\item Show naturality of $\phi$ in $X$
		\item Show naturality of $\phi$ in $A$
		\item Show naturality of $\psi$ in $X$
		\item Show naturality of $\phi$ in $A$
	\end{enumerate}

	\paragraph{1. Defining $\phi$ and $\psi$}
	Define $\phi$ (on natural transformations) as the A-component (of that natural transformation)
	at the identity of A. i.e. for $\alpha \in \mathcal{C}(H_A,X) ,
	\phi(\alpha):= \alpha_A(1_A) $ \\

	Define $\psi$ on an object, $ x \in X(A), $ by defining it's $K$-component for any $K \in \mathcal{A} $:
	\[({\psi(x)})_K : H_A(K) \to X(K)
	\text{ as, for each } p\in Hom_{\mathcal{A} }(K,A), p \mapsto \Big(X(p)\Big)(x) \]
	That is to say that the K-component maps any arrow $p:K \to A$ to the image of x under the map X(p).

	\paragraph {2. Showing inverses} \textbf{Firstly},to show $\psi \circ \phi (\alpha) = \alpha $,
	for any natural transformation $\alpha: H_A \to X$\\

	\[ \psi \circ \phi (\alpha) = \psi \Big(\alpha_A(1_A) \Big) =  \]

	Secondly, for every $x \in X(A)$, need $\psi \circ \phi (x) = x$

	\paragraph{3. Naturality of $\phi$ in $X$}
	Need to show that the following square commutes

	i.e.
	\paragraph{4. Naturality of $\phi$ in $A$}
	Need to show that the following square commutes

	i.e.
	\paragraph{5. Naturality of $\psi$ in $X$}
	Need to show that the following square commutes

	i.e.
	\paragraph{6. Naturality of $\psi$ in $A$}
	Need to show that the following square commutes

	i.e.

\end{proof}








\end{document}
