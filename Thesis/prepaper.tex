\documentclass[18pt,a4paper]{article}
\usepackage[margin=0.7in]{geometry}
\usepackage{amsmath,amsthm,amssymb,tikz-cd}
\usepackage{enumitem,amsfonts,extarrows,xcolor}

\newtheorem{theorem}{Theorem}[section]

\theoremstyle{definition}
\newtheorem{definition}{Definiton}[section]
\newtheorem{lemma}{Lemma}[definition]
\newtheorem{prop}{Proposition}[definition]
\newtheorem{proop}{Proposition}[section]
\newtheorem{lmma}{Lemma}[section]

\tikzset{commutative diagrams/.cd,mysymbol/.style={start anchor=center,end anchor=center,draw=none}}
\newcommand\cen[2][\leq]{\arrow[mysymbol]{#2}[description]{#1}}

\makeatletter
\newcommand{\carrow}{}% just in case
\DeclareRobustCommand{\carrow}{%
	\mathrel{\vphantom{\rightarrow}\mathpalette\circle@arrow\relax}%
}
\newcommand{\circle@arrow}[2]{%
	\m@th
	\ooalign{%
		\hidewidth$#1\circ\mkern1mu$\hidewidth\cr
	$#1\longrightarrow$\cr}%
}

\begin{document}
\section{Yoneda Lemma} %{{{ Yoneda Lemma		Section

\begin{lmma}[$H_A$ or $\mathcal{A} (\_,A)$ ] %{{{ H_A	Definition
	For any category $\mathcal{A}$, fixing an object, $A \in \mathcal{A} $,\\
	there's a functor, $H_A: \mathcal{A} ^{op} \rightarrow Set$ defined as:
	\begin{enumerate}[label=\roman*]
		\item For object $B \in \mathcal{A} $ , $F(B):=Hom(B,A)$
		\item For any morphism in $\mathcal{A} $ , $g : X \rightarrow Y $,\\
			\[H_A(g): \mathcal{A} (Y,A) \rightarrow \mathcal{A}(X,A)
				\text{ , as, } \forall p \in \mathcal{A}(Y,A) \text{ , }
			p \mapsto p \circ g \text{ i.e. } \Big( H_A(g) \Big) (p) := p \circ g \]
	\end{enumerate}
\end{lmma}
\begin{proof}
	Prove that $H_A$ is indeed a functor

\end{proof}

\begin{lmma}
	For a natural transformation $\alpha$, it's a natural isomorphism iff
	each of its components is an isomorphism.
\end{lmma}

\begin{lmma}
	naturality in two variables simultaneously is equivalent to
	naturalityin each variable independently(1.3.29 pg 39)
\end{lmma}

\begin{theorem}{\textbf{Yoneda}} %{{{ Yoneda		Theorem
	If $\mathcal{A} $ is a locally small category, for any object $A \in \mathcal{A} $
	and $X \in [ \mathcal{A}^{op},Set]$,\\ there's exists a natural isomorphism:
	\[ [ \mathcal{A} ^{op},Set ](H_A,X) \cong X(A) \text{ naturally in } A \in \mathcal{A} \]
\end{theorem}
\textbf{Explaination:} \\
First , fix any category, $\mathcal{A} $ . Now ,choose two things (independent of each other):
\begin{enumerate}[label=\roman*]
	\item an object, $A$ from the category $ \mathcal{A}= \mathcal{A} ^{op} $
	\item an object, $X \in [ \mathcal{A}^{op},Set] $, the presheaf category \\
		i.e. a functor $X: A^{op} \rightarrow Set$
\end{enumerate}
Here, $[ \mathcal{A}^{op},Set](	H_A,X) $ denotes morphisms
$H_A \rightarrow X$ in $[ \mathcal{A}^{op},Set] $,
i.e. natural transformations, $\alpha :$
\begin{tikzcd}[row sep=large, column sep=huge]
	A^{op} \arrow[r,bend left=50, "H_A"{name=U, below}]
	\arrow[r, bend right = 50, "X"{name=D, above}]
& Set
\arrow[Rightarrow, from=U , to=D, "\alpha"]
\end{tikzcd}
Each of these natural transformations is a collection of, morphisms in $Set$,
hence each of their components is exactly a function. i.e.
$\forall \alpha \in [ \mathcal{A}^{op},Set](H_A,X), \forall K \in \mathcal{A}, \alpha_K \text{ is a function :} H_A(K) \to X(K)$\\

$X(A)$ is precisely a set, because $X(A)$ is the image of (our chosen object,) $A$, under (our chosen functor,) X. \\

The key idea is that the choice of $A$ and $X$ completely determines all possible maps
(i.e. natural transformations) from functor $H_A \text{ to functor} X$.

Moreover, that this isomorphism is \textit{natural} in A and X. \\
Meaning that $[ \mathcal{A}^{op},Set](H_A,X) \text{ and } X(A)$ are
\textit{functorial} in \textit{both} $A \text{ and } X$

\paragraph{Notation:} \begin{itemize}
	\item Denoting the category of presheaves on $\mathcal{A} $ by $\mathcal{C}$,
		i.e. $\mathcal{C}:=[ \mathcal{A} ^{op}, Set] $
	\item using $\string ^ $ as a map i.e. $\hat a = b$ stands for $a \xrightarrow{\string ^} b$
	\item using $\string ~ $ as a map i.e. $\hat a = b$ stands for $a \xrightarrow{\string ~} b$
\end{itemize}

To prove the theorem, first,
going to show that $[ \mathcal{A} ^{op}, Set](H_A,X)$ is isomorphic to $X(A)$. And then that
this isomorphism is natural.

\begin{proof}Let a locally small category,  $\mathcal{A} $ be given. \\
	Now, fix any object $A \in \mathcal{A} $ and a presheaf on $\mathcal{A}$,
	$X \in \mathcal{C}$

	\paragraph{Showing isomorphism between $[\mathcal{A}^{op},Set](H_A,X)$ and $X(A)$}\mbox{} \\

	Define $\string ^ :\mathcal{C}(H_A,X) \to X(A) $
	as the input's A-component, evaluated at the identity of A(in $\mathcal{A}$). i.e.\\
	\[  \text{ for natural transformation } \alpha:H_A \to X\text{ , define  }
	\hat{\alpha}:= \alpha_A(1_A) \text{, an element of X(A) } \]

	Define $\string ~ : X(A) \to [\mathcal{A}^{op}, Set](H_A,X)$ on element, $ x \in X(A), $
	by defining it's $K$-component for any $K \in \mathcal{A} $ as
	\[\tilde{x}_K : H_A(K) \to X(K)
		\text{ as, for each } p\in H_A(K)=\textit{Hom}_{\mathcal{A}^{op}}(A,K) \text{ , }
	\tilde{x}_K (p):=\Big(X(p)\Big)(x) \]

	Meaning that the $\tilde{x}_K$  maps any arrow $p:K \to A$
	to the image of x under the function $X(p):X(A)\to X(K)$.\\

	Now, to show that $\tilde{x}=(\tilde{x}_K)_{K\in \mathcal{A}} $ is a natural transformation,\\

	for any $q \in \mathcal{A}^{op}(K,L), \text{ the square }
	\begin{tikzcd}
		H_A(K) \arrow[swap]{d}{\tilde{x}_K} \arrow{r}{H_A(q)}
		& H_A(L) \arrow{d}{\tilde{x}_L}\\
		X(K) \arrow[swap]{r}{X(q)}
		& X(L)
	\end{tikzcd} \text{ i.e. }
	\begin{tikzcd}
		\mathcal{A}(K,A) \arrow[swap]{d}{\tilde{x}_K} \arrow{r}{ - \circ q}
		& \mathcal{A} (L,A) \arrow{d}{\tilde{x}_L}\\
		X(K) \arrow[swap]{r}{X(q)}
		& X(L)
	\end{tikzcd}\text{ must commute }$.\\

	So, for any $f:K\to A$, need that $\tilde{x}_L(f \circ q)= X(q) \circ \tilde{x}_K(f)$\\

	Now, LHS=$\tilde{x}_L(f \circ q ) =\Big( X(f \circ q)\Big)(x)$
	while RHS=$X(q) \circ \tilde{x}_K(f) =\Big(X(q)\Big) \big(X(f)(x)\big)
	=\Big(X(q) \circ X(f)\Big) (x) $

	And as $X$ is a contravariant functor, $X(f \circ q)= X(q) \circ X(f)$, giving that LHS=RHS.

	Finally, to show isomorphism, need to show that $\string ^ $ and $\string ~ $ are mutually inverse,

	\[ \text{ for any } x \in X(A),
	\hat{\tilde{x}}=\tilde{x}_A (1_A)=\Big(X(1_A) \Big) (x)=1_{X(A)}(x)=x \]

	And, for any $\alpha \in [\mathcal{A}^{op}, Set](H_A,X)$ , $\tilde{\hat{\alpha}}=\alpha$ i.e. that each of their
	components are equal. As both $\tilde{\hat{\alpha}}$ and $\alpha$ are natural transformations
	between functors that go to the category \textit{Set}, each of the components is a function.

	So, need to show that for any $f \in \mathcal{A} (K,A)=H_A(K)$,
	$\Big(\tilde{\hat{\alpha}}\Big)_K(f)$=$\alpha_K(f)$

	LHS=$\tilde{\hat{\alpha}}_B(f)=\Big(X(f)\Big)(\hat{\alpha})=\Big(X(f)\Big)(\alpha_A(1_A))$
	and RHS=$\alpha_K(f)= \alpha_K(1_A \circ f)$

	Now, as $\alpha$ is a natural transformation, the square
	$\begin{tikzcd}
		\mathcal{A} (A,A) \arrow[swap]{d}{\alpha_A} \arrow{r}{- \circ f}
	& \mathcal{A} (K,A) \arrow{d}{\alpha_K}\\
	X(A) \arrow[swap]{r}{X(f)}
	& X(K)
	\end{tikzcd}$
	commutes, for $1_A$, giving that\\

	$\alpha_K(1_A \circ f)=\Big(X(f) \Big) \big( \alpha_A (1_A) \big)$ thus, RHS=LHS,
	and the isomorphism is shown.

	\paragraph{Showing naturality of this isomorphism }\mbox{} \\



\end{proof}
%}}}

\section{Cayley's Theorem}%
\begin{definition}[Symmetric group on a set] %{{{ Symmetric group on a set	Definition

\end{definition}
\begin{proof}

\end{proof}


\begin{theorem}{\textbf{Cayley's Theorem}} %{{{ Cayley's Theorem		Theorem
	Every group, (G,.) is isomorphic to a subgroup of symmetric group on G.
\end{theorem}

\section{Embedding of a category in Presheaf category}%{{{ Embedding		Section

\begin{definition}[Embedding of a category] %{{{ Embedding of a category		Definition
	A category, $\mathcal{A}$  is said to be embedded in a category, $\mathcal{B}$ if there exists a
	functor $F: \mathcal{A} \to \mathcal{B} $ such that $F$ is full, faithful and injective (on objects).
\end{definition} %}}}





\end{document}
