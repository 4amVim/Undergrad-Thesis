\documentclass[18pt,a4paper]{extarticle}
\usepackage[margin=0.7in]{geometry}
\usepackage{amsmath,amsthm,amssymb,tikz-cd}
\usepackage{enumitem,amsfonts,extarrows,xcolor}

\theoremstyle{definition}
\newtheorem{theorem}{Theorem}[section]

\theoremstyle{definition}
\newtheorem{definition}{Definiton}[section]

\begin{document}
\section{}
\begin{theorem}{\textbf{Yoneda}} %{{{ Yoneda		Theorem
	If $\mathcal{A} $ be a locally small category, then, \\
	\[ [ \mathcal{A} ^{op},Set ](H_A,X) \cong X(A) \text{ naturally in } A \in \mathcal{A} \text{ and }
	X \in [ \mathcal{A}^{op},Set]\]
\end{theorem}
\textbf{Explaination:} \\
First of all, we fix any category, $\mathcal{A} $ . Now we choose two things (independent of each other):
\begin{enumerate}[label=\roman*]
	\item an object, $A$ from the category $ \mathcal{A}= \mathcal{A} ^{op} $
	\item an object, $X$ of the category $[ \mathcal{A}^{op},Set] $
		which is precisely a functor $X: A^{op} \rightarrow Set$
\end{enumerate}
Here, $[ \mathcal{A}^{op},Set](	H_A,X) $ denotes arrows $H_A \rightarrow X$ in $[ \mathcal{A}^{op},Set] $
i.e. natural transformations, $\alpha :$
\begin{tikzcd}[row sep=large, column sep=huge]
	A^{op} \arrow[r,bend left=50, "H_A"{name=U, below}]
\arrow[r, bend right = 50, "X"{name=D, above}]
& Set
\arrow[Rightarrow, from=U , to=D, "\alpha"]
	\end{tikzcd}
$X(A)$ is precisely a set. As $X(A)$ is the image of (our chosen object,) $A$, under (our chosen functor,) X. \\

So, the idea is that our choice of $A \text{ and } X$ completely determines all possible maps
(i.e. natural transformations) from $H_A \text{ to } X$. This answers out big question of "what are all the maps
$H_A  \rightarrow X$ " or "how does $H_A$ see other presheaves on $\mathcal{A} $ ".\\

\textit{More than the above must be true, as it's not just that the two are isomorphic, but they're \textbf{naturally} isomorphic.}

This more is supposed to be that both $[ \mathcal{A}^{op},Set] \text{ and } X(A)$ are
\textit{functorial} in \textit{both} $A \text{ and } X$






\end{document}

