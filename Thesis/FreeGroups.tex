\documentclass{article}
\usepackage[margin=0.7in]{geometry}
\usepackage{amsmath,amsthm,amssymb,xcolor}
\usepackage{enumitem, tikz-cd}

\theoremstyle{definition}
\newtheorem{example}{Example}[section]

\theoremstyle{definition}
\newtheorem{definition}{Definiton}[section]

\newtheorem{theorem}{Theorem}[section]
\newtheorem{corollary}{Corollary}[theorem]

\begin{document}
\paragraph{A note on Free groups(borrowed from Hungerford Ch1)}
\begin{definition}[Concrete category] %{{{ Concrete category		Definition
	A category $ \mathcal{A}  $ , along with a \textcolor{red}{ faithful functor to $ Set $ }
\end{definition}

\begin{definition}[Free object on a set] %{{{ Free object 		Definition
	Let $ A $ be an object of concrete category $ \mathcal{A}  $ ,$ \, X \neq \phi $ a set, and a map $ i:X \rightarrow A $. Then, $ A $ is said to be free on the set X iff
	\begin{equation*}
		B\in \mathcal{B} \text{ and }f:X\rightarrow B\implies
		\begin{tikzcd}
			X \arrow{r}{i} \arrow [swap]{dr}{f} & A \arrow{d}{g} \\ & B
		\end{tikzcd}
		\text{ commutes i.e. }\exists! g \in \mathcal{A}(A,B)\text{ such that } g \circ i = f
	\end{equation*}




\end{definition}

\begin{theorem}[Free objects depend only on the cardinality of the set they're free on] %{{{ Free objects depend only on the cardinality of the set they're free on		Definition
	If $ \text{ F and F' }$ are objects of a concrete category $ \mathcal{A} $ such that they're free on $ X \text{ and } X' $ respectively. Then
	\[ |X|=|X'| \implies F \cong F'\]
\end{theorem}
\begin{proof}
	Let $ A,B $ be free on $ X,Y ;\; i:X \rightarrow A \text{ and } j:Y \rightarrow B $. With $ |X|=|Y| $ , so, there's a bijection, $ p: X \leftrightarrow Y $
	\begin{equation}
		\text{As $ A$  is free on $ X $ ,}
		\begin{tikzcd}
			X \arrow{r}{i} \arrow[swap]{dr}{p \circ j} & A  \arrow{d}{f} \\ & B
		\end{tikzcd}
		\text{ i.e. }
		\begin{tikzcd}
			X \arrow{r}{i} \arrow{d}{p} & A  \arrow{d}{f} \\ Y \arrow{r}{j} & B
		\end{tikzcd}
		\text{ must commute for some unique f.}
	\end{equation}

	\begin{equation}
		\text{Similarly, as $ B$  is free on $ Y $ ,}
		\begin{tikzcd}
			Y \arrow{r}{j} \arrow[swap]{dr}{p^{-1} \circ i} & B  \arrow{d}{g} \\ & A
		\end{tikzcd}
		\text{ i.e. }
		\begin{tikzcd}
			Y \arrow{r}{j} \arrow{d}{p^{-1}} & B  \arrow{d}{g} \\ X \arrow{r}{i} & A
		\end{tikzcd}
		\text{ must commute for some unique g.}
	\end{equation}
	\begin{equation*}
		\text{ Combining (1):}
		\begin{tikzcd}
			A \arrow{r}{f} & B \\ X \arrow{u}{i} \arrow{r}{p} & Y\arrow{u}{j}
		\end{tikzcd}
		\text{ and (2):}
		\begin{tikzcd}
			B \arrow{r}{g} & A \\ Y \arrow{u}{j} \arrow{r}{p^{-1}} & X\arrow{u}{i}
		\end{tikzcd}
		\text{ gives }
		\begin{tikzcd}
			A \arrow{r}{f \circ g} &[4em] A \\ X \arrow{u}{i} \arrow{r}{p \circ p^{-1} = 1_X} & X\arrow{u}{i}
		\end{tikzcd}
		\text{ So that }
		\begin{tikzcd}
			X \arrow{r}{i} \arrow[swap]{dr}{i \circ 1_X = i} &A \arrow{d}{f \circ g} \\ &A
		\end{tikzcd}
	\end{equation*}
	But again, as $ A $ is free on $ X $ , there exists a unique $ \psi $ satisfying $ \psi \circ i = i $. Thus, $ \psi = f \circ g $ , and as $ 1_A \circ i = i $, uniqueness of $ \psi $ gives $ f \circ g = 1_A $. Now, via symmetry, $ g \circ f = 1_B $. Hence, $A \cong B$
\end{proof}
\begin{corollary}
	Two objects of a category are free on the same set only if they're isomorphic.
\end{corollary}

Given X, going to con



\end{document}

