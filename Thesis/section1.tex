\documentclass[18pt,a4paper]{article}
\usepackage[margin=0.7in]{geometry}
\usepackage{amsmath,amsthm,amssymb,tikz-cd}
\usepackage{enumitem,amsfonts,extarrows,xcolor}

\newtheorem{theorem}{Theorem}[section]

\theoremstyle{definition}
\newtheorem{definition}{Definiton}[section]
\newtheorem{lemma}{Lemma}[definition]
\newtheorem{prop}{Proposition}[definition]
\newtheorem{proop}{Proposition}[section]

\tikzset{commutative diagrams/.cd,mysymbol/.style={start anchor=center,end anchor=center,draw=none}}
\newcommand\cen[2][\leq]{\arrow[mysymbol]{#2}[description]{#1}}

\makeatletter
\newcommand{\carrow}{}% just in case
\DeclareRobustCommand{\carrow}{%
  \mathrel{\vphantom{\rightarrow}\mathpalette\circle@arrow\relax}%
}
\newcommand{\circle@arrow}[2]{%
  \m@th
  \ooalign{%
    \hidewidth$#1\circ\mkern1mu$\hidewidth\cr
    $#1\longrightarrow$\cr}%
}

\begin{document}
\section{Definitions}%
\label{sec:definitions}

\begin{definition}[Prorelation] %{{{ Prorelation		Definition
A pre-ordered set of relations $X \to Y$, which is down-directed and an upper set. i.e
A set, $P \subseteq \mathcal{P}(X \times Y)$ such that
	\begin{enumerate}[label=(\roman*)]
		\item A pre-order defined to be containment as relations, $r \subseteq s$
			only if $\forall (x,y) \in X \times Y$, $(x,y) \in r \implies (x,y)\in s$
		\item (Down-directed), $\forall r,s \in P, \exists t \in P $ such that
			$t\subseteq r \text{ and } t \subseteq s$
		\item (Up-set) for any relation $u:X\to Y$, if $\exists p \in P \text{ such that } p\leq u $
			then $u \in P$
\end{enumerate} \end{definition} %}}}
\begin{definition}[Composition of prorelations] %{{{ Composition of prorelations		Definition
	Prorelations can be composed by taking all compositions of their elements as relations:
	for prorelations $P:X\to Y \text{ and } Q:Y\to Z$,
	\[ Q.P:=\{q \circ p : p \in P \text{ and } q \in Q  \} \]
\end{definition}
\begin{definition}[Comparison of Prorelations] %{{{ Comparison of Prorelations		Definition
Two prorelations with same domain, co-domain are comparable as
\[ \text{ for }  P,Q:X \to Y \text{ , } P \leq Q \text{ if } \forall q \in Q, \exists p \in P
\text{ such that } p \subseteq q    \]
\end{definition}
\begin{definition}[Quasi-uniformity] %{{{ Quasi-uniformity		Definition
A prorelation on a set X, $P:X \to X$ is a quasi-uniformity if it follows :
\begin{enumerate}[label=\roman*]
	\item $\forall p \in P $ , for any $x \in X$, $(x,x) \in p$ i.e. $xpx$
	\item $\forall p \in P, \exists p' \in P \text{ such that } p' \circ p' \subseteq p$
\end{enumerate}
And in this case, $(X,A)$ is called a \textit{quasi-uniform space}.
\end{definition}
\begin{definition}[Uniformly Continuous function ] %{{{ Uniformly Continuous function 		Definition
	A function, $f:X \to Y$ is called a uniformly continuous function, $f:(X,A) \to (Y,B)$
	if, $\forall b \in B, \exists a \in A \text{ such that } f \circ a \subseteq b \circ f $.
	meaning that $f.A \leq B.f$ or
	\begin{tikzcd}
		X \cen{dr} \arrow[swap]{d}{A} \arrow{r}{f}
		& Y \arrow{d}{B}\\
	X \arrow[swap]{r}{f}
		& Y
	\end{tikzcd}.

\end{definition}
\begin{definition}[Promodule] %{{{ Promodule		Definition
	A prorelation, $\phi:X \carrow Y$ is called a promodule $\phi: (X,A) \carrow (Y,B)$  if it obeys:
	$ \phi.A \leq \phi \text{ and } B. \phi \leq \phi $ where . denotes composition as prorelations.
\end{definition}
\begin{definition}[Comparison of Promodules] %{{{ Comparison of Promodules		Definition
	Promodules with same domain and co-domain are compared as prorelations, for
	$\phi,\psi:(X,A)\carrow (Y,B)$, $\phi \sqsubseteq \psi$ , only if $\phi \leq \psi$.
\end{definition}
\begin{definition}[Composition of Promodules] %{{{ Composition of Promodules		Definition
	Promodules are composed as prorelations.\\
	For promodules $\phi: (X,A) \carrow (Y,B)$ and $\psi:(Y,B)\carrow(Z,C)$,
	$\psi \phi := \psi.\phi = \{q \circ p :p\in \phi \text{ and } q \in \psi \}$
\end{definition}
\begin{definition}[Opposite relation] %{{{ Opposite relation		Definition
For relation $r:X \to Y$, $r^o$ is defined to be a relation $r^o:Y \to X$ as
\[ \forall (x,y)\in X \times Y, (x,y)\in r \iff (y,x) \in r^o \]
\end{definition}
\begin{definition}[$(-)_*$] %{{{ (-)_*		Definition

\end{definition}
\begin{definition}[$(-)^*$] %{{{ (-)^*		Definition

\end{definition}
\begin{definition}[Fully Faithful] %{{{ Fully Faithful		Definition

\end{definition}

\begin{definition}[Fully Dense] %{{{ Fully Faithful		Definition

\end{definition}
\begin{definition}[Topologically Dense] %{{{ Topologically Dense		Definition

\end{definition}
\section{Propositions}
\begin{definition}[QUnif] %{{{ QUnif		Definition
QUnif is defined to be the category having quasi-uniform spaces as objects, and uniformly continous
maps between them as morphisms.
\end{definition}
\begin{lemma} QUnif does define a category, as
	\begin{enumerate}[label=\roman*]
		\item Composition
		\item Identity
	\end{enumerate}
\end{lemma}

\begin{definition}[ProMod] %{{{ ProMod		Definition

\end{definition}

\begin{lemma} QUnif does define a category, as
	\begin{enumerate}[label=\roman*]
		\item Composition
		\item Identity
	\end{enumerate}
\end{lemma}

\begin{proop}[Promodule Category]

\end{proop}


%Mention about topology so that below definition makes more sense ??%


\begin{proop}[$(-)_*$ is a Functor ]

\end{proop}

\begin{proop}[$(-)^*$ is a Functor ]

\end{proop}

%Talk about adjointness ??

\begin{proop}[Proposition 1]
	Fix a uniformly continuous map, $f:(X,A) \rightarrow (Y,B)$
	\begin{enumerate}[label=(\alph*)]
		\item f is fully faithful $\iff A=f^o.B.f$
		\item f is fully dense $\iff \forall b\in B, \exists b' \in B \text{ such that }  $
	\end{enumerate}
\end{proop}






\newpage
asd
\end{document}
