\documentclass[a4paper]{article}
\usepackage[margin=0.7in]{geometry}
\usepackage{amsmath,amsthm,amssymb,tikz-cd}
\usepackage{enumitem,amsfonts,extarrows,mathtools}
\usepackage{biblatex}
\addbibresource{references.bib}

\newtheorem{theorem}{Theorem}[section]
\newtheorem{coro}[theorem]{Corollary}
\theoremstyle{definition}
\newtheorem{definition}[theorem]{Definition}
\newtheorem{lemma}[theorem]{Lemma}
\newtheorem{prop}[theorem]{Proposition}
\newtheorem{example}[theorem]{Example}

\tikzset{commutative diagrams/.cd,mysymbol/.style={start anchor=center,end anchor=center,draw=none}}
\newcommand\cen[2][\leq]{\arrow[mysymbol]{#2}[description]{#1}}

\makeatletter
\newcommand{\carrow}{}% just in case
\DeclareRobustCommand{\carrow}{%
	\mathrel{\vphantom{\rightarrow}\mathpalette\circle@arrow\relax}%
}

\newcommand{\circle@arrow}[2]{%
	\m@th
	\ooalign{%
		\hidewidth$#1\circ\mkern1mu$\hidewidth\cr
	$#1\longrightarrow$\cr}%
}
\begin{document}
\begin{titlepage}
    \begin{center}
        \vspace*{1cm}

        \Huge
        \textbf{Yoneda Lemma and Quasi-Uniform Spaces}

        \vspace{0.5cm}
        \LARGE
	by
        \vspace{1.5cm}

        \textbf{Ayush Rawat}

        \vfill

	Thesis submitted to Shiv Nadar University in partial fulfillment of the requirement
	for the award of Degree of Bachelor of Science (Research)
        \vspace{0.8cm}


        \Large
	June 2020

    \end{center}
\end{titlepage}
\begin{titlepage}
	\begin{center}
		\vspace*{1cm}

		\Huge

		\textbf{Acknowledgements}

	\end{center}
		\vspace{2cm}
		\LARGE
		I would like to acknowledge my immense thankfulness to my supervisor,
		Dr. Neha Gupta, for making this work possible.
		Her patience, guidance and motivation were invaluable throughout all stages of the work.
		\vfill
\end{titlepage}
\thispagestyle{empty}
\newpage
\begin{abstract}
	We detail a proof of Yoneda Lemma from \cite{Tom} and show Cayley's theorem and
	Yoneda Embedding as its consequences. Following \cite{Clementino_Hofmann_2011}, we define a
	categorical structure on quasi-uniform spaces and promodules. Then we establish Yoneda
	Embedding for quasi-uniform spaces. And prove a weak version of Yoneda Lemma for them.



	This undergraduate thesis \cite{Tom} aims to build up to the interpretation of Yoneda Lemma and Yoneda Embedding
as described in \cite{Clementino_Hofmann_2011}.
We first prove Yoneda Lemma and discuss some of its consequences. Then we define
Then we define the categories QUnif and ProMod, and show functors between them.
Finally, we show Yoneda Embedding and Yoneda Lemma for ProMod.

\end{abstract}
%\addcontentsline{toc}{section}{Acknowledgement}
\tableofcontents
\section{Introduction}

In this section, we set forth some basic definitions, and prove some lemmas that will be required in
order to prove Yoneda lemma.

For any category $\mathcal{A}$, and it's objects $X,Y\in \mathcal{A} $,
we denote $Hom_{\mathcal{A}}^{\mbox{}}(X,Y)$ with $\mathcal{A} (X,Y)$.

\begin{definition} %{{{Category A op		Definition
	For any category $\mathcal{A} $, it's opposite category, $\mathcal{A}^{op}$
	is the category having the objects of $\mathcal{A}$.
	And for objects $A,B \in \mathcal{A} $, a morphism $f \in \mathcal{A}^{op} (A,B)$
	if and only if there is a morphism $g \in \mathcal{A}(B,A)$.
\end{definition}
\begin{prop} %{{{ H_A	Definition
	For a locally small category $\mathcal{A}$, fixing an object $A \in \mathcal{A} $ gives
	a functor, $H_A: \mathcal{A} ^{op} \rightarrow Set$ defined as:
	\begin{enumerate}[label=(\roman*)]
		\item For any object $B \in \mathcal{A} $ , $H_A(B):=\mathcal{A} (B,A)$.
		\item For any morphism, $g : X \rightarrow Y $ in $\mathcal{A}$,
			\[H_A(g): \mathcal{A} (Y,A) \rightarrow \mathcal{A}(X,A)
			\text{ is given by } p \mapsto p \circ g.\]
	\end{enumerate}
\end{prop}
\begin{proof}\setcounter{equation}{0}
	Fix any objects $K,L,M \in \mathcal{A} $.
	\begin{enumerate}[label=\Roman*]
		\item \textbf{ (Composition) } As $H_A$ is a contravariant functor, for
			any morphisms $f\in \mathcal{A} (K,L)$ and $g \in
			\mathcal{A}(L,M)$, we need to show that
			$H_A(g \circ f)=H_A(f) \circ H_A(g)$. Note,
			the composition $g\circ f$ on the left hand side is in $\mathcal{A} ^{op}$.
			Hence, using the definition of $H_A$ gives us that for any $k \in H_A(M) $, we must have
			\begin{align*}
				LHS&=\Big( H_A(g \circ f) \Big) (k)=k\circ g\circ f \\
				\text{ and } RHS&=\Big( H_A(f)\circ H_A(g) \Big) (k)=\Big( H_A(f) \Big) (k \circ g)
			=(k\circ g)\circ f. \end{align*}
		\item \textbf{ (Identity) } We will show that for any $k \in \mathcal{A} (K,L)$, $H_A$ respects the identities
			of $K$ and $L$ in $\mathcal{A}$ (as they're equal to the identities of $K$ and
			$L$ in $\mathcal{A} ^{op}$. Using the definition of $H_A$,
			for any object $L\in \mathcal{A} $ and
			morphism $p \in H_A(L)$, we get the following equations.
			\begin{align*}
				\text{ Right Identity: }& \Big( \big( H_A(1_K) \big) \circ \big( H_A(k) \big) \Big)(p)
				=\big( H_A(1_K) \big) (p \circ k) = p \circ k \circ 1_K=p \circ k = \big( H_A(k) \big)(p)\\
				\text{ Left Identity: }& \Big( \big( H_A(k) \big) \circ \big( H_A(1_L) \big) \Big)(p)
				=\big( H_A(k) \big) (p \circ 1_L) = \big( H_A(k) \big) (p)
			\end{align*}

		\end{enumerate}
		Hence, $H_A$ is indeed a functor.
	\end{proof}
	\begin{definition} %{{{Presheaf Category 	Definition
		For a locally small category $\mathcal{A} $, the category of presheaves on $\mathcal{A} $, denoted by $[\mathcal{A} ^{op},Set]$
		is defined to have functors from $\mathcal{A} ^{op}$ to $Set$ as objects,
		and the natural transformations between them as morphisms.
	\end{definition}
	%{{{ 2 Lemmas
	\begin{lemma}
		Let $ 	\begin{tikzcd}[row sep=small, column sep=large]
			\mathcal{A}  \arrow[r,bend left=50, "F"{name=U, below}]
			\arrow[r, bend right = 50, "G"{name=D, above}]
		& \mathcal{B}
		\arrow[Rightarrow, from=U , to=D, "\alpha"]
		\end{tikzcd}$
		be a natural transformation. If for every $A \in \mathcal{A} $,
		$\alpha_A:F(A) \to G(A)$ is an isomorphism
		then $\alpha$ is a natural isomorphism.
	\end{lemma}
	\begin{proof}\setcounter{equation}{0}
		We are going to show that there exists a natural transformation $\beta:G \to F$.
		Fix any objects $A,B\in \mathcal{A} $ and morphism $k \in \mathcal{A}(A,B)$. As
		$\alpha$ is a natural transformation,
		\begin{equation}
			\alpha_B \circ F(k) = G(k) \circ \alpha_A.
		\end{equation}
		Because $\alpha_A$ is an isomorphism, we get that
		there exists
		$\beta_A:G(A)\to F(A)$ such that
		\begin{align}
			\alpha_A \circ \beta_A = 1_{G(A)} \text{ and }
			\beta_A \circ \alpha_A = 1_{F(A)}.
		\end{align}
		Similarly, $\alpha_B$ gives us the existence of $\beta_B:G(B)\to F(B)$
		such that $\beta_B \circ \alpha_B = 1_{F(B)}$. Multiplying (1) with $\beta_B$ and
		$\beta_A$,
		\begin{equation}
			\beta_B^{} \circ \alpha_B^{} \circ F(k) \circ \beta_A^{} =
			\beta_B^{} \circ G(k) \circ \alpha_A^{} \circ \beta_A^{}
			\implies  F(k) \circ \beta_A^{} =\beta_B^{} \circ G(k).
		\end{equation}
		Thus, $\beta$ is a natural transformation from $F$ to $G$. Using (2) gives us
		that $\big( \alpha \circ \beta \big)_A = 1_{G(A)}$ and
		$\big( \beta \circ \alpha \big)_A = 1_{F(A)}$ for any object $A\in \mathcal{A} $.
		Therefore,$\alpha \circ \beta = 1_G$ and $\beta \circ \alpha =1_F$.
		Hence, $\alpha$ and $\beta$ together give an isomorphism between $F$ and $G$
		in the functor category
		$[\mathcal{A},\mathcal{B}]$.
	\end{proof}
	\begin{lemma}
		Let $\mathcal{A},\mathcal{B}$ and $\mathcal{C}$ be categories. Suppose there
		are functors $F,G:\mathcal{A} \times \mathcal{B} \to \mathcal{C}$.

		For every $A\in \mathcal{A}$, there are functors, $F^A,G^A:\mathcal{B} \to \mathcal{C}$
		defined as taking $B \in \mathcal{B}$ to $F(A,B),\, G(A,B)$ and
		morphism $f$ to $F((1_A,f)),\, G((1_A,f)) $. And, for every $B\in \mathcal{B}$,
		there are functors $F_B,G_B:\mathcal{A} \to \mathcal{C}$ defined as
		taking $A \in \mathcal{A} $ to $F(A,B),\, G(A,B)$ and morphism $g$ to $F((g,1_B)),\, F((g,1_B))$.


		A family of maps, $\big( \alpha_{A,B}: F(A,B) \to G(A,B)
		\big)_{A \in \mathcal{A} , B \in \mathcal{B} } $ is a natural transformation
		$F \to G$ if the following conditions are satisfied:
		\begin{enumerate}[label=(\roman*)]
			\item For each $A \in \mathcal{A}$, the family
				$\big( \alpha_{A,B}: F^A(B) \to G^A(B) \big)_{ B \in \mathcal{B} } $ is
				a natural transformation $F^A \to G^A$;
			\item For each $B \in \mathcal{B}$, the family
				$\big( \alpha_{A,B}: F_B(A) \to G_B(A) \big)_{ A \in \mathcal{A} } $ is
				a natural transformation $F_B \to G_B$.
		\end{enumerate}
	\end{lemma}
	\begin{proof} \setcounter{equation}{0}
		In order to show that $\alpha_{(A,B)}$ is natural in $(A,B)$,
		we need to show that for any $A,A' \in \mathcal{A}$, $\; B,B' \in \mathcal{B}$
		and $(f,g) \in \mathcal{A} \times \mathcal{B} \; \Big((A,B),(A',B')\Big)$,
		the square
		$\begin{tikzcd}
			F\big((A,B)\big) \arrow[swap]{d}{\alpha_{(A,B)}} \arrow{r}{F\big((f,g)\big)}
		& F\big((A',B')\big) \arrow{d}{\alpha_Y}\\
		G(\big((A,B)\big)) \arrow[swap]{r}{G((f,g))}
		& G\big((A',B')\big)
		\end{tikzcd} \text{ commutes.}$
		Fix any objects $A,A' \in \mathcal{A} $ and $B,B' \in \mathcal{B} $.
		Fix any morphism $(g,f) \in \mathcal{A} \times \mathcal{B} \; ((A,B),(A',B'))$,
		where $g \in \mathcal{A} (A,A')$ and $f \in \mathcal{B} (B,B')$.
		\begin{enumerate}[label=(\Roman*)]
			\item We will show that $F^A$ is a functor from $\mathcal{B} $ to $\mathcal{C}$.
				It respects composition:
				\[F^A(f)\circ F^A(g)=(1_A,f)\circ(1_A,g)=(1_A,f\circ g)=F^A(f\circ g).\]
				And respects identity:
				\[F^A(1_B)=F(1_A,1_B)=1_{F_{(A,B)}}=1_{F^A(B)}.\]
				$G^A$ is shown a functor in the same manner.

			\item Condition (ii) gives us that
				$ \text { for any } g \in \mathcal{A}(A,A'), \text{ the square }
				\begin{tikzcd}
					F_A(B) \arrow[swap]{d}{\alpha_{A,B}} \arrow{r}{F_A(g)}
					& F_A(B') \arrow{d}{\alpha_{A,B'}}\\
					G_A(B) \arrow[swap]{r}{G_A(g)}
					& G_A(B')
				\end{tikzcd}
				\text{ commutes}$. By the definition of $F_A$ and $G_A$,
				this square can be written as
				$
				\begin{tikzcd}
					F(A,B) \arrow[swap]{d}{\alpha_{A,B}} \arrow{r}{F((g,1_B))}
					& F(A,B') \arrow{d}{\alpha_{A,B'}}\\
					G(A,B) \arrow[swap]{r}{G((g,1_B))}
					& G(A,B')
				\end{tikzcd}
				$.
			\item Condition (i) gives us that
				$ \text { for any } f \in \mathcal{B}(B,B'), \text{ the square }
				\begin{tikzcd}
					F^{A'}(B) \arrow[swap]{d}{\alpha_{A',B}} \arrow{r}{F^{A'}(f)}
					& F^{A'}(B') \arrow{d}{\alpha_{A',B'}}\\
					G^{A'}(B) \arrow[swap]{r}{G_{A'}(f)}
					& G^{A'}(B')
				\end{tikzcd}
				\text{ commutes}$. By the definition of $F^A$ and $G^A$,
				this square can be written as
				$
				\begin{tikzcd}
					F(A',B) \arrow[swap]{d}{\alpha_{A',B}} \arrow{r}{F((1_{A'},f))}
					& F(A',B') \arrow{d}{\alpha_{A',B'}}\\
					G(A',B) \arrow[swap]{r}{G((1_{A'},f))}
					& G(A',B')
				\end{tikzcd}
				$.
			\item Composing the squares from (II) and (III), we get that the following
				rectangle commutes:
				\begin{equation}	\begin{tikzcd}[row sep=large, column sep=huge]
					F(A,B) \arrow[swap]{d}{\alpha_{A,B}} \arrow{r}{F((g,1_B))}
					& F(A,B') \arrow[swap]{d}{\alpha_{A',B}} \arrow{r}{F((1_{A'},f))}
					& F(A',B') \arrow{d}{\alpha_{A',B'}}\\
					G(A,B) \arrow[swap]{r}{G((g,1_B))}
					& G(A,B')\arrow[swap]{r}{G((1_{A'},f))}	& G(A',B')
				\end{tikzcd}.\end{equation}
				Using (1), we get that:
				\begin{equation} \alpha_{A',B'} \circ \big( F((1_{A'},f))\circ F((g,1_B))\big)
				=\big( G((1_{A'},f))\circ G((g,1_B)) \circ \alpha_{A',B'} .\end{equation}
					As $F$ and $G$ are functors, from (2), we have that
					\[ \alpha_{A',B'} \circ F\big((g,f)\big)
					=G\big((g,f)\big) \circ \alpha_{A',B'}. \qedhere \]
			\end{enumerate}
			\mbox{}
		\end{proof}
		%}}}
\section{Yoneda Lemma} %{{{ Yoneda Lemma		Section
This section starts with a proof of Yoneda Lemma, and then goes on to use it to derive Cayley's Theorem.
After that, we mention Yoneda Embedding as it's consequences.
		\begin{theorem}{\textbf{Yoneda Lemma}} %{{{ Yoneda		Theorem
			\setcounter{equation}{0}
			If $\mathcal{A} $ is a locally small category then, for any object $A \in \mathcal{A} $
			and $X \in [ \mathcal{A}^{op},Set]$,\\ there exists an isomorphism,
			\begin{equation} [ \mathcal{A} ^{op},Set ](H_A,X)\cong X(A) \text{ which is natural in } A \text{ and } X.\end{equation}
		\end{theorem}
		\paragraph{Notation:} \begin{itemize}
			\item We denote the category of presheaves on $\mathcal{A}$ by $\mathcal{C}$.
			\item For the map $\string ^$, instead of writing $\string ^ (a)=b$,
				we use $\hat{a} = b$ to denote $a \mapsto b$.

			\item For the map $\string ~$, instead of writing $\string ~ (a)=b$,
				we use $\tilde{a} = b$ to denote $a \mapsto b$.
			\item $[ \mathcal{A}^{op},Set](	H_A,X) $ denotes the collection of morphisms $\alpha: \begin{tikzcd}[row sep=large, column sep=huge]
					A^{op} \arrow[r,bend left=50, "H_A"{name=U, below}]
					\arrow[r, bend right = 50, "X"{name=D, above}]
					& Set
					\arrow[Rightarrow, from=U , to=D, "\alpha"]
				\end{tikzcd}$.
		\end{itemize}
		To prove the theorem,
		first, we show that $[ \mathcal{A} ^{op}, Set](H_A,X)$ is isomorphic to $X(A) $ as set,
		and then that this isomorphism is natural in $X$ and $A$.
		\begin{proof}\setcounter{equation}{0}
			Let $\mathcal{A} $ be a locally small category.
			Fix an object $A \in \mathcal{A} $ and a presheaf $X$ on $\mathcal{A}$.

			\begin{enumerate}[label=\Roman*]
				\item \textbf{Showing isomorphism between $[\mathcal{A}^{op},Set](H_A,X)$
					and $X(A)$}\\
					Define $\string ^ :\mathcal{C}(H_A,X) \to X(A) $
					for any $\alpha:H_A \to X,$ as  $\hat{\alpha}:= \alpha_A(1_A)$.
					As $1_A \in Set(A,A)=H_A(A)$,
					definition of $\alpha_A$ gives that $\alpha_A(1_A)\in X(A)$.
					%{{{Define ~

					Define $\string ~ : X(A) \to [\mathcal{A}^{op}, Set](H_A,X)$
					for any $ x \in X(A)$ as the natural transformation $\tilde{x} : H_A \to X$ whose
					K-component is the function mapping each morphism $p \in \mathcal{A}(K,A)$
					to $\Big(X(p)\Big)(x)$. That is, $\tilde{x}_K (p):=\Big(X(p)\Big)(x)$.\\

					We are going to show that $\tilde{x}$ is a natural transformation.
					Fix objects $K,L \in \mathcal{A} $ and morphism $q \in \mathcal{A}^{op}(K,L)$.
					$\text{Need to show that the square }
					\begin{tikzcd}
						H_A(K) \arrow[swap]{d}{\tilde{x}_K} \arrow{r}{H_A(q)}
		& H_A(L) \arrow{d}{\tilde{x}_L}\\
		X(K) \arrow[swap]{r}{X(q)}
		& X(L)
					\end{tikzcd} \text{, that is }
					\begin{tikzcd}
						\mathcal{A}(K,A) \arrow[swap]{d}{\tilde{x}_K} \arrow{r}{ - \circ q}
		& \mathcal{A} (L,A) \arrow{d}{\tilde{x}_L}\\
		X(K) \arrow[swap]{r}{X(q)}
		& X(L)
					\end{tikzcd}\text{ commutes }$.\\

					So, for any $f:K\to A$, need that $\tilde{x}_L(f \circ q)= X(q) \circ \tilde{x}_K(f)$. Using
					the definition of $\tilde{x} $ gives the following.
					\begin{align*}
						LHS &=\tilde{x}_L(f \circ q ) =\Big( X(f \circ q)\Big)(x) \\
						RHS &=X(q) \circ \tilde{x}_K(f) =\Big(X(q)\Big) \big(X(f)(x)\big)=\Big(X(q) \circ X(f)\Big) (x)
					\end{align*}
					And as $X$ is a contravariant functor, $X(f \circ q)= X(q) \circ X(f)$,
					giving that LHS=RHS.
					%}}}
					%{{{Isomorphism
					Now going to show that $\string ^$ and $\string ~$ define an isomorphism.
					Need to show that $\string ^ $ and $\string ~ $ are mutually inverse.
					\begin{enumerate}[label=(\roman*)]
						\item  For any $x \in X(A)$,
							$\hat{\tilde{x}}=\tilde{x}_A (1_A)=\Big(X(1_A) \Big) (x)=1_{X(A)}(x)=x$.
						\item For any $\alpha \in \mathcal{C} (H_A,X)$ , need to show that $\tilde{\hat{\alpha}}=\alpha$.
							So, it's required that each of their component are equal.
							As both $\tilde{\hat{\alpha}}$ and $\alpha$ are natural transformations
							between functors that go to the category \textit{Set}, each of the components is a function.
							So, need to show that for any $f \in \mathcal{A} (K,A)=H_A(K)$,
							$\Big(\tilde{\hat{\alpha}}\Big)_K(f)$=$\alpha_K(f)$.
							Using first the definition of $ \string ~$ and then that of $\hat{\alpha}$ gives:
							\begin{align}
								LHS=\tilde{\hat{\alpha}}_B(f)=\Big(X(f)\Big)(\hat{\alpha})=\Big(X(f)\Big)(\alpha_A(1_A))
							\end{align}
							And as $f\in \mathcal{A} (K,A)$, we also have the following.
							\begin{equation}
								RHS=\alpha_K(f)= \alpha_K(1_A \circ f)
							\end{equation}
							Because $\alpha$ is a natural transformation, the square following square commutes for $1_A$.
							\[\begin{tikzcd}
								\mathcal{A} (A,A) \arrow[swap]{d}{\alpha_A} \arrow{r}{- \circ f}
	& \mathcal{A} (K,A) \arrow{d}{\alpha_K}\\
	X(A) \arrow[swap]{r}{X(f)}
	& X(K)
							\end{tikzcd}\]
							This gives that $\alpha_K(1_A \circ f)=\Big(X(f) \Big) \big( \alpha_A (1_A) \big)$.
							Hence, we have from (2) and (3), we get that $RHS=LHS$.
					\end{enumerate}
					%}}}
					%{{{Naturality
				\item \textbf{Showing naturality of this isomorphism }\\
					By Using Lemma 1.4 and 1.5, it's enough to show that $\string ^$
					is natural in $X$ and natural in $A$.

					\begin{enumerate}[label=(\roman*)]
						\item We are going to show the above isomorphism to be natural in X.
							Fix any $A \in \mathcal{A}$. Need that for
							presheaves $X,Y \in \mathcal{C}$ and
							natural transformation $\beta \in \mathcal{C} (X,Y)$, the following
							square commutes.
							\begin{equation*}
								\begin{tikzcd}
									\mathcal{C}(H_A,X) \arrow[swap]{d}{ \string ^}
									\arrow{r}{\beta \circ -}
			& \mathcal{C}(H_A,Y) \arrow{d}{\string ^}\\
			X(A) \arrow[swap]{r}{\beta_A}
			& Y(A)
								\end{tikzcd}
							\end{equation*}
							So, for any $\alpha:H_A \to X$, we need that
							$\big( \string ^ \circ H_\beta \big)(\alpha) = \big( \beta_A \circ \string ^
							\big)(\alpha)$. Using definition of $H_\beta$ and $\string ^$ gives:
							\begin{align}
								LHS & = \big( \string ^ \circ H_\beta \big)(\alpha) =
								\widehat{\big(  H_\beta(\alpha)\big)}=
								\widehat{\big(  \beta \circ \alpha\big)}
								=\big(  \beta \circ \alpha\big)_A(1_A) \\
								RHS & = \big( \beta_A \circ \string ^ \big)(\alpha)
								= \beta_A (\widehat{\alpha}) =\big( \beta_A \circ \alpha_A\big)(1_A)
							\end{align}
							As $\alpha \in \mathcal{C} (H_A,X)$ and $\beta \in \mathcal{C}(X,Y)$
							are morphisms in $\mathcal{C}$, composition in $\mathcal{C}$ gives
							$(\beta \circ \alpha)_A = \beta_A \circ \alpha_A$. From (4) and (5), we
							directly get that $RHS=LHS$.

						\item We are going to show natural in $A$. Fix any $X \in \mathcal{C} $ Need that
							for objects $A,B \in \mathcal{A} $ and morphism $f\in \mathcal{A} ^{op}(A,B)$,
							the following square commutes.
							\begin{equation*}
								\begin{tikzcd}
									\mathcal{C}(H_A,X) \arrow[swap]{d}{ \string ^}
									\arrow{r}{- \circ H_f}
			& \mathcal{C}(H_B,Y) \arrow{d}{\string ^}\\
			X(A) \arrow[swap]{r}{X(f)}
			& X(B)
								\end{tikzcd}
							\end{equation*}
							So, for any $\alpha:H_A \to X$, we need that
							$\big( \string ^ \circ H_f \big)(\alpha)=\Big((X(f)) \circ \string ^ \Big)(\alpha) $.
							Using definition of $H_f$ and $\string ^$, we get:
							\begin{align}
								LHS&=\big( \string ^ \circ H_f \big)(\alpha)= \widehat{\alpha \circ H_f}
								=(\alpha \circ H_f)_B (1_B) = \alpha_B (f \circ 1_B) = \alpha_B(1_A \circ f) \\
								RHS&=\Big((X(f)) \circ \string ^ \Big)(\alpha)
								= (X(f))(\hat{\alpha})= \Big(X(f) \Big)\big(\alpha_A(1_A)\big)
							\end{align}
							The last equality in (6) is justified as $f$ goes from $B$ to
							$A$ in $\mathcal{A} $. By using equality of (2) and (3) from I(i), for
							$f \in \mathcal{A}(B,A)$, we get that
							$\Big(X(f) \Big)\big(\alpha_A(1_A)\big)=\alpha_B( 1_A \circ f) $.
							Hence, $RHS=LHS$. \qedhere
					\end{enumerate}
					%}}}
			\end{enumerate}
		\end{proof}
		\subsection{Cayley's Theorem}%
		%{{{Cayley Write-up
		Informally, given a locally small category$\mathcal{A} $, we can fix a presheaf $X$ on $\mathcal{A}$,
		and for any object $A \in \mathcal{A} $, study the set $X(A)$ and gain information
		about all possible natural transformations between $H_A$ and $X$. Moreover, by part I(ii) of the
		proof of Yoneda Lemma, each of the natural transformations is determined by its action
		on the identity morphisms in $\mathcal{A}$. Thus, no matter how complicated $\mathcal{A}$ is, if we choose $X$ carefully, we can hope to understand the structure of $\mathcal{A} $
		by looking at how $X(A)$ changes as we vary the chosen
		presheaf and object.

		In group theory, Cayley's theorem says every group $G$ is isomorphic to a subgroup of the symmetric
		group on G. Thus, instead of having to study a complicated group directly, we can study a subgroup of
		the symmetric group on it.

		Cayley's theorem and Yoneda Lemma are similar in the sense that both allow us to change the environment
		that we study in by putting few restrictions on what we are allowed to study. Cayley allows us to
		change setting for groups, and Yoneda does that for locally small categories.

		Also, as groups themselves can be considered as small categories, we can apply Yoneda Lemma to any
		group. In fact, we can get Cayley's theorem as a consequence of Yoneda Lemma by a suitable
		choice of $X$ and $A$.
		%}}}
		\begin{definition} %{{{ Symmetric group on a set	Definition
			Symmetric group on a set $X$ is the set of all bijections on $X$, with the binary operation
			defined as composition of bijections.
		\end{definition}
		We will now use Theorem 2.6 and parts of its proof to prove Cayley's theorem.
		The notation $(\string _)$ occurs as a placeholder for the element a map is applied to.
		That is, $(\string _ \circ f)(k)$ is defined to be $k \circ f$.
		And we use the notation $g.f$ to mean the composition of $g$ and $f$ in the group.
		\begin{theorem}{\textbf{Cayley's Theorem}} %{{{ Cayley's Theorem		Theorem
			Every group, $G$ is isomorphic to a subgroup of symmetric group on $G$.
		\end{theorem}
		\begin{proof} \setcounter{equation}{0} Let group $G$ be a group.
			Define category $\mathcal{A}$ with a single object $\star$. And precisely one morphism
			in $\mathcal{A} $ for each element of $G$, with the composition of said morphisms
			being as that of elements of $G$.
			That is, for morphisms $f$ and $g$ in $\mathcal{A}$, $f \circ g$ is defined to
			be the morphism $f.g$. Then, $G$ and $\mathcal{A}(\star, \star)$ have the
			same elements
			and rule of composition, so there exists a group isomorphism $\psi:A(\star,\star) \to G$.

			\begin{enumerate}[label=\Roman*]
				\item \textbf{Natural transformations from $H_\star$
					to $H_\star$ are bijections on G.}

					As $\mathcal{A} ^{op}$ is a category with a single object,
					each natural transformation $\alpha:
					\begin{tikzcd}[row sep=small, column sep=large]
						\mathcal{A} ^{op} \arrow[r,bend left=50, "H_\star"{name=U, below}]
						\arrow[r, bend right = 50, "H_\star"{name=D, above}]
						& Set
						\arrow[Rightarrow, from=U , to=D, "\alpha"]
					\end{tikzcd}$
					has only one component, that is $\alpha_\star$. Therefore, we can
					identify $\alpha$ with $\alpha_\star$.
					Using naturality of $\alpha$, we get that
					\begin{equation}
						\text{ the square }
						\begin{tikzcd}
							H_\star(\star) \arrow[swap]{d}{\alpha_\star} \arrow{r}{H_\star(f)}
			& H_\star(\star) \arrow{d}{\alpha_\star}\\
			H_\star(\star) \arrow[swap]{r}{H_\star(f)}
			& H_\star(\star)
						\end{tikzcd}
						, \text{ that is }
						\begin{tikzcd}
							\mathcal{A} (\star,\star) \arrow[swap]{d}{\alpha_\star} \arrow{r}{\string _ \circ f}
			& \mathcal{A} (\star,\star) \arrow{d}{\alpha_\star}\\
			\mathcal{A} (\star,\star) \arrow[swap]{r}{ \string _ \circ f}
			& \mathcal{A} (\star,\star)
						\end{tikzcd}
						\text{ commutes for any } f \in \mathcal{A}(\star,\star).
					\end{equation}
					Applying the identity of $\star$ in $\mathcal{A}$ in (1)
					gives us the following equation:
					\begin{equation}
						\big( (\string _ \circ f) \circ \alpha_\star \big)(1_\star) =
						\big(  \alpha_\star \circ (\string _ \circ f)\big)(1_\star)
						\implies \alpha_\star(f)=\alpha_\star(1_\star) \circ f
						\implies \alpha_\star(f)=\alpha_\star(1_\star).f \; ,
					\end{equation}
					where the $RHS$ of last implication is given by the definition
					of composition in $\mathcal{A}$.
					Thus, every natural transformation $\alpha$ is defined in
					terms of its value at $1_\star$.
					This can be considered as left multiplication by
					$\alpha_\star(1_\star)$ in $G$, which
					we know is an automorphism of $G$. Thus, $\alpha_\star$, and hence $\alpha$
					can be thought of as a bijection on $G$.

					So far we have shown that
					the collection $[\mathcal{A} ^{op},Set](H_\star,H_\star)$ of
					all $\alpha:H_\star \to H_\star$ is a
					collection of bijections on $G$.
				\item\textbf{The collection $[\mathcal{A}^{op},Set](H_\star,H_\star)$ is a group.}

					We will show that the collection
					$[\mathcal{A} ^{op},Set](H_\star, H_\star)$
					is a group with respect to composition in the category
					$[\mathcal{A} ^{op},Set]$.
					As $[\mathcal{A} ^{op},Set]$ is a category, we have that
					the composition is associative. Also, because this collection contains
					morphisms with the same source and destination,
					it is closed under composition. Identity of
					$[ \mathcal{A} ^{op},Set ] (H_\star,H_\star)$ will act as
					the identity for its group structure.

					We will now show closure under inverses.
					Fix any $\gamma:H_\star \to H_\star$.
					Since $\gamma_\star(1_\star)$ belongs to
					$ \mathcal{A} (\star,\star)$,
					let us call $\psi(\gamma_\star(1_\star))=h \in G$. Thus,
					there exists $h^{-1}\in G$. As $\psi$ is onto,
					there exists $a\in \mathcal{A} (\star,\star)$ such that
					$\psi(a)=h^{-1}$.
					From (2), we know that any natural transformation $\alpha$
					is defined in terms of $\alpha_\star(1_\star) \in
					\mathcal{A}(\star,\star)$. Thus, we define
					$\delta:H_\star \to H_\star$ with $\delta_\star(1_\star)=a$.
					Giving us that
					$h^{-1}=\psi \big(\delta_\star(1_\star) \big) $.
						And as $\psi$ is a group isomorphism,
					\[ 1_\star = \psi^{-1}(h.h^{-1})=\psi^{-1}(h)\,. \psi^{-1}(h^{-1})
					=\big(\gamma_\star(1_\star)\big).\big(\delta_\star(1_\star)\big).\]
This gives us that $\delta$ and $\gamma$ are inverses, as
					\[ \text{ for any } k \in \mathcal{A} (\star,\star), \; \;
						\big( \gamma \circ \delta\big)_\star
						(k)
						= \gamma_\star\big(\delta_\star(k)\big)
						=\gamma_\star \big( \delta_\star(1_\star).k \big)
						=\big(\gamma_\star(1_\star)\big).
						\big( \delta_\star(1_\star) \big).k
					=1_\star.k=k.\]
					Thus, the collection $[ \mathcal{A} ^{op}, Set](H_\star,H_\star)$ is a group.

				\item \textbf{Applying Yoneda Lemma.}

					As the collection of elements of $G$ form a set, we get that $\mathcal{A}(\star,\star)$
					is a set. Hence, $\mathcal{A}$ is a locally small category. Because
					$\mathcal{A} ^{op}$ has the same number of morphisms as $\mathcal{A}$,
					it is also a locally small category, and we may apply Yoneda Lemma to it.
					Taking $A=\star$ and $X=H_\star$ in Theorem 2.1 (1), we get:
					\begin{equation} [ \mathcal{A} ^{op}, Set](H_\star,H_\star)
						\; \hat{\cong} \; H_\star(\star), \end{equation}
					where the isomorphism $\string ^$ is between sets.
				\item \textbf{Showing that $\string ^$ is a group isomorphism.}

					From the proof of Theorem 2.1, we know that the map
					$\string ^$ acts as $\alpha \mapsto \alpha_\star (1_\star)$.
					Hence, for any $\alpha,\beta:H_\star \to H_\star$,
					\begin{equation}
						\widehat {\alpha \circ \beta}= (\alpha \circ \beta)_\star(1_\star)
						=(\alpha)_\star \Big( (\beta)_\star (1_\star) \Big)=
						\Big((\alpha)_\star(1_\star) \Big) . \Big( (\beta)_\star (1_\star) \Big)
						=\hat{\alpha} . \hat{\beta}\;,
					\end{equation}
					where the second-last equality is due to (2) being applicable
					as $\Big((\beta)_\star (1_\star)\Big)$ is an element of $\mathcal{A} (\star,\star)$.
			\end{enumerate}

			Using I and II, we get that $[ \mathcal{A} ^{op}, Set](H_\star,H_\star)$
			is a group with all of it's elements being bijections on $G$.
			Thus, it is a subgroup of the symmetric group on G.
			Using III we have shown that, the isomorphism $\string ^$ in (3) is between groups,
			with the $LHS$ being the above mentioned subgroup.
			And $RHS$ being $\mathcal{A} (\star, \star)$, which is further
			isomorphic to group $G$:
		\[ G \; \overset{\psi}{\cong} \; \mathcal{A} (\star,\star) \; \hat{\cong} \;
			[ \mathcal{A} ^{op}, Set](H_\star,H_\star)\, \leq Sym(G).\]
			This is	precisely the statement of Cayley's theorem.\qedhere
		\end{proof}
		\subsection{Yoneda Embedding}%{{{ Embedding		Section
		\begin{definition} %{{{ Embedding of a category		Definition
			A category $\mathcal{A}$  is said to be embedded in a category $\mathcal{B}$ if
			and only if there exists a functor $F: \mathcal{A} \to \mathcal{B}$
			such that $F$ is full and faithful.
		\end{definition} %}}}
		\begin{lemma}
			If a functor is fully faithful, then it is injective on objects upto isomorphism.
		\end{lemma}
		\begin{proof}
			Let functor $F:\mathcal{A} \to \mathcal{B} $ be fully faithful. Suppose, for
			objects $A,B\in \mathcal{A} $ that $F(A)=F(B)$. We are going to show that
			$A\cong B$. As $F$ is full, there exists $f \in \mathcal{A} (A,B)$ such that
			$F(f)=1_{F(A)}\in \mathcal{B} \big(F(A),F(B)\big)$. Similarly, we also have that
			there exists $g \in \mathcal{A} (B,A)$ such that
			$F(g)=1_{F(B)}\in \mathcal{B} \big(F(B),F(A)\big)$. Because $F$ is a functor,
			\begin{align}
				F\big( g \circ f \big)&=F(g) \circ F(f) = 1_{F(B)} \circ 1_{F(A)}=
				1_{F(A)} \circ 1_{F(A)}=1_{F(A)}=F(1_A); \\
				F\big( f \circ g \big)&=F(f) \circ F(g) = 1_{F(A)} \circ 1_{F(B)}=
				1_{F(B)} \circ 1_{F(B)}=1_{F(B)}=F(1_B).
			\end{align}
			As $F$ is faithful, (5) gives us that $g\circ f=1_A$ and (6) gives us
			$f \circ g = 1_B$. Hence, $A\cong B$.

		\end{proof}

		We are going to define a functor $H_\bullet$, from locally small category $\mathcal{A}$ to the presheaf category on
		$\mathcal{A} $, as taking any object $A \in \mathcal{A} $ to the functor $H_A$.
		And for any $X,Y,K \in \mathcal{A}$, taking morphism $f\in \mathcal{A} (X,Y)$
		to the natural transformation whose $K^{th}$-component is defined as taking any $k \in H_X(K)$
		to $f \circ k \in \mathcal{A} (K,Y)$.

		%{{{ H.		Proposition
		\begin{prop} $H_\bullet$ is a functor from $\mathcal{A}$ to $[\mathcal{A} ^{op},Set]$.
		\end{prop}
		\begin{proof}\setcounter{equation}{0}
			Fix any objects $K,L,M \in \mathcal{A} $.
			\begin{enumerate}[label=\Roman*]
				\item \textbf{ (Composition) }
					Let $f \in \mathcal{A} (K,L)$ and $g \in \mathcal{A} (L,M)$.
					As $H_\bullet(g \circ f)$ and $H_\bullet(g) \circ H_\bullet(f)$
					are natural transformations from $H_K$ to $H_M$,
					need to show that their	$X$-components are equal
					for any $X\in \mathcal{A} ^{op}$. Fix $X\in \mathcal{A} ^{op}$ and
					$k \in H_K(X)$,	and using the definition of $H_\bullet$, we get that
					\begin{align*}
						LHS&=\big( H_\bullet(g \circ f) \big) (k)=
						g\circ f \circ k \\
						\text{ and } RHS&=
						\Big( H_\bullet(f) \circ H_\bullet(g) \Big) (k)
						=\Big(H_\bullet(g)\Big) \big(f \circ k\big)
					=g\circ f \circ k. \end{align*}
				\item \textbf{ (Identity) } We will show that for any $g \in \mathcal{A} (K,L)$, $H_\bullet$
					respects the identities	of $K$ and $L$ in $\mathcal{A}$.
					Thus, for any object $X \in \mathcal{A} $, we need to show that
					$\Big( H_\bullet(g) \circ H_\bullet(1_K) \Big)_X = \Big( H_\bullet(g)\Big) _X
					=\Big( H_\bullet(g) \circ H_\bullet(1_K) \Big)_X $.
					Fix any morphism $p \in H_A(L)$.
					Using the definition of $H_\bullet$,
					we get the following equations.
					\begin{align*}
						\text{ Right Identity: }& \Big( \big( H_\bullet(g) \big) \circ \big( H_\bullet(1_K)
						\big) \Big)(p)
						=\big( H_A(g) \big) (1_K \circ p) =
						\big( H_A(g) \big) (p) \\
								\text{ Left Identity: }& \Big( \big( H_\bullet(1_L) \big) \circ \big( H_\bullet(g)\big) \Big)(p)
								=\big( H_\bullet(1_L) \big) (g \circ p) = g \circ p \circ 1_L=g \circ p
								= \big( H_\bullet(g) \big)(p)
							\end{align*}

				\end{enumerate}
				Hence, $H_\bullet$ is indeed a functor.
			\end{proof}
			%}}}

			\begin{theorem}{ \textbf{Yoneda Embedding}}
				Any locally small category $\mathcal{A}$ can be embedded in the presheaf category on
				$\mathcal{A}$.
			\end{theorem}
			\begin{proof} \setcounter{equation}{0}
				We will show that the functor from Proposition 1.10 is
				full and faithful. Fix any objects $X,Y \in \mathcal{A}$.
				\begin{enumerate}[label=\Roman*]
					\item To show that $H_\bullet$ is a full, we need to show that for every $\alpha
						\in [\mathcal{A} ^{op},Set](H_X,H_Y)$, there exists a morphism
						$f\in \mathcal{A} (X,Y)$ such that $H_\bullet(f)=\alpha$. Thus, we need
						to show that their $K$-components are equal for every $K \in \mathcal{A}$.
						Using the definition of $H_\bullet(f)$, this amounts to showing that
						\begin{equation}  \text{ for any morphism } k \in H_X(K),
							\Big(H_\bullet(f)\Big)_K^{}(k)=\alpha_K(k), \text{ that is }
							f\circ k=\alpha_K(k).
						\end{equation}
						Because $\alpha_X$ goes from $H_X(X)$ to $H_Y(X)$, $\alpha_X(1_X)$ is a morphism
						in $\mathcal{A}(X,Y)$. We will show that choosing this morphism to be $f$
						will give us the required result, that is $\big( \alpha_X(1_X) \big) \circ
						k=\alpha_K(k)$. Using the naturality of $\alpha$,
						\[ \text{ we get that }
							\begin{tikzcd}
								H_X(X) \arrow[swap]{d}{\alpha_X} \arrow{r}{H_X(k)}
			& H_X(K) \arrow{d}{\alpha_K}\\
			H_Y(X) \arrow[swap]{r}{H_Y(k)}
			& H_Y(K)
							\end{tikzcd}
							\text{, that is }
							\begin{tikzcd}
								\mathcal{A} (X,X) \arrow[swap]{d}{\alpha_X} \arrow{r}{\string _ \circ k}
			& \mathcal{A} (K,X) \arrow{d}{\alpha_K}\\
			\mathcal{A} (X,Y) \arrow[swap]{r}{\string _ \circ k}
			& \mathcal{A} (K,Y)
							\end{tikzcd}
						\text{ commutes. }\]
						Thus, for the identity morphism $1_X \in \mathcal{A} (X,X)$, we get the following
						\[ \Big(H_Y(k) \circ \alpha_X\Big)(1_X)=\Big(\alpha_K \circ H_X(k)\Big)(1_X)
							\implies  \alpha_X(1_X) \circ k \implies
						\alpha_K(1_X) \circ k=\alpha_K(k)\]
						Thus, we have that $H_\bullet$ is a full functor.

					\item  Fix any morphisms $f,g$ in $\mathcal{A} (X,Y)$ and suppose
						$H_\bullet(f)=H_\bullet(g)$. In order to show $H_\bullet$ is faithful,
						we need to show that $f=g$. As $H_\bullet(f)$ and $H_\bullet(g)$ are equal natural
						transformations, we have that the action of their $X$-components is equal. Thus,
						in particular, for the identity of $X$, $\big(H_\bullet(f) \big)_X (1_X)
						= \big(H_\bullet(g) \big)_X (1_X)$. Using the definition of $H_\bullet$, we get
						that $f \circ 1_X = g \circ 1_X$. And as both $g$ and $f$ are morphisms
						from $X$, we get that $f=g$.
				\end{enumerate}
			\end{proof}

			\section{Prorelations}
			\begin{definition}%{{{ Prorelation		Definition
				A prorelation is a partially ordered, down-directed, up-set of relations $X \to Y$.\\
				That is, $P \subseteq \mathcal{P}(X \times Y)$ is a prorelation if it satisfies the
				following conditions:
				\begin{enumerate}[label=(\roman*)]
					\item Partial Order: Containment of relations defines a partial order.
						That is, $r \subseteq s$  meaning that for any $(x,y) \in X \times Y$,
						if  $(x,y)\in r\;$ then $\;(x,y)\in s$.
					\item Down-directed: For any $r,s \in P$, there exists $t \in P $ such that
						$t\subseteq r \text{ and } t \subseteq s$.
					\item Up-set: For any relation $u:X\to Y$, if there exists $p \in P$ such that
						$p \subseteq u $ then $u \in P$.
				\end{enumerate}
			\end{definition}
			\begin{example} %{{{ 		Example
				We will define a prorelation on real numbers.
				For any positive real number $\epsilon$, define a relation on $\mathbb{R}$ as
				$A_\epsilon=\{ (x,y) |\; |x-y|<\epsilon \}$.
				The collection of all relations on $\mathbb{R}$ that contains some
				$A_\epsilon$ will be a prorelation, $K$ on $\mathbb{R}$.
				That is, $K=\{ a:\mathbb{R} \to \mathbb{R} \, | \; a \supseteq A_\epsilon \text{ for some } \epsilon>0 \; \}$ forms a prorelation.
				If $k,l \in K$, then there exist $\delta ,\, \epsilon > 0$ such that
				$k \supseteq A_\delta$ and $l \supseteq A_\epsilon$. Thus, the relation $A_{
				\frac{\delta+\epsilon}{2}}$ is in both $k$ and $l$. Moreover,
				$K$ is an up-set by definition.
			\end{example}
			\begin{definition}%{{{ Composition of prorelations	Definition
				A prorelation $P:X \to Y$ can be composed to a prorelation $Q:Y \to Z$ by taking
				composition of the relations belonging to them. Then, the set $Q.P$ is defined as
				$\; Q.P=\{q \circ p : p \in P \text{ and } q \in Q  \}$.
			\end{definition}
			%{{{Composition of prorelations
			\begin{lemma}
				Composition of two prorelations is a prorelation.
			\end{lemma}
			\begin{proof}\setcounter{equation}{0}

				For prorelations $P:X \to Y$ and $Q:Y \to Z$, need to show that Q.P is a prorelation.
				\begin{enumerate}[label=(\roman*)]
					\item (Partial Order) Inclusion of relations gives a partial order.
					\item (Down-Directed) If $k,k' \in Q.P$, then $k=q\,p$ and $k'=q'\,p'$ for some
						$q,q' \in Q$ and $p,p' \in P$. Because $Q$ and $P$ are prorelations,
						and hence down-directed sets there exists, $a \in Q$ such that
						$ a\subseteq q,q'$ and $b \in P$ such that $b \subseteq p,p'$. Thus,
						giving an element, $a\circ b$ of $Q.P$ such that $a \circ b \subseteq k, k'$.
					\item (Up-Set) Let $l:X \to Z$ be a relation, and $k \in Q.P$ such that $l \supseteq k$.
						Define relations $ p:X \to Y $ and $q:Y \to Z$ as,
						$p=\{(x,y): x \in Dom(l) \text{ and } y \in Y\}$ and
						$q=\{(y,z):  y \in Y \text{ and } z \in range(l)\}$.
						Because $k\in Q.P$, there exist $q'\in Q$ and $p' \in P$ such that
						$k = q' \circ p'$. Thus by definition of $p$ and $q$, we get that
						$p \supseteq p'$ and $q \supseteq q'$. Hence $p \in P$ and $q \in Q$ because
						$P$ and $Q$ are up-sets, which gives us that $q\circ p \in Q.P$.
						For any $(x,z) \in l$, by definition
						of $p$ and $q$, we get that for every $y \in Y$, $(x,y) \in p$ and
						$(y,z) \in q $. By
						definition of composition, this gives that $(x,z) \in q \circ p$,
						giving that $l \subseteq q \circ p$. And, by definition
						of $q \circ p$ we get that $l \supseteq q \circ p$. Finally giving that
						$l=q \circ p \in Q.P$. \qedhere
				\end{enumerate}
			\end{proof}
			%}}}
			\begin{definition} %{{{ Containment of Prorelations		Definition

				For prorelations $P,Q:X\to Y$, if for each $q \in Q$, there exists $p \in P$ such that
				$p \subseteq q$, then we write $P\leq Q$.
			\end{definition}
			\begin{definition} %{{{ Opposite relation		Definition
				For a relation $r:X \to Y$, it's opposite relation $r^o:Y \to X$ is defined as
				\[ (y,x) \in r^o \text{ if and only if } (x,y)\in r  \text{ for } x\in X \text{ and }
				y \in Y.\]
			\end{definition}
			%{{{Recurring Lemmas
			\begin{lemma}
				For any function $f:X \to Y$, $\; \Delta_X$ is contained in the composition $ f^o \circ f$.
			\end{lemma}
			\begin{proof}\setcounter{equation}{0}

				As $f$ is a function, it must be defined on every element of it's domain. Thus, for every
				$x \in X$, there exists some $(x,y)$ in $f$. By definition of $f^o$, $(y,x)$ is in $f^o$.
				Hence, by definition of composition, $(x,x)$ is in $f^o \,f$.
			\end{proof}
			\begin{lemma}
				For any relation $r:X \to Y$, the composition $r\circ r^o$ is contained in $\Delta_Y$.
			\end{lemma}
			\begin{proof}\setcounter{equation}{0}

				Suppose there exist $x \in X$ and $y \in Y$ such that
				$x\,r\,y$. By definition of $r^o$, this
				gives us that $y\, r^o \,x$. Using definition of composition,
				$y\, r^o \,x \,r\,y$ gives that	$y\, (r \circ r^o)\, y$.
			\end{proof}
			\begin{lemma} For relations $r,s:X\to Y$ and $t:Y\to Z$, if $\;r \subseteq s$ then $(t \circ r) \subseteq (t \circ s)$.
			\end{lemma}
			\begin{proof}\setcounter{equation}{0}

				Suppose relations $r,s$ and $t$ are as given above, and let $x \,(t \, r)\,z$.
				By definition of composition, there
				exists, $y \in Y$ such that $x\,r\,y$ and $y\,t\,z$.
				Using the hypothesis, as $r \subseteq  s $,
				$x\,r\,y$ gives $x\,s\,y$. And via composition of $x\,s\,y$ with $y \, t \, z $, we get
				$x\,(t\,s)\, z$. We started with any element of $(t \circ r)$ and showed that
				it must also be in $t \circ s$ and thus have that $(t \circ r) \subseteq (t \circ s)$.
			\end{proof}
			\begin{lemma} For relations $r:X\to Y$ and $s,t:Y\to Z$, if $s\subseteq t$ then
				$(s \circ r) \subseteq (t \circ r)$.
			\end{lemma}
			\begin{proof}\setcounter{equation}{0}

				Suppose relations $r,s$ and $t$ are as given above, and let  $x \,(s \, r)\,z$.
				By definition of composition of relations, we get that there exists
				some $y \in Y$ such that $x\,r\,y$ and $y\,s\,z$. Because $s \subseteq t$, $y\,s\,z$
				implies that $y\,t\,z$. Taking the composition, $x\,r\,y\,s\,z$ yields $x(t\,r)z$.
				We started with any element of $(s \circ r)$ and showed that it must also be in
				$t \circ r$ and thus have that $(s \circ r) \subseteq (t \circ r)$.
			\end{proof}
			%}}}

			\section{Quasi-Uniform Spaces}
			\begin{definition} %{{{ Quasi-uniformity		Definition
				A prorelation $P$ on a set $X$ is said to be a quasi-uniformity
				if it satisfies the following conditions:
				\begin{enumerate}[label=(\roman*)]
					\item Every relation in $P$ is reflexive. That is,
						for each $p \in P$, if $x \in X$ then $(x,x) \in p$.
					\item For each $p$ in $P$, there exists $p'$ in $P$ such that
						$p' \circ p' \subseteq p$.
				\end{enumerate}
			\end{definition}
			\begin{example} %{{{ 		Example
				We will show that the prorelation $K$, defined in Example 3.2 is a quasi-uniformity.
				The definition $A_\epsilon=\{ (x,y) |\;	|x-y|<\epsilon \}$ implies that each $A_\epsilon$
				is reflexive. And as every relation in $K$ contains some $A_\epsilon$,
				it must be reflexive as well, hence definition 4.1 (i) holds for $K$.
				Now we are going to show that definition 4.1 (ii) holds for $K$. Fix any
				relation $a \in K$, so, by definition of $K$, there exists $\epsilon$ such that
				$b\supseteq A_\epsilon$. Using $|x-y|=|y-x|$ we get that $A_\epsilon$ is symmetric.
				Thus, for any $\epsilon$, $A_\epsilon \circ A_\epsilon \subseteq A_\epsilon \subseteq b$.
			\end{example}
			\begin{definition} %{{{Quasi-uniform space Definition
				If $X$ is a set, and $A$ is a quasi-uniformity on $X$, then (X,A) is a quasi-uniform space.
			\end{definition}
			\begin{definition} %{{{ Uniformly Continuous function 		Definition
				A function, $f:(X,A) \to (Y,B)$ is said to be uniformly continuous if $f.A \leq B.f$.
				That is, for each $b \in B$, there exists $a \in A$ such that
				$f \circ a \subseteq b \circ f$. Meaning that $
				\begin{tikzcd}
					X \cen{dr} \arrow[swap]{d}{A} \arrow{r}{f}
		& Y \arrow{d}{B}\\
		X \arrow[swap]{r}{f}
		& Y
				\end{tikzcd}
				$.
			\end{definition}
			%{{{ A.A=A
			\begin{lemma} If $A$ is a quasi-uniformity on a set X, then $A.A=A$
			\end{lemma}
			\begin{proof}\setcounter{equation}{0}

				Fix any $a \in A$, as A is a quasi-uniformity, $\exists b \in A: bb \subseteq a$,
				we get that $A.A \leq A$. And as $A$ is a prorelation, and is hence down-directed,
				$\exists c \in A : a.a \supseteq c$, giving that $A.A \geq A$
			\end{proof}
			%}}}

			\subsection{Categories QUnif and ProMod}
			%{{{ Two categories
			We define a category called QUnif as having quasi-uniform spaces as objects
			and uniformly continuous maps between them as morphisms.
			With the composition of morphisms defined as that of functions,
			and identity of object $(X,A)$ is the identity function on set $X$.
			\begin{prop} QUnif is a category.\end{prop}
			\begin{proof}\setcounter{equation}{0}
				\begin{enumerate}[label=(\roman*)]
					\item (Associativity) The composition of functions is associative by definition.
					\item (Identity) For each object $(X,A)$, the identity function
						$\Delta_X:(X,A) \to (X,A)$ is uniformly continuous as
						$\Delta_X.A=A \leq A=A.\Delta_X$.
						\qedhere
				\end{enumerate}
			\end{proof}
			\begin{definition}%{{{ Promodule		Definition
				A prorelation, $\phi:X \carrow Y$ is called a promodule $\phi: (X,A) \carrow (Y,B)$  if it
				satisfies:
				\[ \phi.A \leq \phi \text{ and } B. \phi \leq \phi \]
			\end{definition}

			Now, we define a 2-category called ProMod as having quasi-uniform spaces as 0-cells and the
			promodules between them being 1-cells.
			The promodule $A$ will work as the identity of $(X,A)$.

			Let promodules $P,Q: (X,A) \carrow (Y,B)$. Then, there
			is a 2-cell from $P$ to $Q$ if and only if $P \leq Q$ as prorelations.
			The identity 2-cell of $P$ is the 2-cell corresponding to $P\leq P$.
			For the definition of 2-category, we have referred to \cite{Riehl_2016}.

			\begin{prop}%{{{ ProMod		Definition
				ProMod, as described above is a 2-category.
			\end{prop}
			\begin{proof}\setcounter{equation}{0}
				In order to show that ProMod is a 2-category, need the following:
				\begin{enumerate}[label=(\alph*)]
					\item (1-Identities) For each quasi-uniform space $(X,A)$,
						$A:(X,A) \carrow (X,A)$ a promodule because $A.A=A$ by Lemma 2.2.1.
					\item (1-Composition) Need composition of promodules to be a promodule.\\
						Let $\phi:(X,A)\carrow (Y,B)$ and $\psi:(Y,B)\carrow (Z,C)$ be promodules.
						To show that $\psi.\phi:(X,A) \carrow (Z,C)$ is a promodule, need it to be a
						prorelation that satisfies the two conditions required to be a promodule:
						\begin{enumerate}[label=(\roman*)]
							\item By Lemma 1.2.1, prorelations are closed under composition.
								Hence, $\psi.\phi$ is a prorelation
							\item Need to show that $\psi.\phi.A \leq \psi.\phi$. So, Fix
								$p \in \psi$ and $q \in \phi$. As $\phi$ is a promodule,
								$\phi.A\leq \phi$ gives that there exists
								$ q' \in \phi \text{ and } a\in A$ such that
								$q'\,a \subseteq q$. Thus,
								$p\,q'\,a \subseteq p\,q$.
							\item Need to show that $C.\psi.\phi \leq \psi.\phi$.Fix $p \in \psi$
								and $q\in \phi$. Because
								$\psi$ is a promodule, $C.\psi \leq \psi$ gives that
								there exists $c\in C$ and $p' \in \psi$ such that
								$c\,p' \subseteq p$. Thus, $c\,p'\,q \subseteq p\,q$
						\end{enumerate}
					\item (2-Identities) As every promodule is contained in itself, always have $\psi \leq \psi$.
						Define this comparison to be the identity 2-cell for $\psi$ and denote it by $\leq_\psi$

					\item (Vertical 2-composition) For promodules $\psi,\phi,\delta:(X,A) \carrow (Y,B)$,
						if there is is a 2-cell from $\psi$ to $\phi$ and another one from $\phi$
						to $\delta$ i.e. $\psi \leq \phi \leq \delta$, then by transitivity
						of the partial order, $\psi \leq \delta$ i.e. there's a 2-cell from
						$\psi$ to $\delta$.
					\item (Horizontal 2-composition) If there are promodules
						$\psi,\psi':(X,A) \carrow (Y,B)$ and
						$\phi,\phi':(Y,B) \carrow (Z,C)$ such that $\psi \leq \psi'$ and $\phi \leq
						\phi'$, need to show that $\psi.\phi \leq \psi'.\phi'$. Fix $p' \in \psi'$ and
						$q' \in \phi'$. As $\psi\leq\psi'$, $\exists p\in \psi: p \subseteq p'$
						and as $\psi\leq\psi'$, $\exists q \in \phi: q \subseteq q'$.
						Thus, $p\,q\subseteq p'\,q'$
					\item (1-Identity) Need to show that for any promodule $\phi:(X,A) \carrow (Y,B)$,
						$\phi.A=\phi=B.\phi$. By quasi-uniformity of $A$, every $a \in A$, is
						reflexive. Thus, for any $p \in \phi$ and $a \in A$,
						$p=p. \Delta_X \subseteq p \,a$
						giving that $\phi \leq \phi.A$. And as $\phi$ is a promodule,
						$\phi \geq \phi.A$. Hence, by anti-symmetry of the partial order, $\phi=\phi.A$.

						Similarly, By quasi-uniformity of $B$, every $b \in B$, is
						reflexive. Thus, for any $p \in \phi$ and $b\in B$, $p=\Delta_Y.p\subseteq
						b\,p$ giving that $\phi \leq B.\phi$. And as $\phi$ is a promodule,
						$\phi \geq B.\phi$. Hence, $\phi=B.\phi$.
					\item (1-Associativity) As composition of relations is associative, so too is the
						composition of prorelations directly giving that composition of promodules
						i.e. 1-cells is associative.
					\item (Vertical 2-Identity) Let $\leq:\psi \to \phi$ be a 2-cell i.e.
						$\psi \leq \phi$. By our definition of identity 2-cell, $\leq_\psi.\leq_1$
						means precisely that $\psi \leq \psi \leq \phi $, and by transitivity,
						this is equivalent to $\psi \leq \phi$.
						Similarly, $\leq_1.\leq_\phi$ means exactly that $\psi
						\leq \phi \leq \phi $, and this is equivalent to $\psi \leq \phi$.
					\item (Vertical 2-Associativity) Associativity of the partial order on promodules
						directly gives the associativity of composition of 2-cells in ProMod.
					\item (Horizontal 2-Identity) Let $\psi,\phi: (X,A)\carrow (Y,B)$ be promodules.
						For any 2-cell $\leq:\psi \to \phi $, need to show that the 2-cell given
						by the horizontal composition, $\leq*\leq_A$ is equal to $\leq$, as well as
						equal to $\leq_B *\leq$. So, it's required that $\psi.A \leq \phi.A
						\iff \psi \leq \phi \iff B.\psi \leq B.\phi$. And this holds as a
						direct consequence of (f).
					\item (Horizontal 2-Associativity) As there's a unique 2-cell between any two
						promodules, and composition of promodules is associative,
						horizontal composition of 2-cells is associative.
					\item (2-Identity) For promodules $\psi:(X,A)\carrow (Y,B)$ and
						$:\phi(Y,B) \carrow (Z,C)$ need $(\leq_\psi * \leq_\phi)=\leq_{\psi.\phi}$.
						Both sides of the required equality are 2-cells $\leq:\psi.\phi
						\to \psi.\phi$.	Thus, they are equal by the uniqueness of 2-cells between
						any two 1-cells.
					\item (2-Interchange) Let $\psi,\phi,\delta:(X,A) \carrow (Y,B)$ and
						$\psi',\phi',\delta':(Y,B) \carrow (Z,C)$ be promodules. For
						2-cells $\leq_1:\psi \to \phi $,$\leq_2:\phi \to \delta $,
						$\leq_a:\psi' \to \phi' $ and $\leq_b:\phi' \to \delta'$,need to show
						$(\leq_b.\leq_a)*(\leq_2.\leq_1)=(\leq_b*\leq_2).(\leq_a*\leq_1)$.
						Both RHS and LHS are 2-cells from $\psi.\psi'$ to $\delta.\delta'$ and are
						hence equal. \qedhere
				\end{enumerate}
			\end{proof}
			%}}}
			\subsection{Functors between QUnif and ProMod}
			%{{{Two functors
			We now define a functor from the category QUnif to ProMod, as fixing objects and taking uniformly
			continuous maps $f:(X,A)\to(Y,B)$ to $f.B$.

			\begin{prop} The mapping defined above, $(\string _)_*:\text{QUnif}^{op} \to \text{ProMod}$ as
				\begin{enumerate}[label=(\alph*)]
					\item for $(X,A) \in \text{QUnif}$, $(X,A)_*:=(X,A) \in \text{ProMod}$
					\item for $f:(X,A) \to (Y,B)$ in QUnif,
						$f^* := B.f$
				\end{enumerate}
				is indeed a functor.
			\end{prop}
			\begin{proof}\setcounter{equation}{0}

				We will first show that $B.f={b \circ f: b \in B}$
				is a promodule, and then that $(\string _)_*$ defines a functor.
				\begin{enumerate}[label=(\roman*)]
					\item (Partial-Order) Inclusion of relations acts as the partial order.
					\item (Down-Directed) Fix any $k,k'$ belonging to $B.f$. Thus, there exist
						$b,b'$ in $B$ such that $k=b\,f$ and $k'=b\,f$. Using down-directedness
						of $B$, there exists a $c \in B$ such that $c \subseteq b,b'$. Hence,
						by Lemma 2.4.4, $c\,f \subseteq k,k'$.
					\item (Up-set) Let $k$ belong to $B.f$ and $l:(X,A) \to (Y,B)$ be a
						uniformly continuous function such that $l \supseteq k$.
						Define a relation $b':= \{(f(d),l(d)): d \in Dom(l) \}$.
						By definition, for any $x \in X$ and $z \in Y$ such that $(x,z) \in l$, we get that
						$(f(x),z)\in b'$. And $l \supseteq k=b\, f$ implies $Dom(l)  \supseteq Dom(f)$
						giving $(x,f(x)) \in f$. Thus, by definition of composition,
						$(x,z) \in b'.f$. Conversely, suppose $(x,z) \in b'.f$.
						By definition of composition, there exists $f(x)\in Y$ such that
						$(f(x),z) \in b'$. Again using the definition of
						$b'$, we get that $z=l(x)$ i.e. $(x,z)\in l$. Hence, $l=b'\, f$.
						Now we will show that $b'\supseteq b$. Because $b'\, f=l \supseteq k=b\, f$, for any
						$x\in X$ we have that $b'\big(f(x)\big) \supseteq b\big(f(x)\big)$. Thus,
						$b'|_{f(x)} \supseteq b|_{f(x)}$. By down-directedness of $B$,
						the restriction $b|_{f(x)} \subset b$ implies $ b(x)|_{f(x)} \in B$.
						Finally, $b' \supseteq b'|_{f(x)} \supseteq b|_{f(x)}$ gives $b' \in B$.
						Hence, $b'.f \in B.f$.
					\item Need to show that $(B.f).A \leq B.f$. So, fix any $b\in B$, we will find
						$b' \in B$ and $a\in A$ such that $b' \, f\,a \subseteq bf$.
						By quasi-uniformity of $B$, there exists $b' \in B$ such that $b'\,b'
						\subseteq b$. Using Lemma 2.4.3, we get that $b'\,b'\,f \subseteq b\,f$.
						As $f$ is uniformly continuous, $f.A \leq B.f$ gives that there is some
						$a \in A \text{ such that } f\,a \subseteq b'\,f$. Using this in the
						previous inequality, we get $b'\,f\,a \subseteq b'\,b'\,f\subseteq  b\,f$.
					\item Need to show that $B.B.f \leq B.f$. Fix any $b \in B$, we will find
						$b' \in B$ such that $b'\,b'\,f \subseteq b\,f$.
						By quasi-uniformity of $B$, there exists $b \in B$ such that
						$b'\,b' \subseteq  b$. Using Lemma 2.4.4, we get $b'\,b'\,f \subseteq bf$.
				\end{enumerate}
				Thus, $B.f$ is a promodule. We now proceed to show that $(\string _)_*$ defines a functor.
				\begin{enumerate}[label=(\roman*)]
					\item (Composition) Need to show that $(g\circ f)_*=g_*f_*$ i.e. $C.g.f=C.g.B.f$.

						In order to show $C.g.f \leq C.g.B.f$, fix any $b\in B, c\in C$.
						We will show that $c\,g\,f \subseteq c\,g\,b\,f$. As $f$ is uniformly
						continuous, $f.A \leq B.f$ gives that there exists $a \in A$ such that
						$f\,a \subseteq b\,f$. Using Lemma 2.4.3, we get $(c\,g)f\,a \subseteq
						(c\,g)b\,f$. Now, using reflexiveness of $a$, we get $c\,g\,f \subseteq
						c\,g\,b\,f$.

						Now, to show that $C.g.f \geq C.g.B.f$. Fix any $c \in X$, we will find $c' \in C$ and
						$b \in B$ such that $c\, g\, f\, \supseteq c\, g\, b\, f\, $. By quasi-uniformity of C,
						there exists $c' \in C$ such that $c \subseteq  c'\, c'$. Using Lemma 2.4.4
						gives that $c\, (g\, f)\supseteq c'\, c'\, (g\, f) $. Because $g$ is uniformly
						continuous, $C.g \geq g.B$ gives us $b\in B$ such that $g\, c' \supseteq b\, g$.
						Using this in the previous inequality gives that $c\, g\, f \supseteq c'\, g\, b\, f$.

					\item(Identity) let $(X,A)$ be in object of QUnif and
						$1_{(X,A)}:(X,A)\to(X,A)$ be the identity function on $(X,A)$. That is,
						$1_{(X,A)}$ is defined as $x\mapsto x$.
						Need to show that $(1_{(X,A)})_*=1_{(X,A)_*}$. Using
						functor's definition, $LHS=(1_{(X,A)})_*=A.(1_{(X,A)})=A.1_{(X,A)}=A$
						and $RHS=1_{(X,A)_*}=1_{(X,A)}$
						Using Proposition 3.2(f), we get that $A=1_{(X,A)}=RHS$. \qedhere
				\end{enumerate}
			\end{proof}


			Similar to the above functor, we define a contravariant functor from the category QUnif to ProMod,
			as fixing objects and taking uniformly continuous maps $f:(X,A)\to(Y,B)$ to $B.f^o$.

			\begin{prop} The mapping defined above, $(\string _)^*:\text{QUnif}^{op} \to \text{ProMod}$ as
				\begin{enumerate}[label=(\alph*)]
					\item for $(X,A) \in \text{QUnif}^{op}$, $(X,A)^*:=(X,A) \in \text{ProMod}$
					\item for $f:(X,A) \to (Y,B)$ in QUnif,
						$f^* := f^o .B$
				\end{enumerate}
				is indeed a functor.
			\end{prop}
			\begin{proof}\setcounter{equation}{0}

				Showing that $f^o .B: (Y,B) \carrow (X,A)$ is a promodule.\\
				So, need to show $f^o .B$ a prorelation $Y \to X$
				and that $(f^o .B).B \sqsubseteq f^o .B$ and $A.(f^o .B) \sqsubseteq f^o .B$ \\
				To show prorelation, \begin{enumerate}[label=(\roman*)]
					\item (Partial-order) Inclusion of relations i.e. for $k=f^o \circ b$ and
						$k'=f^o \circ b'$ in $f^o .B$ , $k \subseteq k' \iff b \subseteq b'$
					\item (Down directed) for $k,k' \in f^o .B$, need that $\exists l \in f^o .B
						\text{ such that } l \subseteq k,k'$

						Fix $k,k' \in f^o .B \implies \exists b,b' \in B : k=f^o \circ b \text{ and }
						k' = f^o \circ b'$

						By down-directedness of $B$, there exists $c \in B$ such that
						$ c \subseteq b,b'$, define $l=f^o \circ c$.
						Now, using Lemma 2.4.3 gives  $l= f^o \circ c \subseteq k,k'$.
					\item (Up-set) for a relation $l:Y \to X$ and $k \in f^o .B$ such that $l \supseteq k$
						, need $l \in f^o .B$

						Let $b\in B$ be such that $k=f^o \circ b$ and define
						$b':=\{(y,y'): y \in Dom(l) \text{ and } y' \in (f^o)^{-1}(l(y))\}$\\
						As $l\supseteq k=f^o \circ b$, $Dom(b')=Dom(l)\supseteq Dom(b)$
						\\ and $range(l) \supseteq range(f^o \circ b)\implies
						\forall y \in Dom(b), range (b')=(f^o )^{-1}(l(y)) \supseteq (f^o)^{-1}(f^o \circ b ) = range(b)$\\
						Now, by definition of $b'$, $f^o \circ b' \supseteq l$. To show
						$f^o \circ b \subseteq l$ , \\
						$(x,y)\in f^o \circ b' \implies \exists z \in Y: (x,z)\in b' \text{ and }
						(z,y) \in f^o \implies x \in Dom(l) \text{ and } z \in l(x)$ i.e.
						$(x,z) \in l$

				\end{enumerate}
			\item	To show $(f^o .B).B \leq f^o .B$, need that $\forall b \in B,
				\exists b' \in B : f^o \circ b' \circ b' \subseteq f^o \circ b$,\\
				Fix any $b \in B$, as B is a quasi-uniformity, $\exists b' \in B : b' \circ b' \subseteq b
				\implies f^o \circ b'\circ b' \subseteq f^o \circ b$

				To show $A.(f^o .B) \leq f^o .B$, need that $\forall b \in B$,
				$\exists b' \in B, a\in A : a \circ f^o \circ b' \subseteq f^o \circ b$,\\
				As $f$ is uniformly continuous, $f.A\leq B.f$ i.e. $\forall b \in B, \exists a \in A
				: f \circ a \subseteq b \circ f
				\implies a= f^o \circ f \circ a \subseteq f^o \circ  b \circ f $   \\
				Fix any $b \in B, \text{ so, } \exists b' \in B : b'b' \subseteq b$
				And, for this $b', \exists a : a \subseteq f^ob'f \implies af^ob' \subseteq f^ob'ff^ob'
				\subseteq f^o b'b' \subseteq f^o b \implies af^ob' \subseteq f^o b$\\
			\item	Now, need to show that $(\string _)^*$ respects composition and identity.
				\begin{enumerate}[label=(\roman*)]
					\item (Composition) let $f,g$ be uniformly continuous,
						$(X,A) \xrightarrow{f} (Y,B) \xrightarrow{g} (Z,C)$
						need that $(g \circ f)^*= f^*.g^* $

						LHS=$(g \circ f)^*=(g \circ f)^o .C=(f^o \circ g^o).C$ and
						RHS=$f^*.g^* =(f^o .B).(g^o .C)$\\
						For equality, showing that LHS$\geq$RHS and LHS$\leq$RHS:

						To show $(f^o \circ g^o).C\geq(f^o .B).(g^o .C)$, need that
						$\forall c \in C, \exists b \in B, c' \in C : f^og^oc
						\supseteq f^obgc'$ \\
						Fix any $c \in C, \text{ so, } \exists c' \in C: c' \circ c' \subseteq c
						\implies f^o g^o c \supseteq f^o g^o (c'c')
						=f^o g^o (c' \Delta_Z c') \supseteq f^o g^o c'(gg^o)c'$ \\
						By uniform continuity of g, for $c'\in C,\exists b\in B: gb\subseteq c'g $
						\\Thus, $f^o g^o c \supseteq f^o g^o (c'g)g^oc' \supseteq
						f^o (g^o g)bg^o c'=f^o bg^o c'$.

						To show $(f^o \circ g^o).C\leq(f^o .B).(g^o .C)$, need that
						$\forall b \in B, c \in C, \exists c' \in C: f^o g^o c \subseteq f^obg^oc $
						\\Fix any $c\in C, b\in B$ will show that $c':=c$ works:\\
						As B is a quasi-uniformity, $\Delta_Y \subseteq b\implies f^o \Delta_Y
						g^o c=f^o g^o c \subseteq f^o b	g^o c=f^o b g^o c'$
					\item(Identity) let $(X,A)\in \text{ QUnif }^{op} $, and
						$1_{(X,A)}:(X,A)\to(X,A)$ as $x\mapsto x$ need that
						$(1_{(X,A)})^*=1_{(X,A)^*}$
						LHS=$(1_{(X,A)})^*=(1_{(X,A)})^o.A=1_{(X,A)}.A=A$. \\
						And as $RHS=1_{(X,A)^*}=1_{(X,A)}$
						Using Proposition 3.2(f), we get that $A=1_{(X,A)}=RHS$. \qedhere
				\end{enumerate}
			\end{proof}
			%}}}

			A quasi-uniform space $(X,A)$ defines a topological space as given by the
			following proposition that we borrow from \cite{Fletcher_Lindgren_1982}.
		A subfamily $\mathbb{B}$ of quasi-uniformity $A$ is called a base for $A$
		if each relation in $A$ contains a relation in $\mathbb{B}$.
			\begin{prop}
				Let $\mathbb{B}$ be the base for quasi-uniformity $A$ on $X$.
				For $x \in X$, define $\mathbb{B}(x)=\{B(x) | B \in \mathbb{B}\}$.
				Then there is a unique topology on $X$ such that for each $x\in X$,
				$\mathbb{B}(x)$ is a base for the neighborhood of $x$ in this topology.
			\end{prop}
			We skip the proof as we have no requirement of it. But refer the interested reader to
\cite{Fletcher_Lindgren_1982} for similar results.

			\begin{definition} %{{{Topologically Dense		Definition
				For any quasi-uniform space $(X,A)$, an element $x \in X$ is said to belong in the
				topological closure of set $M\subseteq X$ if and only if for each $a\in A$,
				there exists $y\in M$ such that $x\, a\, y$ and $y\, a\, x$.
			\end{definition}

			\begin{definition} %{{{ Fully Faithful and Dense		Definition
				Let $f:(X,A) \to (Y,B)$ be a uniformly continuous function.
				\begin{enumerate}[label=\Roman*]
					\item f is said to be fully faithful if and only if $f^* .f_*=A$.
					\item f is said to be fully dense if and only if $f_* .f^*=B $.
					\item f is said to be topologically dense of and only if $\overline{f(X)}=Y$.
				\end{enumerate}
			\end{definition}

			%{{{ Proposition 1
			\begin{prop}
				Fix a uniformly continuous map, $f:(X,A) \rightarrow (Y,B)$
				\begin{enumerate}[label=(\alph*)]
					\item f is fully faithful if and only if $A= f^o.B.f$, that is
						$A\geq f^o.B.f$
					\item f is fully dense if and only if for any $b\in B$, $\exists b' \in B$
						such that $b' \subseteq b\,f\,f^o \,b$
					\item f is topologically dense if and only if for any $b\in B$,
						$\; b \, f\, f^o\,b$ is reflexive
					\item f is fully dense if and only if f is topologically dense
				\end{enumerate}
			\end{prop}
			\begin{proof}\setcounter{equation}{0}

			\item
				\begin{enumerate}[label=(\alph*)]
					\item \begin{enumerate}[label=(\roman*)]
						\item $ (\implies) $ Let f be fully faithful i.e. $f^*.f_*=A
							\implies f^o .B.B.f=A $\\
							Need to show that $A= f^o .B.f$ i.e.
							$A\leq f^o .B.f$ and  $A\geq f^o .B.f$\\
							By hypothesis and quasi-uniformity of B,
							$A\geq f^o .B.B.f \geq f^o B.f $\\
							To show $A \leq f^o .B.f$, need that $\forall b \in B, \exists a
							\in A : a \subseteq f^o bf$\\
							Fix $b\in B$, hypothesis gives that $f^o .B.B.f \leq A$ so, \\
							$\exists a \in A: a \subseteq f^o bbf$ and also, by
							quasi-uniformity of B, for $b, \exists b' \in B : b'b' \subseteq b
							\implies f^o b'b'f \subseteq f^o bf$\\
							Combining the above two inequalities,
							$a \subseteq f^o bbf \subseteq f^o bf$\\
						\item $(\impliedby)$ Let $A=f^o .B.f$ need to show $A=f^o.B.B.f$ i.e.
							$A\geq f^o B.B.f \text{ and } A\leq f^o B.B.f$\\
							To show $A\geq f^o .B.B.f$, need to show that $\forall a\in A,
							\exists b,b' \in B : a \supseteq f^o bb'f$\\
							Have that $A\geq f^o .B.f$ and $B.B \leq B$\\
							So, fix $a \in A$, now $\exists b\in B: a \subseteq f^o bf$
							and for this b, $\exists b'\in B: b'b' \subseteq b$.
							Therefore, $a \supseteq f^o bf \supseteq f^o b'b' f$
							To show $A\leq f^o .B.B.f$, need $\forall b,b'\in B, \exists
							a\in A : a \subseteq f^o bb'f$ \\
							Before that, uniform continuity
							of f along with Lemma 2.1.1 gives that \\$f.A\leq B.f
							\implies A=f^o f.A \leq f^o.B.f$ \\
							So, fix $b,b' \in B $, now, as ,

							$A\leq f^o .B.f $
							giving\\ $\exists a \in A : a \subseteq f^o bf$ and
							$\exists a'\in A: a' \subseteq f^o b'f \implies \Delta_X \subseteq
							f^o b'	f$.\\
							Therefore $a=a\Delta_X \subseteq (f^o bf) (f^o b'f)
							\subseteq f^o bb'f$
					\end{enumerate}
				\item	\begin{enumerate}[label=(\roman*)]
					\item $(\implies )$ Let f be fully dense i.e. $B=f_*f^* = B.f.f^o.B$.
						showing that $\forall b \in B,\exists b'\in B: b'
						\subseteq bff^ob$:\\
						So, fix $b\in B$, as $B \leq B.f.f^o.B$, there exists $b' \in
						B$ such that $b' \subseteq bff^ob$.
					\item $(\impliedby)$ Suppose $\forall b \in B, \exists b' \in B:
						b' \subseteq bff^ob$. This gives $B\leq B.f.f^o .B$, in order
						to show equality, also need $B\geq B.f.f^o .B$.
						By quasi-uniformity of B, for any $b \in B$, $\exists b'\in B:b'b'
						\subseteq b$. Now, by Lemma 2.4.2,
						\[ f f^o \subseteq \Delta_Y \implies b'f f^o b' \subseteq b'
						\Delta_Y b'=b'b' \subseteq b\]

				\end{enumerate}
			\item \begin{enumerate}[label=(\roman*)]
				\item $(\implies)$ Let f be topologically dense. We will show that
					for any $b\in B, y \in Y$, $\;(y,y) \in bff^ob$. Fix any $ b\in B$ and
					$y \in Y$. As f is topologically dense, $\overline{f(X)}=Y $, implying
					that $y\in \overline{f(X)}$, by definition giving that
					\[ \exists x\in X \text{ such that }
					(f(x),y) \in b \text{ and } (y,f(x)) \in b \]
					Re-writing the above statement in terms of relations, and
					considering f as a relation:
					\begin{align}
						(f(x),y) \in b &\text{ gives }  x(b \circ f)y \text{ i.e. }
						y \in (b \circ f)(x)\\
						(y,f(x)) \in b &\text{ gives } f(x)\subseteq b(y)
					\end{align}
					Repeatedly applying Lemma 2.4.3 to (2),
					\[ f(x) \subseteq b(y) \implies
						\big(f \circ f^o \big)(f(x) \subseteq \big(f \circ f^o \big)b(y)
						\implies \big(f \circ f^o \circ f\big)(x) \subseteq
					\big(f \circ f^o \circ b \big)(y)\]
					Applying Lemma 2.4.1 to the final inequality
					in the above statement gives that
					\[f(x)= (f\circ  \Delta_X) (x)
						\subseteq \big(f \circ f^o \circ f\big)(x) \subseteq
					\big(f \circ f^o \circ b \big)(y)  \]
					Applying Lemma 2.4.3 and then using (1) on the above inequality completes the result:
					\[ f(x) \subseteq \big(ff^ob \big)(y)
						\implies \big(b \circ f\big)(x) \subseteq
					\big(bff^ob \big)(y) \implies y \in \big(bff^ob \big)(y) \text{ i.e. } y\big(bff^ob \big)y   \]
				\item $(\impliedby)$ Fix any $y \in Y$ and $b \in B$. Also, suppose that
					$\Delta_Y \leq bff^o b$. As f is a function with domain as X,
					$f^o :Y \to X$, $\phi \neq (f^o \circ b)(y)
					\subseteq  X$. So, fix $x\in (f^o \circ b)(y)$, going to show that
					$(f(x),y)\in b$	and $(y,f(x)) \in b$. Again, while viewing f as a
					relation.
					\[ \Delta_Y \leq bff^ob
						\implies \Delta_Y(y) \subseteq bf f^o b(y)=\big(bf\big)
					(f^o b(y))\]
					Last inequality of the above statement gives $y \in (bf)(x)
					\text{ i.e. } (f(x),y)\in b $.\\
					Applying Lemma 2.4.2 to f, and then using Lemma 2.4.4,
					\begin{equation*}
						ff^o  \subseteq \Delta_Y \implies f f^o b \subseteq \Delta_Yb=b
					\end{equation*}
					Thus $f f^o b(y) \subseteq b(y)$ and hence $f(x) \subseteq b(y) \implies
					(y,f(x)) \in b$
			\end{enumerate}
		\item \begin{enumerate}[label=(\roman*)]
			\item
				($\implies $) Let f be topologically dense. As B is a quasi-uniformity, for any $b \in B$,
				\begin{equation}\exists b' \in B : b'b' \subseteq b \text{ and } \Delta_Y \subseteq b'
					\implies b'=b'\Delta_Y \subseteq b'b' \subseteq b
				\end{equation}
				By the characterization of topologically dense in (c), have that $\Delta_Y \subseteq b'f f^o b'$.
				Now, using the (3) and Lemma 2.4.3,
				\[ \Delta_Y \subseteq b'f f^o b' \implies b'=b'\Delta_Y \subseteq b'b'f f^o b' \subseteq bf f^o b'
				\subseteq bf f^o b\]
				Hence, we have $b'\in B : b' \subseteq bf f^o b$ giving us that f is fully dense (from (b)).
			\item ($\impliedby$) From (b), we have for $b \in B$, the existence of $b' \in B$ such that $
				b' \subseteq bf f^o b$. As B is a quasi-uniformity, $\Delta_Y \subseteq b'$. So,
				$\Delta_Y \subseteq bf f^o b$, and from (c), this gives us that f is topologically dense. \qedhere
		\end{enumerate}
\end{enumerate}
\end{proof}
%}}}
\section{Yoneda Lemma in Quasi-Uniform Spaces}

In this section, we will look at Yoneda Lemma and Yoneda Embedding for Quasi-Uniform Spaces.
We use 1 to denote the quasi-uniform space with one element, that is, the quasi-uniform space
$(\{\star\},\{(\star,\star)\})$. Also, when unambiguous, we use $1$ to denote the quasi-uniformity
of the quasi-uniform space 1.

%{{{ A~ is a quasi-uniformity on PX
\begin{definition}%{{{ PX		Definition
	The set $PX$ is defined to be the collection of all promodules from the quasi-uniform space(X,A)
	to the quasi-uniform space 1.
	\[PX:=\{\psi :(X,A) \carrow 1 | \psi \text{ is a promodule} \}\]
\end{definition}
\begin{prop}%{{{ A~		Definition
	For any $a\in A$, $\tilde{a}$ is defined to be a relation $PX \to PX$ as
	\[ \text{ for } \phi,\psi \in PX, \; \phi \, \tilde{a} \, \psi \; \text{ only if } \;
	\phi \leq \psi.a \]
	The set, $\tilde{A}:=\{\tilde{a}:a \in A\}$ defines a quasi-uniformity on $PX$.
\end{prop}
\begin{proof}\setcounter{equation}{0}

	First need to show that $\tilde{A}$ is a prorelation,
	\begin{enumerate}[label=(\roman*)]
		\item (Partial order) For any two relations $\tilde{a},\tilde{b}:PX \to PX $,
			define $ \tilde{a} \leq \tilde{b}$ to be true only if $a \subseteq b$.
		\item(Down-Directed) Need for any $\tilde{a} ,\tilde{b} \in \tilde{A}$,
			the existence of some $\tilde{c} \in A$ such that $c \subseteq a,b$\\
			If $\tilde{a} ,\tilde{b} \in A$ then there exist $a,b \in A$. By down-directedness
			of $A$, there exists a $c \in A$ such that  $c \subseteq a,b$. Now the definition
			of $\tilde{A}$ gives that $\tilde{c} \in \tilde{A}$. And the definition of the
			partial order on $\tilde{A}$ ensures $\tilde{c} \leq \tilde{a} ,\tilde{b}$.
		\item (Upset) For any relation $l:PX \to PX$ , need that if  $\tilde{k}$ belongs to
			$\tilde{A}$ such that $l \geq \tilde{k}$, then $l \in \tilde{A}$.\\
			Fix any $k:PX \to PX$, and $\tilde{k} \in \tilde{A}$ such that $l\geq \tilde{k} $.
			As $k$ is a relation between promodules $X \carrow 1$, it can be thought
			of as a relation $a$ on X, defined as:
			\[a:=\{(x,y): x \in Dom(\psi) \text{ and }y \in Dom(\phi)
			\text{ whenever } \exists \psi,\phi \in PX: \psi l \phi\}\]
			So, $l\geq \tilde{k}$ gives that $\tilde{a}\geq \tilde{k}$ i.e. $a \supseteq k.$
			And as $A$ is an upper-set, we get $a\in A$. Now, by definition of $\tilde{A}$,
			$l \in \tilde{A}$.
	\end{enumerate}
	Secondly, need show that the other two conditions hold for $\tilde{A}$,
	\begin{enumerate}[label=(\roman*)]
		\item For all $\tilde{a} \in \tilde{A}$, need $\tilde{a}$ to be reflexive i.e
			if $\psi \in PX$ then $\psi \, \tilde{a} \, \psi$.\\
			By definition of $\tilde{a}$ , need to show that $\psi \leq \psi.a$.
			So, fix a $p \in \psi$, we will show that $p \subseteq p.a$.
			Quasi-uniformity of $A$ gives that $\Delta_X \subseteq a$. Hence, by Lemma 2.4.3,
			$p=p \, \Delta_X \subseteq p \,a$ .
		\item For all $\tilde{a} \in \tilde{A}$, need to find $\tilde{b}\in \tilde{A}$ such that
			$\tilde{b}\tilde{b} \leq \tilde{a} $\\
			Before showing the result, proving that for any $x,y \in A$,
			$\tilde{x} \, \tilde{y} \leq \widetilde{xy}$
			i.e. $\forall \psi, \phi \in PX$ , $\psi(\tilde{x} \, \tilde{y})\phi \implies
			\psi \, \widetilde{xy} \, \phi $.
			If $\psi_1(\tilde{a} .\tilde{b} )\psi_3$, then, the definition of composition
			gives that $\exists \psi_2$ such that $\psi_1 \, \tilde{b} \, \psi_2 \,
			\tilde{a} \, \psi_3$. Now, the definition of $\tilde{b}$ gives $\psi_1 \leq \psi_2
			\,b$ and that of $\tilde{a}$ gives $\psi_2 \leq \psi_3 \, a$. Combining
			these inequalities, $\psi_1 \leq \psi_2.b \leq \psi_3.ab$.
			Hence, by definition of	$\widetilde{ab}$, $\psi_1 \,(\widetilde{ab})\,\psi_3$.
			Now, to show the result, fix any $\tilde{a}\in \tilde{A}$. Therefore,$a\in A$, and
			by quasi-uniformity of $A$, $\exists b \in A: b \circ b \subseteq
			a$. Thus, by the partial-order defined on $\tilde{A}$, $\widetilde{bb} \leq \tilde{a}$.
			Now, transitivity of the partial order gives us the required result,
			$ \tilde{b} \, \tilde{b} \leq \widetilde{bb} \leq \tilde{a}$. \qedhere
	\end{enumerate}
\end{proof}
%}}}
%{{{ Yoneda Embedding
\begin{prop}[Yoneda Embedding]
\item For a quasi-uniform space $(X,A)$, function $y_X:X \to PX$ is defined by $x\mapsto x^*$ for $x \in X$.
	\begin{enumerate}[label=(\alph*)]
		\item $y_X:(X,A) \rightarrow (PX,\tilde{A})$ is a uniformly continuous map.
		\item $y_X:(X,A) \rightarrow (PX,\tilde{A})$ is fully faithful.
	\end{enumerate}
\end{prop}
\begin{proof}\setcounter{equation}{0}

\item \begin{enumerate}[label=(\alph*)]
	\item In order to show $y_X$ is uniformly continuous, need to show that $y_X.A \leq \tilde{A}.y_X $.
		By definition of $\leq$ , need $\forall a\in A, \exists b \in A:
		y_X \circ b \subseteq \tilde{a} \circ y_X $. Applying the relations
		to some element, x of the set X:
		\begin{equation} \big(y_X \circ b\big)(x) \subseteq \big( \tilde{a} \circ y_X\big)(x) \implies
		y_X(b(x)) \subseteq \tilde{a}(x^*) \end{equation}

		So, for the condition given by (4) to hold, if $y \in b(x)$, then it's required that
		$y^*=y_X(y) \in \tilde{a} (x^*)$ i.e. $x^* \tilde{a}y^*$. Using the definition of $x^*,y^*$
		and $\tilde{a}$,
		\begin{equation} x^* \tilde{a}y^* \iff x^o.A\leq y^o.A.a \iff
		\forall a' \in A, \exists a'' \in A: x^oa'' \subseteq y^oa'a  \end{equation}
		Now, fix any $a \in A$, $x\in X$. Thus, quasi-uniformity of A, gives $a'' \in A$ such that
		$a''a''\subseteq a$.Also, choose some $y \in a''(x)$. Hence, in
		order to show that the condition from (5) holds, need that
		$\forall b \in A, x^o a'' \subseteq y^oba$, and by applying the relations to an element z
		gives the following condition
		\begin{equation} \forall b \in B, \forall x \in X \text{ , }
		\big(x^oa''\big)(z) \subseteq \big(y^oba\big)(z) \end{equation}
		Examining the left side of (6),
		\[ \big( x^oa''\big)(z)=x^o (a''(z))= \begin{cases}
			\phi &\text{ if } x \notin a''(z) \\
			\star & \text{ if } \in a''(z)
		\end{cases} \]
		Thus, to show that (6) holds, need to show that (for any $b\in A$ and $z \in X$):
		\begin{equation} x \in a''(z) \implies z(y^oba)\star \text{ i.e. } y\in(ba)(z)
		\end{equation}
		To show that (7) holds, fix any $z\in X: x \in a''(z)$. Also, by our choice of $y$,
		have that $y \in a''(x)$. And as $b\in A$, it's reflexive, giving that $y \in b(y)$.
		So, by composition of relations, we get:
		\[ za''x \text{ , }  xa''y \text{ and } yby \implies z(a''a''b)y \implies z(ab)y \text{ i.e. }
		y \in (ba)(z)\]
	\item By using Proposition 2.3 (a), need to show that $A\geq y_X^o.\tilde{A}.y_X$ i.e. $\forall
		a\in A, \exists \tilde{b}\in \tilde{A} :  a \supseteq y_X^o \text{ } \tilde{b} \text{ } y_X $.
		Applying to an element, $x\in X$ gives the condition
		\begin{equation}
			\Big( y_X^o \text{ } \tilde{b} \text{ } y_X \Big)(x) \subseteq a(x)
			\implies \Big( y_X^o \text{ } \tilde{b} \Big) (x^*)= y_x^o
			\Big(\tilde{b}(x^*)\Big) \subseteq a(x)
		\end{equation}
		Thus, if $y^* \in PX$ such that $x^* \tilde{b} y^*$, then
		$y \in y_x^o\Big(\tilde{b}(x^*)\Big)$. Now, for (8) to hold, $y \in a(x)$ i.e. $xay$. Thus,
		need only to show that for any $a\in A, \exists b \in A $ such that $\forall x,y \in X,
		x^* \tilde{b}y^* \implies xay $. So, fix $a\in A$, and take $b \in A: bb \subseteq a$.
		Now, let $x^* \tilde{b}y^*$ i.e. $x^o.A \leq y^o .A .b$.
		Hence, $\exists c \in A: x^oc \subseteq y^o bb$. And as c is reflexive,
		\[ xcx \implies x(cx^o)\star \implies x(bby^o)\star \implies x(bb)y \implies xay \qedhere \]
\end{enumerate}
\end{proof}
%}}}
%{{{ Yoneda Lemma
\begin{theorem}[Yoneda Lemma] %{{{ Yoneda Lemma		Theorem
	For every $\psi \in PX$, in the following digram,
	\begin{enumerate}[label=(\alph*)]
		\item $\psi \geq \psi^*.(y_X)^*$
		\item $\psi \in \overline{y_X(X)} \implies \psi \leq \psi^*.(y_X)_*$
	\end{enumerate}
\end{theorem}
\begin{proof}\setcounter{equation}{0}

	\begin{enumerate}[label=(\alph*)]
		\item By definition, $(y_X)_*=\tilde{A}.y_X$, and $\psi^*=\psi^o.\tilde{A}$. Need that
			$\psi \geq (y_X)_*.\psi^* = \psi^o.\tilde{A}.\tilde{A}.y_X$. And applying Lemma 2.2.1 to
			$\tilde{A}$, the required condition becomes $\psi \geq \psi^o .\tilde{A} .y_X$
			Fix $p \in \psi$,
			we will find $a \in A: p \supseteq \psi^o a y_X$. Examining the right side of the condition,
			(for any $a \in A$, $x \in X$  )
			\begin{equation} \Big(\psi^o.\tilde{a} .y_X \Big) (x) = \psi^o.\tilde{a} (x^*)=
				\psi^o\big(\tilde{a}(x^*)\big)= \begin{cases}
					\phi &\text{ if } \psi \notin \tilde{a} (x^*) \\
					\star &\text{ if } \psi \in \tilde{a} (x^*) \\

				\end{cases}\end{equation}

				In case $\psi \notin \tilde{a} (x^*)$, the condition holds trivially. As $\psi$ is a
				promodule, $\psi.A\leq \psi$ gives $\exists q\in \psi, a
				\in A: qa \subseteq p$. Thus, fix $x\in X$ and	$\psi \in PX$ such that
				$x^* \tilde{a} \psi $. We will now show that $xp\star$. Using the definition of $\tilde{a}$,

				\begin{equation} x^* \tilde{a} \psi \implies x^o.A \leq \psi.a
					\implies \exists b\in A: x^ob \subseteq qa
					\implies \forall z \in X, \big(x^o b \big)(z) \subseteq (qa)(z)
				\end{equation}
				Thus, in particular for $z=x$, as $b$ is reflexive, $xbx$, which gives:
				\begin{equation} \big(x^o b \big)(x) \subseteq (qa)(x) \implies x^ox \subseteq (qa)(x) \implies \star \in (qa)(x) \end{equation}
				But, as $qa \subseteq p$ , (11) gives that $xp\star$.
			\item Suppose $\psi \in \overline{y_X(X)}$, need to show $\psi \leq \psi^*.(y_X)_*=\psi^o.\tilde{A}.y_X$ i.e.
				for $a \in A,$ $\exists p \in \psi: p \subseteq \psi^o.\tilde{a}.y_X$. For any $x \in Dom(p)$,
				the condition requires:
				\begin{equation}
					p(x) \subseteq  \psi^o.\tilde{a} .y_X(x)=\psi^o\big(\tilde{a} (x^*)\big)
				\end{equation}
				By definition of $p$, for (12) to hold, need that $xp\star \implies \psi \in \tilde{a}(x^*) $. Fix
				any $a\in A$, we will find $p \in \psi$ such that (12) holds. By quasi-uniformity of A,
				$\exists b \in A: bb \subseteq a$. From Proposition 2.5(a), $y_X$ is uniformly continuous,
				$y_X.A \leq \tilde{A}.y_X $ giving that $\exists c \in A: y_xc \subseteq \tilde{b}y_X $. Thus, for
				any $z,w\in X$ such that $z c w$,
				\begin{equation} \big(y_X c \big)(z) \subseteq \big(\tilde{b} y_X\big)(z) \implies
					y_X(c(z)) \subseteq \tilde{b}(z^*) \implies w^* \in \tilde{b}(z^*) \text{ i.e. } z^* \tilde{b} w^*
				\end{equation}
				As $A$ is a quasi-uniformity, $\exists d\in A: dd \subseteq c$. Also, because $A$ is a down-
				directed set, $\exists a' \in A: a' \subseteq b,d $. This along with (13) gives that
				for any $x,y \in X$
				\begin{equation} x(a'a')y \implies x(dd)y \implies xcy \implies x^* \tilde{b} y^* \end{equation}
				Now, because $\psi \in \overline{y_X(X)}$, we get
				$\exists x^* \in y_X(X) \text{ such that } \psi \tilde{a'}x^* \text{ and } x^* \tilde{a'} \psi $.
				By definition of $\tilde{a}$, $\psi \tilde{a'}x^*$ gives
				\begin{equation} \psi \leq x^o.A.a' \implies \exists p \in \psi: p \subseteq x^o a'a' \end{equation}
				Fix any $z \in X : zp \star$, using (15) and (14) gives:
				\begin{equation} zp \star \xRightarrow z(x^o a'a') \star \xRightarrow{(15)} z(a'a')x
				\xRightarrow{(14)} z^* \tilde{b}x^*   \end{equation}

				Finally, by definition of the partial order on $\tilde{A}$,$a' \subseteq b \implies \tilde{a'}
				\subseteq \tilde{b} $. Therefore,  $x^* \tilde{a'}\psi \implies x^* \tilde{b}\psi$.
				Now, using (16), $z^* \tilde{b}x^* \text{ and } x^*\tilde{b}\psi  $ gives the desired result
				$z^* \tilde{b} x^* $. \qedhere
		\end{enumerate}
	\end{proof}
	%}}}
	%}}}
	%{{{ Yoneda Corollary
	%\begin{lemma}
	%	Composition of a right-adjoint and an equivalence is a right adjoint.
	%\end{lemma}
	\begin{coro}
		For $\psi \in PX$, $\psi \in \overline{y_X(X)}$ if and only if $\psi$ is a right-adjoint.
	\end{coro}
	\begin{proof}\setcounter{equation}{0}
		Fix any $\psi \in PX$. \setcounter{equation}{0}
		\begin{enumerate}[label=(\roman*)]
			\item ($\implies$)
				Let $\psi \in \overline{y_X(X)}$, from Theorem 4.4, we get that
				$\psi = \psi^*.(y_X^{})_*^{}$. In order to show $\psi$ is a
				right-adjoint, by using Lemma 4.5, it is enough to show that $\psi^*$ is a
				right adjoint and that $(y_X^{})_*^{}$ is an equivalence.
				\begin{enumerate}[label=\Roman*]
					\item In order to show that $(y_X^{})_*^{}$ is an equivalence,
						we need that $A=(y_X^{})^*_{}.(y_X^{})_*$ and
						$\tilde{A}= (y_X^{})_*.(y_X^{})^*_{}$.\\
						From proposition 4.3 (b), we have that $y_X^{}$
						is fully faithful, and by Proposition 3.13 (a), this gives us
						that $A=(y_X^{})^*_{}.(y_X^{})_*$.
						\begin{itemize}
							\item 	We are now going to show that
								$\tilde{A} \leq (y_X^{})_*.(y_X^{})^*$.
								Fix any $a,b\in A$, we need to find $c\in A$
								such that $\tilde{c} \subseteq \tilde{a}\,
								y_X^{} \, y_X^{o} \, \tilde{b} $.
								\[\big(\tilde{a}\, y_X^{}.y_X^{o}\, \tilde{b}\big) (\psi)
									=\big(\tilde{a}\,\tilde{b}\big) (\psi)
									\supseteq \tilde{c} \tilde{c} (\psi)
								\supseteq \tilde{c}(\psi) \]
								In the above equation, the equality holds because $\psi \in \overline{y_X^{}(X)}$, gives
								the existence of $x^*=\tilde{b}(\psi)$. And the first inequality is given by down-directedness of
								$\tilde{A}$, whereas the second one holds because $\tilde{c}$ is reflexive, as $\tilde{A}$ is a
								quasi-uniformity.
							\item To show that $\tilde{A} \geq (y_X^{})_*.(y_X^{})^*$, fix any $a\in A$. By
								quasi-uniformity of $\tilde{A}$, there exists $\tilde{b}\in \tilde{A}$ such that
								$\tilde{b} \, \tilde{b} \subseteq a$. We will show that $\tilde{a} \supseteq
								\tilde{b}\,y_X^{} \, y_X^{o} \, \tilde{b} $.
								\[ \psi \big(\tilde{b}\,y_X^{} \, y_X^{o} \, \tilde{b} \big) \phi
								\implies \psi \big( \tilde{b} \tilde{b} \big) \phi \implies \psi \tilde{a} \phi\]


						\end{itemize}

					\item In order to show that
						$\psi^*$ is a right adjoint to $\psi_*$,
						we need to show that
						$\tilde{A} \geq \psi_\star.\psi^\star $ and
						$\psi_\star.\psi^\star \geq 1$.
						\begin{itemize}
							\item To show that $\tilde{A}\geq \psi_*.\psi^*=\psi_*.\psi^o.\tilde{A} $,
								fix any $a\in A$. We will show that $\psi_*.\psi^o.\tilde{a} \subseteq \tilde{a}$.
								Using definition of $\psi_*$, for any $\phi \in \overline{y_X^{}(X)}$, we get that
								\[\big(\psi_*.\psi^o.\tilde{a}\big)(\phi)=
									\psi_*.\psi^o(\tilde{a}(\phi))= \begin{cases}
										\phi &\text{ if }\tilde{a} (\phi)\neq \psi \\
										\psi=\psi_*.\psi^o(\psi) &\text{ if } \tilde{a} (\phi)=\psi
									\end{cases}\]
									The above equation gives that $\phi\big(\psi_*.\psi^o.\tilde{a}\big)\psi$ implies
									$\phi \tilde{a} \psi$.
								\item We will show that $\psi_\star.\psi^\star \geq 1$, that is
									$\star(\psi^o.\tilde{a} .\psi_*)\star$. Using definition of $\psi_*$,
									\[\big(\psi^o.\tilde{a} .\psi_*\big)(\star)=\big(\psi^o.\tilde{a}\big) (\psi_*(\star))
									= \big( \psi^o.\tilde{a} \big) (\psi)=\psi^o \big( \tilde{a}(\psi)  \big) \]
									By the quasi-uniformity of $\tilde{A}$, we get that $\tilde{a}$ is reflexive, and hence,
									$\psi \tilde{a} \psi$. So, from the above equation, we have that
									$\star \in \psi^o(\psi) \subseteq \big(\psi^o.\tilde{a} .\psi_*\big)(\star) $.
							\end{itemize}
					\end{enumerate}

				\item ($\impliedby$) Suppose $\psi$ is a right adjoint. Need to show that for any
					$a \in A$, $\exists x^* \in y_X(X)$ such that $\psi \, \tilde{a} \,x^*
					\tilde{a} \, \psi$. Fix $a \in A$. Because $\psi$ is a right-adjoint, there
					exists a promodule $\phi: 1 \carrow X$ such that $\phi.\psi \leq A$ and
					$1\leq \psi.\phi$. From $\phi.\psi \leq A$, we get that
					\begin{equation}\exists p \in \phi, q \in \psi \text{ such that }
						a \supseteq p.q
					\end{equation}
					Because $\phi$ and $\psi$ are promodules,
					\begin{align}
						A.\phi \leq \phi & \text{ gives the existence of } p' \in \phi \text{ such that } p\supseteq a'p'  \\
						A.\psi \leq \psi & \text{ gives the existence of } q' \in \psi \text{ and } a'' \in A
						\text{ such that } q \supseteq a''q'
					\end{align}
					Now, from $1 \leq \psi.\phi$, we get that $q'\,p'$ is reflexive i.e. $\star (q' \,p')
					\star$. By the definition of composition we get the existence of an $x \in X$
					such that $\star \, p' \, x \,q'\,\star$. Now, considering $x$ as a map, $x:1 \to X$
					defined as $\star \mapsto x$,
					\begin{align}
						x \, q' \, \star \text{ i.e. } &\star \in q'(x) \text{ gives that }
						q' \supseteq x^o \\
						\star \,p' \, x \text{ i.e. } &x \in p'(\star) \text{ gives that }
						p'\supseteq x
					\end{align}
					Thus, by using inequalities (1),(2) and (3), we get that
					\begin{equation}
						a \supseteq p\,q \supseteq a'\,p'\,q'\,a''
					\end{equation}
					By definition of $\tilde{a}$, to show $\psi \, \tilde{a} \, x^*$,
					we need that $\psi \leq x^*\,a=x^o.\,A.\,a$. Showing
					$\text{ for any } b\in A$, $\;  x^o\,b\,a \supseteq q'$:
					\[ x^o\,b\,a \supseteq x^o\,b\,a'\,p'\,q' \supseteq x^o\,b\,a'\,x\,q'
					\supseteq x^o\,x\,q' \subseteq q'\]
					Where the first inequality comes from (6) by using reflexiveness of $a''$ and then
					left-multiplying by $x^o$. The second inequality comes from (5),
					third one from reflexiveness of $b$ and $a'$,
					and the last one is given by Lemma 2.4.1.

					In order to show $x^* \, \tilde{a} \, \psi$, by definition of $\tilde{a}$,
					need that $x^o.A=x^* \leq \psi\,a$. Fix $k\in \psi$. We will show
					$k\,a \supseteq x^o \, a''$.

					\begin{equation}
						a\supseteq a'\,p'\,q'\,a'' \supseteq p'\,q'\,a''
						\supseteq p' \,x^o \,a''
					\end{equation}

					Where the first inequality is given by (6), second one
					is due to reflexiveness	of $a'$ and the third inequality comes by using (4).
					Left-multiplying (7) with $k$ gives the following.
					\begin{equation}
						ka \supseteq k\,p'\,x^o\,a'' \text{ that is, for any } z\in X,
						\; \; z (k\, a) \star \implies z ( k\,p'\,x^o\,a'') \star
					\end{equation}
					As $\psi$ is a right adjoint to $\phi$, we have $1\leq \psi.\phi$, giving that
					$\star (k \, p' )\star $. So, using the implication in(8), we get that
					$z(k\, a) \star $ implies $z ( x^o\,a'') \star (k\,p') \star$, which in turn
					gives that $z(x^o \, a'')\star$. Hence, we get that $ka \supseteq x^o\, a''$
					\qedhere
			\end{enumerate}
		\end{proof}
		%}}}
\newpage
\nocite{Leinster_2014}
		\printbibliography
		\end{document}
