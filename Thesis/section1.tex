\documentclass[18pt,a4paper]{article}
\usepackage[margin=0.7in]{geometry}
\usepackage{amsmath,amsthm,amssymb,tikz-cd}
\usepackage{enumitem,amsfonts,extarrows,xcolor}

\newtheorem{theorem}{Theorem}[section]

\theoremstyle{definition}
\newtheorem{definition}{Definiton}[section]
\newtheorem{lemma}{Lemma}[definition]
\newtheorem{prop}{Proposition}[definition]
\newtheorem{proop}{Proposition}[section]

\tikzset{commutative diagrams/.cd,mysymbol/.style={start anchor=center,end anchor=center,draw=none}}
\newcommand\cen[2][\leq]{\arrow[mysymbol]{#2}[description]{#1}}

\makeatletter
\newcommand{\carrow}{}% just in case
\DeclareRobustCommand{\carrow}{%
  \mathrel{\vphantom{\rightarrow}\mathpalette\circle@arrow\relax}%
}
\newcommand{\circle@arrow}[2]{%
  \m@th
  \ooalign{%
    \hidewidth$#1\circ\mkern1mu$\hidewidth\cr
    $#1\longrightarrow$\cr}%
}

\begin{document}
\section{Definitions}%
\label{sec:definitions}

\begin{definition}[Prorelation] %{{{ Prorelation		Definition
A partial-ordered set of relations $X \to Y$, which is down-directed and an upper set. i.e
A set, $P \subseteq \mathcal{P}(X \times Y)$ such that
	\begin{enumerate}[label=(\roman*)]
		\item A partial-order defined to be containment as relations, $r \subseteq s$
			only if $\forall (x,y) \in X \times Y$, $(x,y) \in r \implies (x,y)\in s$
		\item (Down-directed), $\forall r,s \in P, \exists t \in P $ such that
			$t\subseteq r \text{ and } t \subseteq s$
		\item (Up-set) for any relation $u:X\to Y$, if $\exists p \in P \text{ such that } p\leq u $
			then $u \in P$
\end{enumerate} \end{definition} %}}}
\begin{definition}[Composition of prorelations] %{{{ Composition of prorelations		Definition
	Prorelations can be composed by taking all compositions of their elements as relations:
	for prorelations $P:X\to Y \text{ and } Q:Y\to Z$,
	\[ Q.P:=\{q \circ p : p \in P \text{ and } q \in Q  \} \]
\end{definition}
\begin{definition}[Comparison of Prorelations] %{{{ Comparison of Prorelations		Definition
Two prorelations with same domain, co-domain are comparable as
\[ \text{ for }  P,Q:X \to Y \text{ , } P \leq Q \text{ if } \forall q \in Q, \exists p \in P
\text{ such that } p \subseteq q    \]
\end{definition}
\begin{definition}[Quasi-uniformity] %{{{ Quasi-uniformity		Definition
A prorelation on a set X, $P:X \to X$ is a quasi-uniformity if it follows :
\begin{enumerate}[label=\roman*]
	\item $\forall p \in P $ , for any $x \in X$, $(x,x) \in p$ i.e. $xpx$
	\item $\forall p \in P, \exists p' \in P \text{ such that } p' \circ p' \subseteq p$
\end{enumerate}
And in this case, $(X,A)$ is called a \textit{quasi-uniform space}.
\end{definition}
\begin{definition}[Uniformly Continuous function ] %{{{ Uniformly Continuous function 		Definition
	A function, $f:X \to Y$ is called a uniformly continuous function, $f:(X,A) \to (Y,B)$
	if, $\forall b \in B, \exists a \in A \text{ such that } f \circ a \subseteq b \circ f $.
	meaning that $f.A \leq B.f$ or
	\begin{tikzcd}
		X \cen{dr} \arrow[swap]{d}{A} \arrow{r}{f}
		& Y \arrow{d}{B}\\
	X \arrow[swap]{r}{f}
		& Y
	\end{tikzcd}.

\end{definition}
\begin{definition}[Promodule] %{{{ Promodule		Definition
	A prorelation, $\phi:X \carrow Y$ is called a promodule $\phi: (X,A) \carrow (Y,B)$  if it obeys:
	$ \phi.A \leq \phi \text{ and } B. \phi \leq \phi $ where . denotes composition as prorelations.
\end{definition}
\begin{definition}[Comparison of Promodules] %{{{ Comparison of Promodules		Definition
	Promodules with same domain and co-domain are compared as prorelations, for
	$\phi,\psi:(X,A)\carrow (Y,B)$, $\phi \sqsubseteq \psi$ , only if $\phi \leq \psi$.
\end{definition}
\begin{definition}[Composition of Promodules] %{{{ Composition of Promodules		Definition
	Promodules are composed as prorelations.\\
	For promodules $\phi: (X,A) \carrow (Y,B)$ and $\psi:(Y,B)\carrow(Z,C)$,
	$\psi \phi := \psi.\phi = \{q \circ p :p\in \phi \text{ and } q \in \psi \}$
\end{definition}
\begin{definition}[Opposite relation] %{{{ Opposite relation		Definition
For relation $r:X \to Y$, $r^o$ is defined to be a relation $r^o:Y \to X$ as
\[ \forall (x,y)\in X \times Y, (x,y)\in r \iff (y,x) \in r^o \]
\end{definition}
\begin{lemma}
	For any relation $r:X \to Y$, $r^o \circ r = \Delta_X$
\end{lemma}
\begin{lemma}
	For any relation $r:X \to Y$, $r \circ r^o \subseteq \Delta_Y$
\end{lemma}
\begin{definition}[$(-)_*$] %{{{ (-)_*		Definition

\end{definition}
\begin{definition}[$(-)^*$] %{{{ (-)^*		Definition

\end{definition}
\begin{definition}[Fully Faithful] %{{{ Fully Faithful		Definition

\end{definition}

\begin{definition}[Fully Dense] %{{{ Fully Faithful		Definition

\end{definition}
\begin{definition}[Topologically Dense] %{{{ Topologically Dense		Definition

\end{definition}
\section{Propositions}
\begin{definition}[QUnif] %{{{ QUnif		Definition
QUnif is defined to be the category having quasi-uniform spaces as objects, and uniformly continous
maps between them as morphisms.
\end{definition}
\begin{lemma} QUnif does define a category, as
	\begin{enumerate}[label=\roman*]
		\item Composition
		\item Identity
	\end{enumerate}
\end{lemma}

\begin{definition}[ProMod] %{{{ ProMod		Definition

\end{definition}

\begin{lemma} ProMod does define a category, as
	\begin{enumerate}[label=\roman*]
		\item Composition
		\item Identity
	\end{enumerate}
\end{lemma}

%Mention about topology so that below definition makes more sense ??%


\begin{proop}[$(-)_*:$QUnif$\to ProMod$ is a Functor ]

\end{proop}

\begin{proop}[$(-)^*:$QUnif$^{op} \to ProMod$ is a Functor ]
	Defined as fixing objects and taking morphisms to their image under $(-)^*$
		\begin{enumerate}[label=(\alph*)]
			\item for $(X,A) \in \text{QUnif}^{op}$, $(X,A)^*:=(X,A) \in \text{ProMod}$
			\item for $f:(X,A) \to (Y,B)$ in QUnif,
				$f^* := f^o .B$
		\end{enumerate}
\end{proop}
\begin{proof}
	\item \paragraph{Showing that $f^o .B: (Y,B) \carrow (X,A)$ is a promodule } \mbox{} \\
	So, need to show $f^o .B$ a prorelation $Y \to X$
	and that $(f^o .B).B \sqsubseteq f^o .B$ and $A.(f^o .B) \sqsubseteq f^o .B$ \\
	To show prorelation, \begin{enumerate}[label=(\roman*)]
		\item (Partial-order) Inclusion of relations i.e. for $k=f^o \circ b$ and
			$k'=f^o \circ b'$ in $f^o .B$ , $k \subseteq k' \iff b \subseteq b'$
		\item (Down directed) for $k,k' \in f^o .B$, need that $\exists l \in f^o .B
		\text{ such that } l \subseteq k,k'$

		Fix $k,k' \in f^o .B \implies \exists b,b' \in B : k=f^o \circ b \text{ and }
		k' = f^o \circ b'$

		And as $B$ is a quasi-uniformity, it's down directed so,
		$\exists c \in B: c \subseteq b,b' \implies l:= f^o \circ c \subseteq k,k'$
	\item (Up-set) for a relation $l:Y \to X$ and $k \in f^o .B$ such that $l \supseteq k$
		 , need $l \in f^o .B$

		 Let $b\in B$ be such that $k=f^o \circ b$ and define
		 $b':=\{(y,y'): y \in domain(l) \text{ and } y' \in (f^o)^{-1}(l(y))$\\
			 As $l\supseteq k=f^o \circ b$, $domain(b')=domain(l)\supseteq domain(b)$
			 \\ and $range(l) \supseteq range(f^o \circ b)\implies
			 \forall y \in domain(b), range (b')=(f^o )^{-1}(l(y)) \supseteq (f^o)^{-1}(f^o \circ b ) = range(b)$\\
			 Now, by definition of $b'$, $f^o \circ b' \supseteq l$. To show
			 $f^o \circ b \subseteq l$ , \\
			 $(x,y)\in f^o \circ b' \implies \exists z \in Y: (x,z)\in b' \text{ and }
			 (z,y) \in f^o \implies x \in domain(l) \text{ and } z \in l(x)$ i.e.
			 $(x,z) \in l$

	\end{enumerate}
\item
	To show $(f^o .B).B \leq f^o .B$, need that $\forall b \in B,
	\exists b' \in B : f^o \circ b' \circ b' \subseteq f^o \circ b$,\\
	Fix any $b \in B$ as B is a quasi-uniformity, $\exists b' \in B : b' \circ b' \subseteq b
	\implies f^o \circ b'\circ b' \subseteq f^o \circ b$

	To show $A.(f^o .B) \leq f^o .B$, need that $\forall b \in B,
	\exists b' \in B, a\in A : a \circ f^o \circ b' \subseteq f^o \circ b$,\\
	As $f$ is uniformly continuous, $f.A\leq B.f$ i.e. $\forall b \in B, \exists a \in A
	: f \circ a \subseteq b \circ f
	\implies a= f^o \circ f \circ a \subseteq f^o \circ  b \circ f $   \\
	Fix any $b \in B, \text{ so, } \exists b' \in B : b'b' \subseteq b$
	(for brevity,omitting $\circ$ to explicitly denote composition  )\\
	And, for this $b', \exists a : a \subseteq f^ob'f \implies af^ob' \subseteq f^ob'ff^ob'
	\subseteq f^o b'b' \subseteq f^o b \implies af^ob' \subseteq f^o b$\\
\item
	Now, need to show that $(-)^*$ respects composition and identity.
	\begin{enumerate}[label=(\roman*)]
		\item (Composition) let $f,g$ be uniformly continuous,
			$(X,A) \xrightarrow{f} (Y,B) \xrightarrow{g} (Z,C)$
			need that $(g \circ f)^*= f^*.g^* $

			LHS=$(g \circ f)^*=(g \circ f)^o .C=(f^o \circ g^o).C$ and
			RHS=$f^*.g^* =(f^o .B).(g^o .C)$\\
			For equality, showing that LHS$\geq$RHS and LHS$\leq$RHS:

			To show $(f^o \circ g^o).C\geq(f^o .B).(g^o .C)$, need that
			$\forall c \in C, \exists b \in B, c' \in C : f^og^oc
			\supseteq f^obgc'$ \\
			Fix any $c \in C, \text{ so, } \exists c' \in C: c' \circ c' \subseteq c
			\implies f^o g^o c \supseteq f^o g^o (c'c')
			=f^o g^o (c' \Delta_Z c') \supseteq f^o g^o c'(gg^o)c'$ \\
			By uniform conntinuity of g, for $c'\in C,\exists b\in B: gb\subseteq c'g $
			\\Thus, $f^o g^o c \supseteq f^o g^o (c'g)g^oc' \supseteq
			f^o (g^o g)bg^o c'=f^o bg^o c'$.

			To show $(f^o \circ g^o).C\leq(f^o .B).(g^o .C)$, need that
			$\forall b \in B, c \in C, \exists c' \in C: f^o g^o c \subseteq f^obg^oc $
			\\Fix any $c\in C, b\in B$ will show that $c':=c$ works:\\
			As B is a quasi-uniformity, $\Delta_Y \subseteq b\implies f^o \Delta_Y
			g^o c=f^o g^o c \subseteq f^o b	g^o c=f^o b g^o c'$
		\item(Identity) let $(X,A)\in \text{ QUnif }^{op} $, and
			$1_{(X,A)}:(X,A)\to(X,A)$ as $x\mapsto x$ need that
			$(1_{(X,A)})^*=1_{(X,A)^*}$
			LHS=$(1_{(X,A)})^*=(1_{(X,A)})^o.A=1_{(X,A)}.A=A$.
			\\Now, it's required that $A$ is the identity of $(X,A)$ in ProMod.\\
			So, fix $\phi:(X,A) \carrow (Y,B)$, need to show $\phi.A=\phi$\\
			As $\phi$ is a promodule, $\phi.A \leq \phi$ and as A is quasi-uniformity on X,\\
			$\forall a\in A, \Delta_X \subseteq a \implies
			\forall a \in A, \forall p \in \phi, p=p\Delta_X\subseteq pa \implies
			\phi \leq \phi.A$\\
			Also, fix $\psi:(Y,B) \carrow (X,A)$, need to show $A.\psi=\psi$\\
			As $\psi$ is a promodule, $A.\psi \leq \psi$ and as A is quasi-uniformity on X,\\
			$\forall a\in A, \Delta_X \subseteq a \implies
			\forall a \in A, \forall q \in \psi, q=\Delta_X q \subseteq aq \implies
			\psi \leq \psi.A$
	\end{enumerate}



\end{proof}
\newpage
%Talk about adjointness ??

\begin{proop}[Proposition 1]
	Fix a uniformly continuous map, $f:(X,A) \rightarrow (Y,B)$
	\begin{enumerate}[label=(\alph*)]
		\item f is fully faithful $\iff A=f^o.B.f$
		\item f is fully dense $\iff \forall b\in B, \exists b' \in B \text{ such that }  $
	\end{enumerate}
\end{proop}






\newpage
asd
\end{document}
