\documentclass[18pt,a4paper]{article}
\usepackage[margin=0.7in]{geometry}
\usepackage{amsmath,amsthm,amssymb,tikz-cd}
\usepackage{enumitem,amsfonts,extarrows,xcolor}

\newtheorem{theorem}{Theorem}[section]

\theoremstyle{definition}
\newtheorem{definition}{Definiton}[section]
\newtheorem{lemma}{Lemma}[definition]
\newtheorem{prop}{Proposition}[definition]
\newtheorem{proop}{Proposition}[section]

\tikzset{commutative diagrams/.cd,mysymbol/.style={start anchor=center,end anchor=center,draw=none}}
\newcommand\cen[2][\leq]{\arrow[mysymbol]{#2}[description]{#1}}

\makeatletter
\newcommand{\carrow}{}% just in case
\DeclareRobustCommand{\carrow}{%
	\mathrel{\vphantom{\rightarrow}\mathpalette\circle@arrow\relax}%
}
\newcommand{\circle@arrow}[2]{%
	\m@th
	\ooalign{%
		\hidewidth$#1\circ\mkern1mu$\hidewidth\cr
	$#1\longrightarrow$\cr}%
}

\begin{document}
\section{Definitions}%
\label{sec:definitions}

\begin{definition}[Prorelation] %{{{ Prorelation		Definition
	A partial-ordered set of relations $X \to Y$, which is down-directed and an upper set. i.e
	A set, $P \subseteq \mathcal{P}(X \times Y)$ such that
	\begin{enumerate}[label=(\roman*)]
		\item A partial-order defined to be containment as relations, $r \subseteq s$
			only if $\forall (x,y) \in X \times Y$, $(x,y) \in r \implies (x,y)\in s$
		\item (Down-directed), $\forall r,s \in P, \exists t \in P $ such that
			$t\subseteq r \text{ and } t \subseteq s$
		\item (Up-set) for any relation $u:X\to Y$, if $\exists p \in P \text{ such that } p\leq u $
			then $u \in P$
	\end{enumerate} \end{definition} %}}}
	\begin{definition}[Composition of prorelations] %{{{ Composition of prorelations		Definition
		Prorelations can be composed by taking all compositions of their elements as relations:
		for prorelations $P:X\to Y \text{ and } Q:Y\to Z$,
		\[ Q.P:=\{q \circ p : p \in P \text{ and } q \in Q  \} \]
	\end{definition}
	\begin{definition}[Comparison of Prorelations] %{{{ Comparison of Prorelations		Definition
		Two prorelations with same domain, co-domain are comparable as
		\[ \text{ for }  P,Q:X \to Y \text{ , } P \leq Q \text{ if } \forall q \in Q, \exists p \in P
		\text{ such that } p \subseteq q    \]
	\end{definition}
	\begin{definition}[Quasi-uniformity] %{{{ Quasi-uniformity		Definition
		A prorelation on a set X, $P:X \to X$ is a quasi-uniformity if it follows :
		\begin{enumerate}[label=\roman*]
			\item $\forall p \in P $ , for any $x \in X$, $(x,x) \in p$ i.e. $xpx$
			\item $\forall p \in P, \exists p' \in P \text{ such that } p' \circ p' \subseteq p$
		\end{enumerate}
		And in this case, $(X,A)$ is called a \textit{quasi-uniform space}.
	\end{definition}
	\begin{definition}[Uniformly Continuous function ] %{{{ Uniformly Continuous function 		Definition
		A function, $f:X \to Y$ is called a uniformly continuous function, $f:(X,A) \to (Y,B)$
		if, $\forall b \in B, \exists a \in A \text{ such that } f \circ a \subseteq b \circ f $.
		meaning that $f.A \leq B.f$ or
		\begin{tikzcd}
			X \cen{dr} \arrow[swap]{d}{A} \arrow{r}{f}
		& Y \arrow{d}{B}\\
		X \arrow[swap]{r}{f}
		& Y
		\end{tikzcd}.

	\end{definition}
	\begin{definition}[Promodule] %{{{ Promodule		Definition
		A prorelation, $\phi:X \carrow Y$ is called a promodule $\phi: (X,A) \carrow (Y,B)$  if it obeys:
		$ \phi.A \leq \phi \text{ and } B. \phi \leq \phi $ where . denotes composition as prorelations.
	\end{definition}
	\begin{definition}[Comparison of Promodules] %{{{ Comparison of Promodules		Definition
		Promodules with same domain and co-domain are compared as prorelations, for
		$\phi,\psi:(X,A)\carrow (Y,B)$, $\phi \sqsubseteq \psi$ , only if $\phi \leq \psi$.
	\end{definition}
	\begin{definition}[Composition of Promodules] %{{{ Composition of Promodules		Definition
		Promodules are composed as prorelations.\\
		For promodules $\phi: (X,A) \carrow (Y,B)$ and $\psi:(Y,B)\carrow(Z,C)$,
		$\psi \phi := \psi.\phi = \{q \circ p :p\in \phi \text{ and } q \in \psi \}$
	\end{definition}
	\begin{definition}[Opposite relation] %{{{ Opposite relation		Definition
		For relation $r:X \to Y$, $r^o$ is defined to be a relation $r^o:Y \to X$ as
		\[ \forall (x,y)\in X \times Y, (x,y)\in r \iff (y,x) \in r^o \]
	\end{definition}
	\begin{lemma}
		For any relation $r:X \to Y$, $r^o \circ r = \Delta_X$
	\end{lemma}
	\begin{lemma}
		For any relation $r:X \to Y$, $r \circ r^o \subseteq \Delta_Y$
	\end{lemma}
	\begin{lemma} For relations $r,s:X\to Y$ and $t:Y\to Z$,
		for any $x,x' \in X$, $r(x)\subseteq s(x') \implies (t \circ r)(x) \subseteq
		(t \circ s)(x)$

	\end{lemma}
	\begin{lemma} For relations $r:X\to Y$ and $s,t:Y\to Z$,
		$s\subseteq t \implies (s \circ r) \subseteq
		(t \circ r)(x)$

	\end{lemma}
	\begin{definition}[$(-)_*$] %{{{ (-)_*		Definition

	\end{definition}
	\begin{definition}[$(-)^*$] %{{{ (-)^*		Definition

	\end{definition}
	\begin{definition}[Fully Faithful] %{{{ Fully Faithful		Definition

	\end{definition}

	\begin{definition}[Fully Dense] %{{{ Fully Faithful		Definition

	\end{definition}
	\begin{definition}[Topologically Dense] %{{{ Topologically Dense		Definition

	\end{definition}
	\section{Propositions}
	\begin{definition}[QUnif] %{{{ QUnif		Definition
		QUnif is defined to be the category having quasi-uniform spaces as objects, and uniformly continous
		maps between them as morphisms.
	\end{definition}
	\begin{lemma} QUnif does define a category, as
		\begin{enumerate}[label=\roman*]
			\item Composition
			\item Identity
		\end{enumerate}
	\end{lemma}

	\begin{definition}[ProMod] %{{{ ProMod		Definition

	\end{definition}

	\begin{lemma} ProMod does define a category, as
		\begin{enumerate}[label=\roman*]
			\item Composition
			\item Identity
		\end{enumerate}
	\end{lemma}

	%Mention about topology so that below definition makes more sense ??%

	%{{{Two functors
	\begin{proop}[$(-)_*:$QUnif$\to ProMod$ is a Functor ]

	\end{proop}
	\begin{proof}

	\end{proof}

	\begin{proop}[$(-)^*:$QUnif$^{op} \to ProMod$ is a Functor ]
		Defined as fixing objects and taking morphisms to their image under $(-)^*$
		\begin{enumerate}[label=(\alph*)]
			\item for $(X,A) \in \text{QUnif}^{op}$, $(X,A)^*:=(X,A) \in \text{ProMod}$
			\item for $f:(X,A) \to (Y,B)$ in QUnif,
				$f^* := f^o .B$
		\end{enumerate}
	\end{proop}
	\begin{proof}
	\item \paragraph{Showing that $f^o .B: (Y,B) \carrow (X,A)$ is a promodule } \mbox{} \\
		So, need to show $f^o .B$ a prorelation $Y \to X$
		and that $(f^o .B).B \sqsubseteq f^o .B$ and $A.(f^o .B) \sqsubseteq f^o .B$ \\
		To show prorelation, \begin{enumerate}[label=(\roman*)]
			\item (Partial-order) Inclusion of relations i.e. for $k=f^o \circ b$ and
				$k'=f^o \circ b'$ in $f^o .B$ , $k \subseteq k' \iff b \subseteq b'$
			\item (Down directed) for $k,k' \in f^o .B$, need that $\exists l \in f^o .B
				\text{ such that } l \subseteq k,k'$

				Fix $k,k' \in f^o .B \implies \exists b,b' \in B : k=f^o \circ b \text{ and }
				k' = f^o \circ b'$

				And as $B$ is a quasi-uniformity, it's down directed so,
				$\exists c \in B: c \subseteq b,b' \implies l:= f^o \circ c \subseteq k,k'$
			\item (Up-set) for a relation $l:Y \to X$ and $k \in f^o .B$ such that $l \supseteq k$
				, need $l \in f^o .B$

				Let $b\in B$ be such that $k=f^o \circ b$ and define
				$b':=\{(y,y'): y \in domain(l) \text{ and } y' \in (f^o)^{-1}(l(y))$\\
					As $l\supseteq k=f^o \circ b$, $domain(b')=domain(l)\supseteq domain(b)$
					\\ and $range(l) \supseteq range(f^o \circ b)\implies
					\forall y \in domain(b), range (b')=(f^o )^{-1}(l(y)) \supseteq (f^o)^{-1}(f^o \circ b ) = range(b)$\\
					Now, by definition of $b'$, $f^o \circ b' \supseteq l$. To show
					$f^o \circ b \subseteq l$ , \\
					$(x,y)\in f^o \circ b' \implies \exists z \in Y: (x,z)\in b' \text{ and }
					(z,y) \in f^o \implies x \in domain(l) \text{ and } z \in l(x)$ i.e.
					$(x,z) \in l$

			\end{enumerate}
		\item	To show $(f^o .B).B \leq f^o .B$, need that $\forall b \in B,
			\exists b' \in B : f^o \circ b' \circ b' \subseteq f^o \circ b$,\\
			Fix any $b \in B$ as B is a quasi-uniformity, $\exists b' \in B : b' \circ b' \subseteq b
			\implies f^o \circ b'\circ b' \subseteq f^o \circ b$

			To show $A.(f^o .B) \leq f^o .B$, need that $\forall b \in B$,
			$\exists b' \in B, a\in A : a \circ f^o \circ b' \subseteq f^o \circ b$,\\
			As $f$ is uniformly continuous, $f.A\leq B.f$ i.e. $\forall b \in B, \exists a \in A
			: f \circ a \subseteq b \circ f
			\implies a= f^o \circ f \circ a \subseteq f^o \circ  b \circ f $   \\
			Fix any $b \in B, \text{ so, } \exists b' \in B : b'b' \subseteq b$
			(for brevity, omitting $\circ$ to explicitly denote composition  )\\
			And, for this $b', \exists a : a \subseteq f^ob'f \implies af^ob' \subseteq f^ob'ff^ob'
			\subseteq f^o b'b' \subseteq f^o b \implies af^ob' \subseteq f^o b$\\
		\item	Now, need to show that $(-)^*$ respects composition and identity.
			\begin{enumerate}[label=(\roman*)]
				\item (Composition) let $f,g$ be uniformly continuous,
					$(X,A) \xrightarrow{f} (Y,B) \xrightarrow{g} (Z,C)$
					need that $(g \circ f)^*= f^*.g^* $

					LHS=$(g \circ f)^*=(g \circ f)^o .C=(f^o \circ g^o).C$ and
					RHS=$f^*.g^* =(f^o .B).(g^o .C)$\\
					For equality, showing that LHS$\geq$RHS and LHS$\leq$RHS:

					To show $(f^o \circ g^o).C\geq(f^o .B).(g^o .C)$, need that
					$\forall c \in C, \exists b \in B, c' \in C : f^og^oc
					\supseteq f^obgc'$ \\
					Fix any $c \in C, \text{ so, } \exists c' \in C: c' \circ c' \subseteq c
					\implies f^o g^o c \supseteq f^o g^o (c'c')
					=f^o g^o (c' \Delta_Z c') \supseteq f^o g^o c'(gg^o)c'$ \\
					By uniform conntinuity of g, for $c'\in C,\exists b\in B: gb\subseteq c'g $
					\\Thus, $f^o g^o c \supseteq f^o g^o (c'g)g^oc' \supseteq
					f^o (g^o g)bg^o c'=f^o bg^o c'$.

					To show $(f^o \circ g^o).C\leq(f^o .B).(g^o .C)$, need that
					$\forall b \in B, c \in C, \exists c' \in C: f^o g^o c \subseteq f^obg^oc $
					\\Fix any $c\in C, b\in B$ will show that $c':=c$ works:\\
					As B is a quasi-uniformity, $\Delta_Y \subseteq b\implies f^o \Delta_Y
					g^o c=f^o g^o c \subseteq f^o b	g^o c=f^o b g^o c'$
				\item(Identity) let $(X,A)\in \text{ QUnif }^{op} $, and
					$1_{(X,A)}:(X,A)\to(X,A)$ as $x\mapsto x$ need that
					$(1_{(X,A)})^*=1_{(X,A)^*}$
					LHS=$(1_{(X,A)})^*=(1_{(X,A)})^o.A=1_{(X,A)}.A=A$.
					\\Now, it's required that $A$ is the identity of $(X,A)$ in ProMod.\\
					So, fix $\phi:(X,A) \carrow (Y,B)$, need to show $\phi.A=\phi$\\
					As $\phi$ is a promodule, $\phi.A \leq \phi$ and as A is quasi-uniformity on X,\\
					$\forall a\in A, \Delta_X \subseteq a \implies
					\forall a \in A, \forall p \in \phi, p=p\Delta_X\subseteq pa \implies
					\phi \leq \phi.A$\\
					Also, fix $\psi:(Y,B) \carrow (X,A)$, need to show $A.\psi=\psi$\\
					As $\psi$ is a promodule, $A.\psi \leq \psi$ and as A is quasi-uniformity on X,\\
					$\forall a\in A, \Delta_X \subseteq a \implies
					\forall a \in A, \forall q \in \psi, q=\Delta_X q \subseteq aq \implies
					\psi \leq \psi.A$
			\end{enumerate}



		\end{proof}
		%}}}

		%Talk about adjointness ??


		%{{{ Proposition 1
		\begin{proop}[Proposition 1]
			Fix a uniformly continuous map, $f:(X,A) \rightarrow (Y,B)$
			\begin{enumerate}[label=(\alph*)]
				\item f is fully faithful $\iff A\geq f^o.B.f$
				\item f is fully dense $\iff \forall b\in B, \exists b' \in B \text{ such that }
					b' \subseteq bff^o b$
				\item f is topologically dense $\iff\forall b\in B,\Delta_Y\subseteq b\circ f\circ f^o\circ b$
				\item f is fully dense $\iff$ f is topologically dense
			\end{enumerate}
		\end{proop}
		\begin{proof}
		\item
			\begin{enumerate}[label=(\alph*)]
				\item \begin{enumerate}[label=(\roman*)]
					\item $ (\implies) $ Let f be fully faithful i.e. $f^*.f_*=A
						\implies f^o .B.B.f=A $\\
						Need to show that $A= f^o .B.f$ i.e.
						$A\leq f^o .B.f$ and  $A\geq f^o .B.f$\\
						By hypothesis and quasi-uniformity of B,
						$A\geq f^o .B.B.f \geq f^o B.f $\\
						To show $A \leq f^o .B.f$, need that $\forall b \in B, \exists a
						\in A : a \subseteq f^o bf$\\
						Fix $b\in B$, hypothesis gives that $f^o .B.B.f \leq A$ so, \\
						$\exists a \in A: a \subseteq f^o bbf$ and also, by
						quasi-uniformity of B, for $b, \exists b' \in B : b'b' \subseteq b
						\implies f^o b'b'f \subseteq f^o bf$\\
						Combining the above two inequalities,
						$a \subseteq f^o bbf \subseteq f^o bf$\\
					\item $(\impliedby)$ Let $A=f^o .B.f$ need to show $A=f^o.B.B.f$ i.e.
						$A\geq f^o B.B.f \text{ and } A\leq f^o B.B.f$\\
						To show $A\geq f^o .B.B.f$, need to show that $\forall a\in A,
						\exists b,b' \in B : a \supseteq f^o bb'f$\\
						Have that $A\geq f^o .B.f$ and $B.B \leq B$\\
						So, fix $a \in A$, now $\exists b\in B: a \subseteq f^o bf$
						and for this b, $\exists b'\in B: b'b' \subseteq b$.
						Therefore, $a \supseteq f^o bf \supseteq f^o b'b' f$
						To show $A\leq f^o .B.B.f$, need $\forall b,b'\in B, \exists
						a\in A : a \subseteq f^o bb'f$ \\
						So, fix $b,b' \in B $, now, by hypothesis,

						NOPE, USE UNIFORM CTY OF f HERE
						$A\leq f^o .B.f $
						giving\\ $\exists a \in A : a \subseteq f^o bf$ and
						$\exists a'\in A: a' \subseteq f^o b'f \implies \Delta_X \subseteq
						f^o b'	f$.\\
						Therefore $a=a\Delta_X \subseteq (f^o bf) (f^o b'f)
						\subseteq f^o bb'f$
				\end{enumerate}
			\item	\begin{enumerate}[label=(\roman*)]
				\item $(\implies )$ Let f be fully dense i.e. $B=f_*f^* = B.f.f^o.B$.
					showing that $\forall b \in B,\exists b'\in B: b'
					\subseteq bff^ob$:\\
					So, fix $b\in B$, as $B \leq B.f.f^o.B$, there exists $b' \in
					B$ such that $b' \subseteq bff^ob$.
				\item $(\impliedby)$ Suppose $\forall b \in B, \exists b' \in B:
					b' \subseteq bff^ob$. This gives $B\leq B.f.f^o .B$, in order
					to show equality, also need $B\geq B.f.f^o .B$.
					By quasi-uniformity of B, for any $b \in B$, $\exists b'\in B:b'b'
					\subseteq b$. Now, by Lemma 1.9.2,
					\[ f f^o \subseteq \Delta_Y \implies b'f f^o b' \subseteq b'
					\Delta_Y b'=b'b' \subseteq b\]

			\end{enumerate}
		\item \begin{enumerate}[label=(\roman*)]
			\item $(\implies)$ Let f be topologically dense, going to show that
				$\forall b\in B$ , $ (y,y) \in bff^ob$. So, fix any $ b\in B$ and
				$y \in Y$. Now, by definition of $\overline{f(X)}=Y $ , we get
				\[\exists x\in X \text{ such that }
				(f(x),y) \in b \text{ and } (y,f(x)) \in b \]
				Re-writing the above statement in terms of relations, and
				considering f as a relation:
				\begin{align}
					(f(x),y) \in b &\text{ gives }  x(b \circ f)y \text{ i.e. }
					y \in (b \circ f)(x)\\
					(y,f(x)) \in b &\text{ gives } f(x)\subseteq b(y)
				\end{align}
				Repeatedly applying Lemma 1.9.3 to (2),
				\[ f(x) \subseteq b(y) \implies
					\big(f \circ f^o \big)(f(x) \subseteq \big(f \circ f^o \big)b(y)
					\implies \big(f \circ f^o \circ f\big)(x) \subseteq
				\big(f \circ f^o \circ b \big)(y)\]
				Applying Lemma 1.9.1 to the above statement gives that
				\[f(x)=\big(f \circ f^o \circ f\big)(x) \subseteq
				\big(f \circ f^o \circ b \big)(y)  \]
				Applying Lemma 1.9.3 and then using (1) to this inequality completes the result:
				\[ f(x) \subseteq \big(ff^ob \big)(y)
					\implies \big(b \circ f\big)(x) \subseteq
				\big(bff^ob \big)(y) \implies y \in \big(bff^ob \big)(y) \text{ i.e. } y\big(bff^ob \big)y   \]
			\item $(\impliedby)$ Fix any $y \in Y$ and $b \in B$. Also, suppose that
				$\Delta_Y \leq bff^o b$. As f is a function with domain as X,
				$f^o :Y \to X$, $\phi \neq (f^o \circ b)(y)
				\subseteq  X$. So, fix $x\in (f^o \circ b)(y)$, going to show that
				$(f(x),y)\in b$	and $(y,f(x)) \in b$. Again, while viewing f as a
				relation.
				\[ \Delta_Y \leq bff^ob
					\implies \Delta_Y(y) \subseteq bf f^o b(y)=\big(bf\big)
				(f^o b(y))\]
				Using Lemma 1.9.3 on the above statement, gives $y \in (bf)(x)
				\text{ i.e. } (f(x),y)\in b $.\\
				Applying Lemma 1.9.3 to f, and then using Lemma 1.9.4,
				\begin{equation*}
					ff^o  \subseteq \Delta_Y \implies f f^o b \subseteq \Delta_Yb=b
				\end{equation*}
				Thus $f f^o b(y) \subseteq b(y)$ and hence $f(x) \subseteq b(y) \implies
				(y,f(x)) \in b$
		\end{enumerate}
	\item
		\begin{enumerate}[label=(\roman*)]
			\item
				($\implies $) Let f be topologically dense. As B is a quasi-uniformity, for any $b \in B$,
				\begin{equation}\exists b' \in B : b'b' \subseteq b \text{ and } \Delta_Y \subseteq b'
					\implies b'=b'\Delta_Y \subseteq b'b' \subseteq b
				\end{equation}
				By the characterisation of topologically dense in (c), have that $\Delta_Y \subseteq b'f f^o b'$.
				Now, using the (3) and Lemma 1.9.2,
				\[ \Delta_Y \subseteq b'f f^o b' \implies b'=b'\Delta_Y \subseteq b'b'f f^o b' \subseteq bf f^o b'
				\subseteq bf f^o b\]
				Hence, we have $b'\in B : b' \subseteq bf f^o b$ giving us that f is fully dense (from (b)).
			\item ($\impliedby$) From (b), we have for $b \in B$, the existstence of $b' \in B$ such that $
				b' \subseteq bf f^o b$. As B is a quasi-uniformity, $\Delta_Y \subseteq b'$. So,
				$\Delta_Y \subseteq bf f^o b$, and from (c), this gives us that f is topologically dense.
		\end{enumerate}
\end{enumerate}
\end{proof}
%}}}
%{{{ A~ is a quasi-uniformity on PX
\begin{definition}[$PX$] %{{{ PX		Definition
	$PX:=\{\psi : \psi:(X,A) \carrow 1 \text{ is a promodule} \}$
\end{definition}
\begin{definition}[$\tilde{a}$] %{{{ a~		Definition
	for any $a\in A$, $\tilde{a}$ is defined to be a relation $PX \to PX$ as
	\[ \text{ for } \phi,\psi \in PX, \phi \tilde{a} \psi \text{ only if } \phi \leq \psi.a \]
\end{definition}
\begin{proop}[Prorelation $\tilde{A}$] %{{{ A~		Definition
	The set, $\tilde{A}:=\{\tilde{a}:a \in A\}$ defines a quasi-uniformity on $PX$.
\end{proop}
\begin{proof}
	First, need to show that $\tilde{A}$ is a prorelation,
	\begin{enumerate}[label=(\roman*)]
		\item (Partial order) Define, for any two relations $\tilde{a},\tilde{b}:PX \to PX $,
			that $ \tilde{a} \leq \tilde{b}$ only if $a \subseteq b$
		\item(Down-Directed) Need that $\forall \tilde{a} ,\tilde{b} \in \tilde{A} ,
			\exists \tilde{c} \in A : c \subseteq a,b$\\
			$\tilde{a} ,\tilde{b} \in A \implies a,b \in A \implies \exists c \in A: c \subseteq
			a,b \implies \tilde{c} \leq \tilde{a} ,\tilde{b} $
		\item (Upset) Need that, for any relation $l:PX \to PX$ , if  $\exists \tilde{k} \in
			\tilde{A} \text{ such that }  l \geq \tilde{k} \text{ , then } l \in \tilde{A} $\\
			Fix any $k:PX \to PX$, and $\tilde{k} \in \tilde{A}$ such that $l\geq \tilde{k} $\\
			Now, $k$ is a relation between promodules $X \carrow 1$. Thus, it can be thought
			of as a relation on X,\\
			$a:=\{(x,y): x \in domain(\psi) and y \in domain(\phi)
			\text{ whenever } \exists \psi,\phi \in PX: \psi l \phi\}$ \\
			So, $l=\tilde{a}$ and thus, $\tilde{a} \geq \tilde{k} \implies a \supseteq k
			\implies a\in A \implies l \in \tilde{A} $
	\end{enumerate}
	Now to show that the other two conditions hold,
	\begin{enumerate}[label=(\roman*)]
		\item need that $\forall \tilde{a} \in \tilde{A} , \forall \psi \in PX, \psi \tilde{a} \psi$\\
			So, need to show that $\psi \leq \psi.a$ i.e. $\forall p\in \psi, \exists q\in \psi:
			q \subseteq p.a$. Take $q:=p$, and as A is a quasi-uniformity,
			$\Delta_X \subseteq a \implies p=p.\Delta_X \subseteq p.a$
		\item Need that $\forall \tilde{a} \in \tilde{A} , \exists \tilde{b}\in \tilde{A}:
			\tilde{b}\tilde{b} \leq \tilde{a} $\\
			Before that, showing , for any $x,y \in A, \tilde{x} \tilde{y} \leq \widetilde{xy} $
			i.e. $\forall \psi, \phi \in PX$ , $\psi(\tilde{x}.\tilde{y})\phi \implies
			\psi \widetilde{xy} \phi $\\
			Let $\psi_1(\tilde{a} .\tilde{b} )\psi_3 \implies \exists \psi_2:
			\psi_1 \tilde{b} \psi_2 \tilde{a} \psi_3 \implies
			\psi_1 \leq \psi_2.b \text{ and } \psi_2 \leq \psi_3.a
			\implies \psi_1 \leq \psi_2.b \leq \psi_3.ab \implies \psi_1(\widetilde{ab})\psi_3$
			\\
			Fix any $\tilde{a}\in \tilde{A} \implies a\in A \implies \exists b \in A: b \circ b \subseteq
			a \implies \widetilde{bb} \leq \tilde{a} \implies \tilde{b} \tilde{b} \leq
			\widetilde{bb} \leq \tilde{a}  $

	\end{enumerate}
\end{proof}
%}}}
\newpage
%{{{ Yoneda Embedding
\begin{proop}[Yoneda Embedding]
\item For a quasi-uniform space $(X,A)$, function $y_X:X \to PX$ is defined by $x\mapsto x^*$ for $x \in X$.
	\begin{enumerate}[label=(\alph*)]
		\item $y_X:(X,A) \rightarrow (PX,\tilde{A})$ is a uniformly continuous map
		\item $y_X:(X,A) \rightarrow (PX,\tilde{A})$ is fully faithful
	\end{enumerate}
\end{proop}
\begin{proof}
\item \begin{enumerate}[label=(\alph*)]
	\item In order to show $y_X$ is uniformly continuous, need to show that $y_X.A \leq \tilde{A}.y_X $.
		By definition of $\leq$ , need $\forall a\in A, \exists b \in A:
		y_X \circ b \subseteq \tilde{a} \circ y_X $. Applying the relations
		to some element, x of the set X:
		\begin{equation} \big(y_X \circ b\big)(x) \subseteq \big( \tilde{a} \circ y_X\big)(x) \implies
		y_X(b(x)) \subseteq \tilde{a}(x^*) \end{equation}

		So, for the condition given by (4) to hold, if $y \in b(x)$, then it's required that
		$y^*=y_X(y) \in \tilde{a} (x^*)$ i.e. $x^* \tilde{a}y^*$. Using the definition of $x^*,y^*$
		and $\tilde{a}$,
		\begin{equation} x^* \tilde{a}y^* \iff x^o.A\leq y^o.A.a \iff
		\forall a' \in A, \exists a'' \in A: x^oa'' \subseteq y^oa'a  \end{equation}
		Now, fix any $a \in A$, $x\in X$. Thus, quasi-uniformity of A, gives $a'' \in A$ such that
		$a''a''\subseteq a$.Also, choose some $y \in a''(x)$. Hence, in
		order to show that the condition from (5) holds, need that
		$\forall b \in A, x^o a'' \subseteq y^oba$, and by applying the relations to an element z
		gives the following condition
		\begin{equation} \forall b \in B, \forall x \in X \text{ , }
		\big(x^oa''\big)(z) \subseteq \big(y^oba\big)(z) \end{equation}
		Examining the left side of (6),
		\[ \big( x^oa''\big)(z)=x^o (a''(z))= \begin{cases}
			\phi &\text{ if }a''(z)\neq x \\
			\star & \text{ if } a''(z)=x
		\end{cases} \]
		Thus, to show that (6) holds, need to show that (for any $b\in A$ and $z \in X$):
		\begin{equation} a''(z)=x \implies z(y^oba)\star \text{ i.e. } y\in(ba)(z)
		\end{equation}
		To show that (7) holds, fix any $z\in X: a''(z)=x$. Also, by our choice of $y$,
		have that $y \in a''(x)$. And as $b\in A$, it's reflexive, giving that $y \in b(y)$.
		So, by composition of relations, we get:
		\[ za''x \text{ , }  xa''y \text{ and } yby \implies z(a''a''b)y \implies z(ab)y \text{ i.e. }
		y \in (ba)(z)\]
	\item By using Proposition2.3(a), need to show that $A\geq y_X^o.\tilde{A}.y_X$ i.e. $\forall
		a\in A, \exists \tilde{b}\in \tilde{A} :  a \supseteq y_X^o \text{ } \tilde{b} \text{ } y_X $.
		Applying to an element, $x\in X$ gives the condition
		\begin{equation}
			\Big( y_X^o \text{ } \tilde{b} \text{ } y_X \Big)(x) \subseteq a(x)
			\implies \Big( y_X^o \text{ } \tilde{b} \Big) (x^*)= y_x^o
			\Big(\tilde{b}(x^*)\Big) \subseteq a(x)
		\end{equation}
		Thus, if $y^* \in PX$ such that $x^* \tilde{b} y^*$, then
		$y \in y_x^o\Big(\tilde{b}(x^*)\Big)$. Now, for (8) to hold, $y \in a(x)$ i.e. $xay$. Thus,
		need only to show that for any $a\in A, \exists b \in A $ such that $\forall x,y \in X,
		x^* \tilde{b}y^* \implies xay $. So, fix $a\in A$, and take $b \in A: bb \subseteq a$.
		Now, let $x^* \tilde{b}y^*$ i.e. $x^o.A \leq y^o .A .b$.
		Hence, $\exists c \in A: x^oc \subseteq y^o bb$. And as c is reflexive,
		\[ xcx \implies x(cx^o)\star \implies x(bby^o)\star \implies x(bb)y \implies xay \]



\end{enumerate}
\end{proof}
%}}}





\newpage
asd
\end{document}
