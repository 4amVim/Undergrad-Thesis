\documentclass{article}
\usepackage[margin=0.7in]{geometry}
\usepackage{amsmath,amsthm,amssymb,xcolor}
\usepackage{enumitem}

\theoremstyle{definition}
\newtheorem{example}{Example}[section]

\theoremstyle{definition}
\newtheorem{definition}{Definiton}[section]

\begin{document}
\section{Categories} \label{sec:Categories}
\begin{definition}[Category] %{{{ Category		Definition
A category, $\mathcal{A}$ is defined to have each of the following,
\begin{enumerate}[label=(\roman*)]
	\item A collection of objects, denoted by ob($\mathcal{A}$) and written A,B,C $\in \mathcal{A}$.\\
		Such that each object has an `identity', $ 1_A \in \mathcal{A}(A,A) , 1_B \in \mathcal{A}(B,B), 1_C \in \mathcal{A}(C,C)$
	\item For each pair of objects, a collection of `links'/morphisms between them, denoted by $\mathcal{A}(A,B)$ \\and written as f $ \in \mathcal{A}(A,B) \; g \in \mathcal{A}(B,C) $. Such that,
		\begin{enumerate}[label=(\alph*)]
			\item morphisms with matching domain,co-domain can be `chained'/composed $ (g,f)=g \circ f $
			\item with this composition being associative, $ (h \circ g)\circ f=h \circ ( g \circ f) $
			\item and they are `fixed' by the identity $ f \circ 1_A =f= 1_B \circ f $
		\end{enumerate}
\end{enumerate}
\end{definition}
\begin{example} {\textbf{Non-trivial Identity \;}}%{{{Non-trivial Identity 		Example
Consider the objects to be groups, and morphisms to be direct product between them:
\begin{enumerate}[label=\roman*]%{{{Definition
	\item ob $ (\mathcal{A}) = \{ G | $  G is a group$ \} $
	\item $  \mathcal{A}(A,B) : = A \times B $
	\item  $ \mathcal{A}(B,C) \ \circ \mathcal{A}(A,B) \mapsto \mathcal{A}(A,C) $
\end{enumerate}
So, there's a unique morphism between any two objects i.e groups. And the identity morphism,
\[ \forall A,B  \in \mathbb{\mathcal{A} }\;, \text{ if f $ \in \mathcal{A}(A,B)$, then }  f \circ 1_A \in \mathcal{A}(A,B) \times \mathcal{A}(A,A) \mapsto \mathcal{A}(A,B) \text{ and } 1_B \circ f \in \mathcal{A}(B,B) \times \mathcal{A}(A,B) \mapsto \mathcal{A}(A,B) \]
Thus, $ ob(\mathcal{A} ) $ along with $ \circ  $ is actually a group. And hence has a unique inverse.
\textcolor{blue} {But how exactly?}
\end{example}
\begin{example}{\textbf{Set}} %{{{ Set		Example
	The objects are defined to be sets, and morphisms are the functions between them, with the usual composition law:
	\begin{enumerate}[label=\roman*] %{{{ Definition
	\item ob $ (\mathcal{A}) = \{ S | $  S is a set$ \} $
	\item $  (f:A \mapsto B )\in \mathcal{A}(A,B)$
	\item  $ (g \in \mathcal{A}(B,C))  \circ (f \in \mathcal{A}(A,B)) \mapsto  g(f) \in \mathcal{A}(A,C) $
	\end{enumerate}
\end{example}
\begin{example}{\textbf{Grp}} %{{{ Grp		Example
Objects are groups,with homomorphisms between them being the morphisms, and composition being as usual:
	\begin{enumerate}[label=\roman*]
	\item $  ob(\mathcal{A} ) = \{G| G \text{ is a group } \}  $
	\item $ \mathcal{A} (A,B)= Hom(A,B) \text{ i.e. all } f \text{ such that } \forall x,y \in A f((x) \, ._A \, (y))=(f(x)) \,._B\,(f(y))      $
	\item  composition is defined as that between two group homomorphisms
\end{enumerate}
In this example, the set of all morphisms along with composition forms a group.
\end{example}
\begin{example}{\textbf{Ring}} %{{{ Ring		Example
Objects are rings, and arrows are ring homomorphisms between them.
\begin{enumerate}[label=\roman*]
	\item $  ob(\mathcal{A} ) = \{G| G \text{ is a ring } \}  $
	\item $ \mathcal{A} (A,B) = Hom(A,B) $
	\item  composition is defined as that between two ring homomorphisms
\end{enumerate}
\end{example}
\begin{example}{\textbf{Vect$ _k$ }} %{{{ Vect$ _k$ 		Example
Objects are vector spaces \textit{over field k} , and the morphisms between them are linear transformations
	\begin{enumerate}[label=\roman*]
	\item $ ob(\mathcal{A} ) = \{A | A $ is  a vector space$   \}$
	\item $ \mathcal{A} (A,B) = \mathcal{L} (A,B) $
	\item  composition is defined as that of linear transformations
\end{enumerate}
\end{example}
\begin{definition}[Isomorphism] %{{{ Isomorphism		Definition
	An isomorphism, between objects, is a morphism between them such that it's `inverse' is also a morphism. So,
	\[ f: A \mapsto B \text{ is an isomorphism } \iff \exists g \in \mathcal{A}(B,A): gf=1_A \text{ and } fg=1_B\]
\end{definition}

\section{Functors}
\label{sec:Functors}
\begin{definition}{Functor}%{{{Functor Definition
	A functor is a map between categories, written $ F:\mathcal{A} \mapsto \mathcal{B}  $ , consists :
\begin{enumerate}[label=(\roman*)]
	\item function taking objects of $ \mathcal{A}  $ to those of $ \mathcal{B} $ i.e. $ ob(\mathcal{A} ) \mapsto ob(\mathcal{B} ) $ . Written as $ A \mapsto F(A) $ .
	\item associative, identity-preserving function taking links between objects of $ \mathcal{A}  $ to those for $ \mathcal{B}  $ , $ f\mapsto F(f) $  , i.e.
		\begin{align*}
					\forall A,B \in \mathbb{\mathcal{A} }, \mathcal{A}(A,B) \mapsto \mathcal{B}(F(A),F(B)) \text{ such that }   &  (a)\; f\in \mathcal{A} (A,B) \,,g \in \mathcal{A}(B,C) \implies     F(g \circ  f) = F(g) \circ  F(f) = F(g \circ f) \\
			      &  (b) A \in \mathcal{A} \implies F(1_A )=1_{F(A)}
		 \end{align*}
\end{enumerate}
\end{definition}
\begin{example}{Forgetful Functors}
	\begin{enumerate}[label=(\alph*)]
		\item
	\end{enumerate}
\end{example}

\section{Natural Isomorphisms}
\label{sec:Natural Isomorphisms}
\begin{example}
\end{example}


\end{document}

