\documentclass{article}
\usepackage[margin=0.7in]{geometry}
\usepackage{amsmath,amsthm,amssymb,xcolor}
\usepackage{enumitem,mathtools}

\theoremstyle{definition}
\newtheorem{example}{Example}[section]

\theoremstyle{definition}
\newtheorem{definition}{Definiton}[section]

\begin{document}
\section{Categories} \label{sec:Categories}
\begin{definition}[Category] %{{{ Category		Definition
A category, $\mathcal{A}$ is defined to have each of the following,
\begin{enumerate}[label=(\roman*)]
	\item A collection of objects, denoted by ob($\mathcal{A}$) and written A,B,C $\in \mathcal{A}$.\\
		Such that each object has an `identity', $ 1_A \in \mathcal{A}(A,A) , 1_B \in \mathcal{A}(B,B), 1_C \in \mathcal{A}(C,C)$
	\item For each pair of objects, a collection of `links'/morphisms between them, denoted by $\mathcal{A}(A,B)$ \\and written as f $ \in \mathcal{A}(A,B) \; g \in \mathcal{A}(B,C) $. Such that,
		\begin{enumerate}[label=(\alph*)]
			\item morphisms with matching domain,co-domain can be `chained'/composed $ (g,f)=g \circ f $
			\item with this composition being associative, $ (h \circ g)\circ f=h \circ ( g \circ f) $
			\item and they are `fixed' by the identity $ f \circ 1_A =f= 1_B \circ f $
		\end{enumerate}
\end{enumerate}
\end{definition}
\begin{example} {\textbf{Non-trivial Identity \;}}%{{{Non-trivial Identity 		Example
Consider the objects to be groups, and morphisms to be direct product between them:
\begin{enumerate}[label=\roman*]%{{{Definition
	\item ob $ (\mathcal{A}) = \{ G | $  G is a group$ \} $
	\item $  \mathcal{A}(A,B) : = A \times B $
	\item  $ \mathcal{A}(B,C) \ \circ \mathcal{A}(A,B) \mapsto \mathcal{A}(A,C) $
\end{enumerate}
So, there's a unique morphism between any two objects i.e groups. And the identity morphism,
\[ \forall A,B  \in \mathbb{\mathcal{A} }\;, \text{ if f $ \in \mathcal{A}(A,B)$, then }  f \circ 1_A \in \mathcal{A}(A,B) \times \mathcal{A}(A,A) \mapsto \mathcal{A}(A,B) \text{ and } 1_B \circ f \in \mathcal{A}(B,B) \times \mathcal{A}(A,B) \mapsto \mathcal{A}(A,B) \]
Thus, $ ob(\mathcal{A} ) $ along with $ \circ  $ is actually a group. And hence has a unique inverse.
\textcolor{blue} {But how exactly?}
\end{example}
\begin{example}{\textbf{Set}} %{{{ Set		Example
	The objects are defined to be sets, and morphisms are the functions between them, with the usual composition law:
	\begin{enumerate}[label=\roman*] %{{{ Definition
	\item ob $ (\mathcal{A}) = \{ S | $  S is a set$ \} $
	\item $  (f:A \mapsto B )\in \mathcal{A}(A,B)$
	\item  $ (g \in \mathcal{A}(B,C))  \circ (f \in \mathcal{A}(A,B)) \mapsto  g(f) \in \mathcal{A}(A,C) $
	\end{enumerate}
\end{example}
\begin{example}{\textbf{Pre-ordered Set}} %{{{ Pre-ordered Set		Example
A pre-ordered , can be made into a category via the binary operation, so that the morphism $ a \mapsto b $ is defined iff $ a\leq b $ where $ \leq  $ is the preorder.
The interesting part about this category is that there's at most one morphism between any two objects.
\end{example}
\begin{example}{\textbf{Grp}} %{{{ Grp		Example
Objects are groups,with homomorphisms between them being the morphisms, and composition being as usual:
	\begin{enumerate}[label=\roman*]
	\item $  ob(\mathcal{A} ) = \{G| G \text{ is a group } \}  $
	\item $ \mathcal{A} (A,B)= Hom(A,B) \text{ i.e. all } f \text{ such that } \forall x,y \in A f((x) \, ._A \, (y))=(f(x)) \,._B\,(f(y))      $
	\item  composition is defined as that between two group homomorphisms
\end{enumerate}
In this example, the set of all morphisms along with composition forms a group.
\end{example}
\begin{example}{\textbf{Ring}} %{{{ Ring		Example
Objects are rings, and arrows are ring homomorphisms between them.
\begin{enumerate}[label=\roman*]
	\item $  ob(\mathcal{A} ) = \{G| G \text{ is a ring } \}  $
	\item $ \mathcal{A} (A,B) = Hom(A,B) $
	\item  composition is defined as that between two ring homomorphisms
\end{enumerate}
\end{example}
\begin{definition}[Dual Category] %{{{ Dual Category		Definition
	Given a category $ \mathcal{A}  $ , it's opposite/dual, $ \mathcal{A} ^{op}  $ is a category with the same objects, but reversed arrows, while keeping the composition :
	\[ ob(\mathcal{A}^{op} ) =ob(\mathcal{A}) \text{ and } \forall A,B \in ob(\mathcal{A} )\; , \; \mathcal{A} ^{op}(A,B)=\mathcal{A} (B,A) \]
\end{definition}
\begin{example}{\textbf{Vect$ _k$ }} %{{{ Vect$ _k$ 		Example
Objects are vector spaces \textit{over field k} , and the morphisms between them are linear transformations
	\begin{enumerate}[label=\roman*]
	\item $ ob(\mathcal{A} ) = \{A | A $ is  a vector space$   \}$
	\item $ \mathcal{A} (A,B) = \mathcal{L} (A,B) $
	\item  composition is defined as that of linear transformations
\end{enumerate}
\end{example}
\begin{definition}[Isomorphism] %{{{ Isomorphism		Definition
	An isomorphism, between objects, is a morphism between them such that it's `inverse' is also a morphism. So,
	\[ f: A \mapsto B \text{ is an isomorphism } \iff \exists g \in \mathcal{A}(B,A): gf=1_A \text{ and } fg=1_B\]
\end{definition}
\begin{definition}[Product Category] %{{{ Product Category		Definition
	Somewhat like a cartesian product of categories. Given categories $ \mathcal{A}  $ and $ \mathcal{B}  $ , $ \mathcal{A}  \times \mathcal{B}  $ is defined as:
	\begin{enumerate}[label=\roman*]
		\item $ ob( \mathcal{A}  \times \mathcal{B}) := ob(\mathcal{A} ) \times ob(\mathcal{B} )$
		\item $(\mathcal{A} \times \mathcal{B})((A,B),(A',B')):=\mathcal{A} (A,A')\times\mathcal{B} (B,B') $
		\item $ (f,g) \in \mathcal{A}\times \mathcal{B} ((A,B),(C,D)) \; , \;(a,b) \in \mathcal{A}\times \mathcal{B} ((C,D),(E,F)) \implies (a,b) \circ (f,g) := (a \circ f, b \circ g)$
		\item $ \forall (A,B) \in ob(\mathcal{A}\times \mathcal{B})\; , \; 1_{(A,B)}:=(1_A,1_B) $
	\end{enumerate}
\end{definition}


\begin{example}[CAT] %{{{ CAT		Example
	The category of all categories with morphisms being functors.
	\begin{enumerate}[label=\roman*]
	\item $ ob(\mathcal{A} ) = \{A | A $ is  a category$   \}$
	\item $ \mathcal{A} (A,B) = F (A,B) $
	\item $ F: \mathcal{A}  \mapsto \mathcal{B}\;,\; G: \mathcal{B} \mapsto \mathcal{C} \implies G \circ F := H: \mathcal{A} \mapsto \mathcal{C} $
	\end{enumerate}
 And thus, the identity of $ \mathcal{A}  $ is the functor, $ 1_{\mathcal{A}} : \mathcal{A}  \mapsto \mathcal{A}  $
\end{example}

\section{Functors}
\label{sec:Functors}
\begin{definition}[(Covariant)Functor]%{{{Functor Definition
	A functor is a map between categories, written $ F:\mathcal{A} \mapsto \mathcal{B}  $ , consists :
\begin{enumerate}[label=(\roman*)]
	\item function taking objects of $ \mathcal{A}  $ to those of $ \mathcal{B} $ i.e. $ ob(\mathcal{A} ) \mapsto ob(\mathcal{B} ) $ . Written as $ A \mapsto F(A) $ .
	\item associative, identity-preserving function taking links between objects of $ \mathcal{A}  $ to those for $ \mathcal{B}  $ , $ f\mapsto F(f) $  , i.e.
		\begin{align*}
					\forall A,B \in \mathbb{\mathcal{A} },\; \mathcal{A}(A,B) \mapsto \mathcal{B}(F(A),F(B)) \text{ such that }   &  (a)\; f\in \mathcal{A} (A,B) \,,g \in \mathcal{A}(B,C) \implies     F(g \circ  f) = F(g) \circ  F(f) = F(g \circ f) \\
			      &  (b) A \in \mathcal{A} \implies F(1_A )=1_{F(A)}
		 \end{align*}
\end{enumerate}
\end{definition}

\begin{example}{\textbf{Forgetful Functors}} %{{{ Forgetful Functors		Example
They essentially ignore some of the structure of the 'domain'.
\begin{enumerate}[label=(\alph*)]
		\item $ U:Grp\mapsto Set $ takes groups to their underlying set, and homomorphisms to maps between the sets.Similarly, $ Ring \mapsto Set $ and $ Vect_k \mapsto Set $
		\item Let $ Ab $ be the category of abelian groups, then,   $ U:Ring \mapsto Ab $ takes rings to their additive group, `forgetting' the multiplicative group. And if $ Mon $ is the category of monoids, $ U:Ring \mapsto Mon $ `forgets' the additive group.
		\item $ U:Ab \mapsto Grp $ just takes each abelian group to itself, and does the same for (homo)morphisms.
	\end{enumerate}
\end{example}
\begin{example}{\textbf{Free Functors}} %{{{ Free Functors		Example
	\begin{enumerate}[label=(\alph*)]
		\item let $ F(S) $ denote the free group on a set S. Then, $ F:Set \mapsto Grp $ is a `free' functor taking sets to their free group, and thus the maps between them become homomorphisms between their free groups. As,
			\[ f \in Set(S,S') \mapsto F(f) \in Grp(F(S),F(S')) \text{ i.e. } f:s \mapsto s' \text{ goes to } F(f) \text{ defined as } g:=F(s) \mapsto f(g)    \]
		\item Similarly, there's a `free' functor $ F:Set \mapsto CRing $ to the category of commutative rings. Defined as taking sets to polynomial rings having each element as a commuting variable, and coefficients from $ \mathbb{Z}$  .
		\item Fix any field $ \mathbb{F} $ , and define F(S) to be a vector space over it with (Shrauder)basis S. As basis completely determines a vector space,
			\[ F(S):= \{ L: S \mapsto \mathbb{F} \; |\; \text{ L takes only finitely many s } \in \text{ S to a non-zero k } \in \mathbb{F} \} \text{ i.e. } F(S) \mapsto \sum_{s \in S} k_ss  \]
			\[ \text{ and } f \in Set(S,S') \text{ goes to } F(f): L(F(S),F(S'))\]
	\end{enumerate}
\end{example}
\begin{example}
	Let $ \mathcal{G}, \mathcal{H}$ be the one object categories of monoids G,H respectively. Then, due to composition being associative and identity preserving, possible functors are precisely the homomorphisms.
\end{example}
\begin{example}
	Let monoid G be regarded as a one-object category, $ \mathcal{G} $. Then, functor $ F:\mathcal{G} \mapsto Set $ has one object, a set S. And, $ \forall g \in G, \; F(g):S \mapsto S  $ is defined as $ (F(g))(s)=g*s$ where * is an associative identity-preserving function.Thus, $ (g,s) \mapsto g.s $ i.e. S is a left G-set.
\end{example}

\begin{definition}[Contravariant Functor] %{{{ Contravariant Functor		Definition
	For categories $ \mathcal{A} \text{ and }\mathcal{B}  $ , $ \mathcal{A} ^{op} \mapsto \mathcal{B} $ is a contravariant functor from $ \mathcal{A}  $ to $ \mathcal{B}  $.
\end{definition}
\begin{example}
	Let k be a field and $ V,V',W $ be vector spaces over it. Then fixing W,
	\[ \forall f \in Hom(V,V'), \exists f^*:Hom(V',W) \mapsto Hom(V,W) \text{ as } g \in Hom(V',W) \implies V \xrightarrow{f}V'\xrightarrow{g}W  \]
	\textcolor{red}{recheck the following argument}
	So, for each $ V \in ob(Vect_k)$, $ Hom(V,W) $ defines a (contravariant) functor on $ Vect_k $, as, fixing W=V, the above argument can be restated as
	\[ f \in Vect_k^{op}(V',V)=Vect_k(V,V') \mapsto g \in Vect_k(V',V)   \]
\end{example}

\begin{definition}[Faithful Functor] %{{{ Faithful Functor		Definition
	A functor $ F:\mathcal{A}  \mapsto \mathcal{B} $ is faithful iff the map$ \mathcal{A} (A,A') \mapsto \mathcal{B}(F(A),F(A'))$ is injective for any $ A,A' \in \mathcal{A} $ i.e. each arrow between $ A,A' $ goes to at most one arrow between $ F(A),F(A') $
\end{definition}
\begin{definition}[Full Functor] %{{{ Full Functor		Definition
	A functor $ F:\mathcal{A}  \mapsto \mathcal{B} $ is full iff the map$ \mathcal{A} (A,A') \mapsto \mathcal{B}(F(A),F(A'))$ is surjective for any $ A,A' \in \mathcal{A} $ i.e. each arrow between $  A,A' $goes to at least one arrow between $ F(A),F(A') $
\end{definition}

\begin{definition}[Subcategory] %{{{ Subcategory		Definition
A subcategory of $ \mathcal{A}  $ is a category with objects from $ \mathcal{A} $, but not necessarily all of them. Similarly for the morphisms.
\end{definition}

\begin{definition}[Full Subcategory] %{{{ Full Subcategory		Definition
A full subcategory of $ \mathcal{A} $ that retains as many morphisms of $ \mathcal{A} $ as possible.
\end{definition}



\section{Natural Isomorphisms}
\label{sec:Natural Isomorphisms}
\begin{example}
\end{example}


\pagebreak
To be continued.
\end{document}

