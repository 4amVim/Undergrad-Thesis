\documentclass{article}
\usepackage[margin=0.7in]{geometry}
\usepackage{amsmath,amsthm,amssymb,xcolor}
\usepackage{enumitem}

\theoremstyle{definition}
\newtheorem{example}{Example}[section]

\theoremstyle{definition}
\newtheorem{definition}{Definiton}[section]

\begin{document}
\section{Categories} \label{sec:Categories}
\begin{definition}[Category] %{{{Define Category
A category, $\mathcal{A}$ is defined to have each of the following,
\begin{enumerate}[label=(\roman*)]
	\item A collection of objects, denoted by ob($\mathcal{A}$) and written A,B,C $\in \mathcal{A}$.\\
		Such that each object has an `identity', $ 1_A \in \mathcal{A}(A,A) , 1_B \in \mathcal{A}(B,B), 1_C \in \mathcal{A}(C,C)$
	\item For each pair of objects, a collection of `links'/morphisms between them, denoted by $\mathcal{A}(A,B)$ \\and written as f $ \in \mathcal{A}(A,B) \; g \in \mathcal{A}(B,C) $. Such that,
		\begin{enumerate}[label=(\alph*)]
			\item morphisms with matching domain,co-domain can be `chained'/composed $ (g,f)=g \circ f $
			\item with this composition being associative, $ (h \circ g)\circ f=h \circ ( g \circ f) $
			\item and they are `fixed' by the identity $ f \circ 1_A =f= 1_B \circ f $
		\end{enumerate}
\end{enumerate}
\end{definition}

\begin{example} [Non-trivial Identity]
Consider the objects to be groups, and morphisms to be direct product between them:
\begin{enumerate}[label=\roman*]%{{{Definition
	\item ob $ (\mathcal{A}) = \{ G | $  G is a group$ \} $
	\item $  \mathcal{A}(A,B) : = A \times B $
	\item  $ \mathcal{A}(A,B) \ \circ \mathcal{A}(B,C) \mapsto \mathcal{A}(A,C) $
\end{enumerate}
So, there's a unique morphism between any two objects i.e groups. And the identity morphism,
\[ \forall A,B  \in \mathbb{\mathcal{A} }\;, \text{ if f $ \in \mathcal{A}(A,B)$, then }   1_A \circ f \in \mathcal{A}(A,A) \times \mathcal{A}(A,B) \mapsto \mathcal{A}(A,B) \]
Thus, $ ob(\mathcal{A} ) $ along with $ \circ  $ is actually a group. And hence has a unique inverse.
\textcolor{blue} {asd}
\end{example}
\section{Functors}
\label{sec:Functors}
\begin{example}
\end{example}

\section{Natural Isomorphisms}
\label{sec:Natural Isomorphisms}
\begin{example}
\end{example}


\end{document}

