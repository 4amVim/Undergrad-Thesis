\documentclass{article}
\usepackage[margin=0.7in]{geometry}
\usepackage{amsmath,amsthm,amssymb,xcolor}
\usepackage{enumitem}

\theoremstyle{definition}
\newtheorem{example}{Example}[section]

\theoremstyle{definition}
\newtheorem{definition}{Definiton}[section]

\begin{document}
This document will contain definitions and other trivial notes.
\section{`Trivial' definitons}

\begin{definition}[Monoid] %{{{ Monoid		Definition
	A semi-group with identity.
\end{definition}

\begin{definition}[Inclusion Map] %{{{ Inclusion Map		Definition
	A map $ f:A \mapsto B $ that takes $ x\in A $ to $ x \in B $
\end{definition}

\begin{definition}[(left) G-Set] %{{{ (left) G-Set		Definition
Let $ G $ be a group, and $ X $ a set. Then, f is a left group action on of G on X, or X is a left G-Set iff
\[ f:G \times X \mapsto X : \forall x \in X \; , \;  [ [ f(e_G,x)=x \text{ and } \forall a,b \in G \; , \; f(ab,x)=f(a, f(b,x))  \]
\end{definition}

\begin{definition}[Pre-order] %{{{ Pre-order		Definition
	A reflexive, transitive binary relation.
\end{definition}

\begin{definition}[Partial Order] %{{{ Partial Order		Definition
Pre-order that's antisymmetric.
\end{definition}

\begin{definition}[Total Order] %{{{ Total Order		Definition
Partial order with trichotomy.
\end{definition}





\end{document}

