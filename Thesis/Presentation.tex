\documentclass[8pt]{beamer}
%Compiled by Overleaf
\mode<presentation> { \usetheme{Frankfurt} %\setbeamertemplate{footline}[page number] 
\setbeamertemplate{navigation symbols}{} }
\usepackage{amsthm,amsfonts,amssymb,amscd,amsxtra,tikz-cd}
\usepackage{biblatex}
\usepackage{appendixnumberbeamer}

\addbibresource{references.bib}


\title[Yoneda Lemma and Quasi-Uniform Spaces]{ \Huge Yoneda Lemma and Quasi-Uniform Spaces}

\author{\huge Ayush Rawat} % Your name
\institute[SNU] % Your institution as it will appear on the bottom of every slide, may be shorthand to save space
{\Large
Shiv Nadar University\\ % Your institution for the title page
\medskip}

\date{22 June 2020}
\newtheorem{defi}{Definition}
\newtheorem{thm}{Theorem}
\newtheorem{lema}{Lemma}
\newtheorem{prop}{Proposition}
\newtheorem{ex}{Example}
\newtheorem{coro}{Corollary}

\begin{document}

\begin{frame}
\titlepage % Print the title page as the first slide
\end{frame}
\begin{frame}{Abstract}
We work out the details of the proof for Yoneda Lemma using the text from \cite{Tom}.
	Roughly speaking, Yoneda Lemma allows us to embed locally small categories into $Set$ via
	representable functors. We then give two consequences of the Lemma: first is to show
	that Cayley's theorem
	from group theory is a particular case of Yoneda Lemma, and second is to
	derive Yoneda Embedding, a fully faithful functor from locally small categories to
	their presheaf category.
	Further,
	we discuss quasi-uniform spaces from the paper \cite{Clementino_Hofmann_2011}. Here
	we discuss categories of quasi-uniform spaces and Promodules. We define the Yoneda
	embedding and prove a (weak) Yoneda Lemma for quasi-uniform spaces. We stop our work
	here; though the paper goes on a step further to discuss the Cauchy completion monad for
	quasi-uniform spaces.
\end{frame}

\section{Introduction}
\begin{frame}{Representables}
\begin{definition} %{{{Category A op		Definition
	For any category $\mathcal{A} $, it's opposite category, $\mathcal{A}^{op}$
	is the category having the objects of $\mathcal{A}$.
	And for objects $A,B \in \mathcal{A} $, a morphism $f \in \mathcal{A}^{op} (A,B)$
	if and only if there is a morphism $g \in \mathcal{A}(B,A)$.
\end{definition}
\pause
\begin{prop} %{{{ H_A	Definition
	For a locally small category $\mathcal{A}$, fixing an object $A \in \mathcal{A} $ gives
	a functor, $H_A: \mathcal{A} ^{op} \rightarrow Set$ defined as:
	\begin{enumerate}
		\item For any object $B \in \mathcal{A} $ , $H_A(B):=\mathcal{A} (B,A)$.
		\item For any morphism, $g : X \rightarrow Y $ in $\mathcal{A}$,
			\[H_A(g): \mathcal{A} (Y,A) \rightarrow \mathcal{A}(X,A)
			\text{ is given by } p \mapsto p \circ g.\]
	\end{enumerate}
\end{prop}
\pause
	A functor $ \mathcal{A} ^{op} \to Set$ that is isomorphic to $H_A$
	is called a representable.
\end{frame}
\begin{frame}[fragile]{Required Results}
If a transformation is natural in two individual variables simultaneously, then it is natural their pair.
	\begin{lema}
		Let $\mathcal{A},\mathcal{B}$ and $\mathcal{C}$ be categories. Suppose there
		are functors $F,G:\mathcal{A} \times \mathcal{B} \to \mathcal{C}$.

		For every $A\in \mathcal{A}$, there are functors, $F^A,G^A:\mathcal{B} \to \mathcal{C}$
		defined as taking $B \in \mathcal{B}$ to $F(A,B),\, G(A,B)$ and
		morphism $f$ to $F((1_A,f)),\, G((1_A,f)) $. And, for every $B\in \mathcal{B}$,
		there are functors $F_B,G_B:\mathcal{A} \to \mathcal{C}$ defined similarly.

		A family of maps, $\big( \alpha_{A,B}: F(A,B) \to G(A,B)
		\big)_{A \in \mathcal{A} , B \in \mathcal{B} } $ is a natural transformation
		$F \to G$ if the following conditions are satisfied:
		\begin{enumerate}
			\item For each $A \in \mathcal{A}$, the family
				$\big( \alpha_{A,B}: F^A(B) \to G^A(B) \big)_{ B \in \mathcal{B} } $ is
				a natural transformation $F^A \to G^A$;
			\item For each $B \in \mathcal{B}$, the family
				$\big( \alpha_{A,B}: F_B(A) \to G_B(A) \big)_{ A \in \mathcal{A} } $ is
				a natural transformation $F_B \to G_B$.
		\end{enumerate}
	\end{lema}
\pause
	Following is an equivalent definition of natural isomorphism.
	\begin{lema}
		Let $ 	\begin{tikzcd}
			\mathcal{A}  \arrow[r,bend left=50, "F"{name=U, below}]
			\arrow[r, bend right = 50, "G"{name=D, above}]
		& \mathcal{B}
		\arrow[Rightarrow, from=U , to=D, "\alpha"]
		\end{tikzcd}$
		be a natural transformation. If for every $A \in \mathcal{A} $,
		$\alpha_A:F(A) \to G(A)$ is an isomorphism
		then $\alpha$ is a natural isomorphism.
	\end{lema}
\end{frame}
\section{Yoneda Lemma}
\begin{frame}{Yoneda Lemma}
		\begin{theorem}
			If $\mathcal{A} $ is a locally small category then, for any object $A \in \mathcal{A} $
			and $X \in [ \mathcal{A}^{op},Set]$,\\ there exists an isomorphism,
			\begin{equation} [ \mathcal{A} ^{op},Set ](H_A,X)\cong X(A) \text{ which is natural in } A \text{ and } X.\end{equation}
		\end{theorem}
		\pause
		Outline of the proof:
		\begin{itemize}
		    \item<1-> Define a map from $[ \mathcal{A} ^{op},Set ](H_A,X)$ to $X(A)$.
		    \item<2-> Define a map from $X(A)$ to $[ \mathcal{A} ^{op},Set ](H_A,X)$.
		    \item<3-> Show the maps to be mutually inverse.
		    \item<4-> Prove the resulting isomorphism natural in $A$.
		    \item<5-> Prove it natural in $X$.
		\end{itemize}
\end{frame}
\section{Consequences}
\begin{frame}
\frametitle{Cayley's Theorem}
\begin{theorem}{\textbf{Cayley's Theorem}} %{{{ Cayley's Theorem		Theorem
			Every group, $G$ is isomorphic to a subgroup of symmetric group on $G$.
\end{theorem}
Outline of the proof:
Take a group $G$.
		\begin{itemize}
		    \item<1-> Define category $\mathcal{A}$ to have a single
		    object, $\star$. With morphisms corresponding to elements of $G$.
		    With their product also being as elements of $G$. So that
		    $\mathcal{A}(\star,\star)$ is isomorphic to group $G$.
		    \item<2-> Prove that each member of the collection $[ \mathcal{A} ^{op},Set ](H_\star,H_\star)$ can be considered
		    as a bijection on $G$.
		    \item<3-> Show that the above collection of bijections is actually a group.
		    \item<4-> Use Yoneda Lemma to get a set isomorphism between
		    $[ \mathcal{A} ^{op},Set ](H_\star,H_\star)$ and $\mathcal{A}(\star,\star)$.
		    \item<5-> Prove this set isomorphism to be a group isomorphism.
		\end{itemize}
\end{frame}

\begin{frame}{Yoneda Embedding}
    \begin{definition} %{{{ Embedding of a category		Definition
			A category $\mathcal{A}$  is said to be embedded in a category $\mathcal{B}$ if
			and only if there exists a functor $F: \mathcal{A} \to \mathcal{B}$
			such that $F$ is full and faithful.
		\end{definition} %}}}
		\pause
		\begin{prop} %{{{ H_A	Definition
	For a locally small category $\mathcal{A}$, fixing an object $A \in \mathcal{A} $ gives
	a functor, $H_\bullet: \mathcal{A} \to [ \mathcal{A} ^{op},Set$ ] defined as:
	\begin{enumerate}
		\item For any object $B \in \mathcal{A} $ , $H_\bullet(B):=H_A(B)$,
		\item For any morphism, $g : X \rightarrow Y $ in $\mathcal{A}$,
			for any $K\in \mathcal{A}$, $H_\bullet(g)$ takes $p \in H_X(K)$ to 
			$g \circ p$ in $H_Y(K)$.
	\end{enumerate}
\end{prop}
\pause
\begin{theorem}{ \textbf{Yoneda Embedding}}
				Any locally small category can be embedded in the presheaf category on it.
			\end{theorem}
			\pause
			The proof proceeds by showing functor $H_\bullet$ to be
			full and faithful.
\end{frame}
\section{Definitions}
\begin{frame}{Prorelation}
\begin{definition}%{{{ Prorelation		Definition
				A prorelation is a partially ordered, down-directed, up-set of relations $X \to Y$.\\
				That is, $P \subseteq \mathcal{P}(X \times Y)$ is a prorelation if it satisfies the
				following conditions:
				\begin{enumerate}
					\item Partial Order: Containment of relations defines a partial order.
						That is, $r \subseteq s$  meaning that for any $(x,y) \in X \times Y$,
						if  $(x,y)\in r\;$ then $\;(x,y)\in s$.
					\item Down-directed: For any $r,s \in P$, there exists $t \in P $ such that
						$t\subseteq r \text{ and } t \subseteq s$.
					\item Up-set: For any relation $u:X\to Y$, if there exists $p \in P$ such that
						$p \subseteq u $ then $u \in P$.
				\end{enumerate}
			\end{definition}
		\pause
			\begin{example} %{{{ 		Example
				For any positive real number $\epsilon$, define a relation on $\mathbb{R}$ as
				$A_\epsilon=\{ (x,y) |\; |x-y|<\epsilon \}$.
				The collection of all relations on $\mathbb{R}$ that contains some
				$A_\epsilon$ will be a prorelation, $K$ on $\mathbb{R}$.
				That is, $K=\{ a:\mathbb{R} \to \mathbb{R} \, | \; a \supseteq A_\epsilon \text{ for some } \epsilon>0 \; \}$ forms a prorelation.
				If $k,l \in K$, then there exist $\delta ,\, \epsilon > 0$ such that
				$k \supseteq A_\delta$ and $l \supseteq A_\epsilon$. Thus, the relation $A_{
				\frac{\delta+\epsilon}{2}}$ is in both $k$ and $l$. Moreover,
				$K$ is an up-set by definition.
			\end{example}
			\pause
			\begin{lemma}
				Composition of two prorelations is a prorelation.
			\end{lemma}
\end{frame}

\begin{frame}{Quasi-Uniform Space}
Quasi-uniformity is a particular kind of prorelation.
    \begin{definition} %{{{ Quasi-uniformity		Definition
				A prorelation $P$ on a set $X$ is said to be a quasi-uniformity
				if it satisfies the following conditions:
				\begin{enumerate}
					\item Every relation in $P$ is reflexive. That is,
						for each $p \in P$, if $x \in X$ then $(x,x) \in p$.
					\item For each $p$ in $P$, there exists $p'$ in $P$ such that
						$p' \circ p' \subseteq p$.
				\end{enumerate}
			\end{definition}
			\pause
			\begin{example} %{{{ 		Example
				As earlier, define $A_\epsilon=\{ (x,y) |\; |x-y|<\epsilon \}$.
				The collection of all relations on $\mathbb{R}$ that contains some
				$A_\epsilon$, that is, $K=\{ a:\mathbb{R} \to \mathbb{R} \, | \; a \supseteq A_\epsilon \text{ for some } \epsilon>0 \; \}$ forms a prorelation.
				If $k,l \in K$, then there exist $\delta ,\, \epsilon > 0$ such that
				$k \supseteq A_\delta$ and $l \supseteq A_\epsilon$. Thus, the relation $A_{
				\frac{\delta+\epsilon}{2}}$ is in both $k$ and $l$. Moreover,
				$K$ is an up-set by definition.
			\end{example}
			\pause
			\begin{definition} %{{{Quasi-uniform space Definition
				If $X$ is a set, and $A$ is a quasi-uniformity on $X$, then (X,A) is a quasi-uniform space.
			\end{definition}
			\end{frame}
			
\begin{frame}
			The following gives us a partial order on prorelations:
			\begin{definition} %{{{ Containment of Prorelations		Definition

				For prorelations $P,Q:X\to Y$, if for each $q \in Q$, there exists $p \in P$ such that
				$p \subseteq q$, then we write $P\leq Q$.
			\end{definition}
			\pause
			We will require the following definition in the next slide.
				\begin{definition} %{{{ Uniformly Continuous function 		Definition
				A function, $f:(X,A) \to (Y,B)$ is said to be uniformly continuous if and only if $f.A \leq B.f$.
				That is, for each $b \in B$, there exists $a \in A$ such that
				$f \circ a \subseteq b \circ f$.
			\end{definition}
\end{frame}

\section{QUnif and ProMod}
\begin{frame}{Category definition}
On this slide, we define two categories.
\begin{definition}
    The collection of quasi-uniform spaces can be given a categorical structure by considering the uniformly continuous maps between two spaces as the morphisms between them. We call this category QUnif. Composition is as that of functions,
    and identity morphisms are the identity functions.
\end{definition}\pause
\begin{definition}%{{{ Promodule		Definition
				A prorelation, $\phi:X \to Y$ is called a promodule $\phi: (X,A) \to (Y,B)$  if it
				satisfies:
				\[ \phi.A \leq \phi \text{ and } B. \phi \leq \phi .\]
			\end{definition}
			\pause
			\begin{definition}
				Now, we define a 2-category called ProMod as having quasi-uniform spaces as 0-cells and the
			promodules between them being 1-cells.
			The promodule $A$ will work as the identity of $(X,A)$.

			Let promodules $P,Q: (X,A) \to (Y,B)$. Then, there
			is a 2-cell from $P$ to $Q$ if and only if $P \leq Q$ as prorelations.
		\end{definition}
\end{frame}
\begin{frame}{Functors from QUnif to ProMod}
We have a covariant functor from QUnif to ProMod:
\begin{prop} Functor $(\string _)_*:\text{QUnif} \to \text{ProMod}$ defined as:
				\begin{enumerate}
					\item for $(X,A) \in \text{QUnif}$, $(X,A)_*:=(X,A) \in \text{ProMod}$,
					\item for $f:(X,A) \to (Y,B)$ in QUnif,
						$f_* := B.f$,
				\end{enumerate}
			\end{prop}
			\pause 
			as well as a contravariant functor from QUnif to ProMod,
			\begin{prop} Functor $(\string _)^*:\text{QUnif}^{op}_{} \to \text{ProMod}$ defined as as:
				\begin{enumerate}
					\item for $(X,A) \in \text{QUnif}$, $(X,A)_*:=(X,A) \in \text{ProMod}$,
					\item for $f:(X,A) \to (Y,B)$ in QUnif,
						$f^* := f^{o}.B$.
				\end{enumerate}
			\end{prop}
\end{frame}

\begin{frame}{Topological definitions}
\begin{definition} %{{{Topologically Dense		Definition
				For any quasi-uniform space $(X,A)$, an element $x \in X$ is said to belong in the
				topological closure of set $M\subseteq X$ if and only if for each $a\in A$,
				there exists $y\in M$ such that $x\, a\, y$ and $y\, a\, x$.
			\end{definition}
			\pause

			\begin{definition} %{{{ Fully Faithful and Dense		Definition
				Let $f:(X,A) \to (Y,B)$ be a uniformly continuous function.
				\begin{enumerate}
					\item f is said to be fully faithful if and only if $f^* .f_*=A$.
					\item f is said to be fully dense if and only if $f_* .f^*=B $.
					\item f is said to be topologically dense of and only if $\overline{f(X)}=Y$.
				\end{enumerate}
			\end{definition}
			\pause

The following proposition gives us easier to apply versions of previous definitions.
\begin{prop}
				Let $f:(X,A) \rightarrow (Y,B)$ be a uniformly continuous map.
				\begin{enumerate}
					\item f is fully faithful if and only if $A= f^o.B.f$, that is
						$A\geq f^o.B.f$.
					\item f is fully dense if and only if for any $b\in B$, $\exists b' \in B$
						such that $b' \subseteq b\,f\,f^o \,b$.
					\item f is topologically dense if and only if for any $b\in B$,
						$\; b \, f\, f^o\,b$ is reflexive.
					\item f is fully dense if and only if f is topologically dense.
				\end{enumerate}
			\end{prop}
\end{frame}
\section{Yoneda Lemma in QUS}
\begin{frame}{Quasi-uniform space of promodules}

\begin{definition}%{{{ PX		Definition
	The set $PX$ is defined to be the collection of all promodules from the
	quasi-uniform space $(X,A)$ to the quasi-uniform space 1.
	\[PX:=\{\psi :(X,A) \to 1 | \psi \text{ is a promodule} \}\]
\end{definition}\pause
On this set, we can define a quasi-uniformity.
\begin{prop}%{{{ A~		Definition
	For any $a\in A$, $\tilde{a}$ is defined to be a relation on $PX$ as:
	\[ \text{ for } \phi,\psi \in PX, \; \phi \, \tilde{a} \, \psi \; \text{if and only if } \;	\phi \leq \psi.a \; \;.\]
	The set, $\tilde{A}:=\{\tilde{a}:a \in A\}$ defines a quasi-uniformity on $PX$.
\end{prop}\pause
Just like we could embed any category into its 
presheaf category via $H_\bullet$, we can embed any quasi-uniform
space into its 'quasi-uniform space of promodules':
\begin{prop}
For a quasi-uniform space $(X,A)$, function $y_X:X \to PX$ is defined by $x\mapsto x^*$ for $x \in X$.
	\begin{itemize}
		\item $y_X^{}:(X,A) \rightarrow (PX,\tilde{A})$ is a uniformly continuous map.
		\item $y_X^{}:(X,A) \rightarrow (PX,\tilde{A})$ is fully faithful.
	\end{itemize}
\end{prop}
\end{frame}

\begin{frame}{Yoneda Lemma for Quasi-Uniform Spaces}
To think of Yoneda Lemma in quasi-uniform spaces,
\begin{itemize}
    \item consider the promodule $(y_X^{})_*$ to be the representable $H_A$ in our initial statement of Yoneda Lemma,
    \item consider the promodule $\psi^*$ to be the presheaf $X$ in our initial statement of Yoneda Lemma.
\end{itemize}
This gives us a weak version of Yoneda Lemma that holds in quasi-uniform space:
\pause
\begin{theorem}%{{{ Yoneda Lemma		Theorem
	The following statements hold for any $\psi \in PX$:
	\begin{enumerate}
		\item $\psi \geq \psi^*.(y_X)_*$,
		\item $\psi \in \overline{y_X(X)} \implies \psi \leq \psi^*.(y_X)_*$.
	\end{enumerate}
\end{theorem} \pause 
And the full strength Yoneda Lemma holds only for elements of $PX$ that
are also in the topological closure of $y_X^{}(X)$:
\begin{coro}
		For $\psi \in PX$, $\psi \in \overline{y_X(X)}$ if and only if $\psi$ is a right-adjoint.
	\end{coro}
\end{frame}

\begin{frame}{References}
\nocite{Fletcher_Lindgren_1982}\nocite{Riehl_2016}
        \printbibliography
\end{frame}

\appendix

\section{QUS YL}
\begin{frame}{Showing $\psi \geq \psi^*.(y_X^{})^*_{}$}
\begin{itemize}
\item<1-> By definition, $(y_X)_*=\tilde{A}.y_X$, and $\psi^*=\psi^o.\tilde{A}$, \\
so, we need to show that $\psi \geq (y_X)_*.\psi^* = \psi^o.\tilde{A}.\tilde{A}.y_X$. \\
As $\tilde{A}.\tilde{A}=\tilde{A}$, we need $\psi \geq \psi^o.\tilde{A}.y_X$.\\
\\
Fix $p \in \psi$. By definition of 2-cells in ProMod, we reqiure the following. 
\item<1->For this $p$ we need to find $a \in A$ such that $p \supseteq \psi^o \tilde{a} y_X^{}$.
\item<2->
Examining the right side of the condition, for any $a \in A$, $x \in X$
			\begin{equation*}
		\Big(\psi^o.\tilde{a} .y_X \Big) (x) = =
				\psi^o\big(\tilde{a}(x^*)\big)= \begin{cases}
					\phi &\text{ if } \psi \notin \tilde{a} (x^*) \\
					\star &\text{ if } \psi \in \tilde{a} (x^*)
				\end{cases}.\end{equation*}
				So, if $\psi \notin \tilde{a} (x^*)$, then, as $p \supseteq \phi$, 
				we're done. 
				Thus, let $\psi \in \tilde{a} (x^*)$, that is, $x^* \tilde{a} \psi$.
				
				\item <2-> Hence, $x \big( \psi^o.\tilde{a}. y_X^{} \big) \star$, we now need to show
				$ x p \star$
				\item<3-> As $\psi$ is a promodule, $\psi.A \leq \psi$ gives us the
				the existence of $q \in \psi$ and $a \in A$ such that $qa \subseteq p$. Using the definition of $\tilde{a}$, $x^* \tilde{a} \psi$ gives:
				\begin{equation*} x^o.A \leq \psi.a
					\implies \exists b\in A: x^ob \subseteq qa
					\implies \forall z \in X, \big(x^o b \big)(z) \subseteq (qa)(z)
				.\end{equation*}
				\item<4-> By quasi-uniformity of $A$, $b$ is reflexive, giving that
				in particular, for $xbx$,\begin{equation*} \big(x^o b \big)(x) \subseteq (qa)(x) \implies x^ox \subseteq (qa)(x) \implies \star \in (qa)(x) \implies \star \in p(x) \end{equation*}
\end{itemize}
\end{frame}
\begin{frame}{Showing $\psi \in \overline{y_X(X)} \implies 
\psi \leq \psi^*.(y_X^{})^*_{}$}

\begin{enumerate}
    \item <1-> By definition, we need to show that for any $a \in A$, $\exists p \in \psi: p \subseteq \psi^o.\tilde{a}.y_X$. \\ Fix any $x \in Dom(p)$, will show 
    $p(x) \subseteq  \psi^o.\tilde{a} .y_X(x)=\psi^o\big(\tilde{a} (x^*)\big)$
    \item <2-> So, need that $xp\star \implies \psi \in \tilde{a}(x^*)$
    Fix any $a \in A$, let's find $p \in \psi$ such that the above inequality holds.
    \item <3-> Quasi-uniformity of A gives that $\exists b \in A: bb \subseteq a$.
    As we know that $y_X^{}$ is uniformly continuous, from Yoneda Embedding in QUS,
    we have that $y_X.A \leq \tilde{A}.y_X $. So, $\exists c \in A$ such that 
    $y_xc \subseteq \tilde{b}y_X $. Thus, for any $z$ such that $zcw$,
    \begin{equation*} \big(y_X c \big)(z) \subseteq \big(\tilde{b} y_X\big)(z) \implies
					y_X(c(z)) \subseteq \tilde{b}(z^*) \implies w^* \in \tilde{b}(z^*) \text{ that is, } z^* \tilde{b} w^*\;.
				\end{equation*}
			\item<4-> As $A$ is a quasi-uniformity, $\exists d\in A: dd \subseteq c$. Also, because $A$ is a down-
				directed set, $\exists a' \in A: a' \subseteq b,d $. Thus, using the
				above chain of implications, for any $x,y \in X$,
				$ x(a'a')y \implies x(dd)y \implies xcy \implies x^* \tilde{b} y^* \; .$
				\item<5-> Now, because $\psi \in \overline{y_X(X)}$, we get
				$\exists x \in X \text{ such that } \psi \tilde{a'}x^* \text{ and } x^* \tilde{a'} \psi $.
				\item<6-> Using the definition of $\tilde{a'}$, from $\psi \tilde{a'}x^*$,
				we have $\psi \leq x^o.A.a'$, i.e. $\exists p \in \psi: p \subseteq x^o a'a'$.
				\item<7-> Fix any $z \in X$ such that $zp \star$. Using (6), definition of $x^o_{}$ and then
				(4), we get that we need to show: \begin{equation*} zp \star \implies z(x^o a'a') \star \implies z(a'a')x
				\implies z^* \tilde{b}x^*  \; . \end{equation*}
				\item<8-> Finally, as $\tilde{a'} \subseteq \tilde{b}$, From (5),
				we have that $z^* \tilde{b}x^* \text{ and } x^*\tilde{b}\psi  $. Using composition, $z^* \tilde{b} x^* $.
				
				
\end{enumerate}

\end{frame}

\section{QUS YE}
\begin{frame}{Showing $y_X^{}$ is uniformly continuous}
\begin{enumerate}

\item<1-> As $y_X^{}:(X,A) \rightarrow (PX,\tilde{A})$, we need to show $y_X^{}.A \leq \tilde{A}.y_X^{} $.
\item<2-> So, need that for each $a\in A$, there exists $b \in A$
		such that $y_X \circ b \subseteq \tilde{a} \circ y_X$.
\item<3->  Applying the relations
		to some element, $x$ of the set $X$ gives us:
		\begin{equation*} \big(y_X \circ b\big)(x) \subseteq \big( \tilde{a} \circ y_X\big)(x) \implies
		y_X(b(x)) \subseteq \tilde{a}(x^*). \end{equation*}
\item<4-> So, for the condition given above to hold, if $y \in b(x)$, then it's required that
		$y^*=y_X(y) \in \tilde{a} (x^*)$ i.e. $x^* \tilde{a}y^*$. Using the definition of $x^*,y^*$
		and $\tilde{a}$,
		\begin{equation*} x^* \tilde{a}y^* \iff x^o.A\leq y^o.A.a \iff
		\forall a' \in A, \exists a'' \in A: x^oa'' \subseteq y^oa'a  \end{equation*}
\item<5-> Fix any $a \in A$, $x\in X$. Thus, quasi-uniformity of A, gives $ \exists a'' \in A$ such that
		$a''a''\subseteq a$. Now, choose $y \in a''(x)$. Hence, in
		order to show that the condition above holds, need that
		$\forall b \in A, x^o a'' \subseteq y^oba$.
		Applying the relations to an element $z \in X$ gives the following condition:
		\begin{equation} \forall b \in B, \forall x \in X \text{ , }
		\big(x^oa''\big)(z) \subseteq \big(y^oba\big)(z). \end{equation}
		\item<6-> Examining the left side,
		\[ \big( x^oa''\big)(z)=x^o (a''(z))= \begin{cases}
			\phi &\text{ if } x \notin a''(z) \\
			\star & \text{ if } \in a''(z)
		\end{cases} .\]
\end{enumerate}
\end{frame}
\begin{frame}{Showing $y_X^{}$ is uniformly continuous}
\begin{enumerate}
  \setcounter{enumi}{6}
  \item<1-> Thus, to show that (1) holds, need to show that, for any $b\in A$ and $z \in X$:
		\begin{equation*} x \in a''(z) \implies z(y^oba)\star \text{ i.e. } y\in(ba)(z)
		\end{equation*}
		\item<2-> Fix any $z\in X: x \in a''(z)$. Also, by our choice of $y$,
		have that $y \in a''(x)$. Now, going to show that the above condition holds:
		As $A$ is a quasi-uniformity, $b$ is reflexive, giving that $y \in b(y)$.
		\item<3-> Hence, by composition of relations, we get:
		\[ za''x \text{ , }  xa''y \text{ and } yby \implies z(a''a''b)y \implies z(ab)y \text{ i.e. }
		y \in (ba)(z).\]
		\end{enumerate}
		\end{frame}
\begin{frame}{Showing $y_X^{}$ is full and faithful}
		\begin{itemize}
		\item<1-> By using the four-part proposition's first result, we just need to show that $A\geq y_X^o.\tilde{A}.y_X$.
		\item<2-> That amounts to showing $\forall
		a\in A, \exists \tilde{b}\in \tilde{A} :  a \supseteq y_X^o \text{ } \tilde{b} \text{ } y_X $.
		\item<3->
		Applying the relations to an element, $x\in X$ gives the condition:
		\begin{equation}
			\Big( y_X^o \text{ } \tilde{b} \text{ } y_X \Big)(x) \subseteq a(x)
			\implies \Big( y_X^o \text{ } \tilde{b} \Big) (x^*)= y_x^o
			\Big(\tilde{b}(x^*)\Big) \subseteq a(x).
		\end{equation}
		\item<4->
		Thus, if $y^* \in PX$ such that $x^* \tilde{b} y^*$, then
		$y \in y_x^o\Big(\tilde{b}(x^*)\Big)$. For the above condtition to hold, need that
		$y \in a(x)$, that is, $xay$.
		\item<5-> Thus,
		need only to show that for any $a\in A$, there exists $b\in A $ such that
		for any $x,y \in X$, $x^* \tilde{b}y^*$ implies $xay$.
		\item<6-> So, fix $a\in A$, and take $b \in A: bb \subseteq a$.
		Now, let $x^* \tilde{b}y^*$ i.e. $x^o.A \leq y^o .A .b$.
		
		\item<7-> Hence, $\exists c \in A: x^oc \subseteq y^o bb$. And as c is reflexive,
		\[ xcx \implies x(cx^o)\star \implies x(bby^o)\star \implies x(bb)y \implies xay\;. \qedhere \]
		\end{itemize}
\end{frame}

\section{Coro}
\begin{frame}[allowframebreaks]{Proof: Yoneda Corollary}
\setcounter{equation}{0}
		Fix any $\psi \in PX$. \setcounter{equation}{0}
		\begin{enumerate}
			\item ($\implies$)
				Let $\psi \in \overline{y_X(X)}$, from Theorem 5.4, we get that
				$\psi = \psi^*.(y_X^{})_*^{}$. In order to show $\psi$ is a
				right-adjoint, we will show that $\psi^*$ is a
				right adjoint and that $(y_X^{})_*^{}$ is an equivalence.
				\begin{enumerate}
					\item In order to show that $(y_X^{})_*^{}$ is an equivalence,
						we need that $A=(y_X^{})^*_{}.(y_X^{})_*$ and
						$\tilde{A}= (y_X^{})_*.(y_X^{})^*_{}$.\\
						From proposition 5.3 (b), we have that $y_X^{}$
						is fully faithful, and by Proposition 4.14 (a), this gives us
						that $A=(y_X^{})^*_{}.(y_X^{})_*$.
						\begin{itemize}
							\item 	We are now going to show that
								$\tilde{A} \leq (y_X^{})_*.(y_X^{})^*$.
								Fix any $a,b\in A$, we need to find $c\in A$
								such that $\tilde{c} \subseteq \tilde{a}\,
								y_X^{} \, y_X^{o} \, \tilde{b} $.
								\[\big(\tilde{a}\, y_X^{}.y_X^{o}\, \tilde{b}\big) (\psi)
									=\big(\tilde{a}\,\tilde{b}\big) (\psi)
									\supseteq \tilde{c} \tilde{c} (\psi)
								\supseteq \tilde{c}(\psi) \]
								In the above equation, the equality holds because $\psi \in \overline{y_X^{}(X)}$, gives
								the existence of $x^*=\tilde{b}(\psi)$. And the first inequality is given by down-directedness of
								$\tilde{A}$, whereas the second one holds because $\tilde{c}$ is reflexive, as $\tilde{A}$ is a
								quasi-uniformity.
							\item To show that $\tilde{A} \geq (y_X^{})_*.(y_X^{})^*$, fix any $a\in A$. By
								quasi-uniformity of $\tilde{A}$, there exists $\tilde{b}\in \tilde{A}$ such that
								$\tilde{b} \, \tilde{b} \subseteq a$. We will show that $\tilde{a} \supseteq
								\tilde{b}\,y_X^{} \, y_X^{o} \, \tilde{b} $:
								\[ \psi \big(\tilde{b}\,y_X^{} \, y_X^{o} \, \tilde{b} \big) \phi
								\implies \psi \big( \tilde{b} \tilde{b} \big) \phi \implies \psi \tilde{a} \phi \; .\]


						\end{itemize}

					\item In order to show that
						$\psi^*$ is a right adjoint to $\psi_*$,
						due to the 2-categorical structure of ProMod,
						we need to show that
						$\tilde{A} \geq \psi_\star.\psi^\star $ and
						$\psi_\star.\psi^\star \geq 1$.
						\begin{itemize}
	\item To show that $\tilde{A}\geq \psi_*.\psi^*=\psi_*.\psi^o.\tilde{A} $,
								fix any $a\in A$. We will show that $\psi_*.\psi^o.\tilde{a} \subseteq \tilde{a}$.
								Using definition of $\psi_*$, for any $\phi \in \overline{y_X^{}(X)}$, we get:
								\[\big(\psi_*.\psi^o.\tilde{a}\big)(\phi)=
									\psi_*.\psi^o(\tilde{a}(\phi))= \begin{cases}
										\phi &\text{ if }\tilde{a} (\phi)\neq \psi \\
										\psi=\psi_*.\psi^o(\psi) &\text{ if } \tilde{a} (\phi)=\psi
									\end{cases}.\]
									The above equation gives that $\phi\big(\psi_*.\psi^o.\tilde{a}\big)\psi$ implies
									$\phi \tilde{a} \psi$.
									Hence,
									we have that
									$\tilde{a}\supseteq
									\psi_*.\psi^o.\tilde{a} $.
								\item We will show that $\psi_\star.\psi^\star \geq 1$, that is
									$\star(\psi^o.\tilde{a} .\psi_*)\star$. Using definition of $\psi_*$,
									\[\big(\psi^o.\tilde{a} .\psi_*\big)(\star)=\big(\psi^o.\tilde{a}\big) (\psi_*(\star))
									= \big( \psi^o.\tilde{a} \big) (\psi)=\psi^o \big( \tilde{a}(\psi)  \big). \]
									By the quasi-uniformity of $\tilde{A}$, we get that $\tilde{a}$ is reflexive, and hence,
									$\psi \tilde{a} \psi$. So, from the above equation, we have that
									$\star \in \psi^o(\psi) \subseteq \big(\psi^o.\tilde{a} .\psi_*\big)(\star) $.
							\end{itemize}
					\end{enumerate}

				\item ($\impliedby$) Suppose $\psi$ is a right adjoint. Need to show that for any
					$a \in A$, $\exists x^* \in y_X(X)$ such that $\psi \, \tilde{a} \,x^*
					\tilde{a} \, \psi$. Fix $a \in A$. Because $\psi$ is a right-adjoint, there
					exists a promodule $\phi: 1 \to X$ such that $\phi.\psi \leq A$ and
					$1\leq \psi.\phi$. From $\phi.\psi \leq A$, we get that:
					\begin{equation}\exists p \in \phi, q \in \psi \text{ such that }
						a \supseteq p.q \; .
					\end{equation}
					Because $\phi$ and $\psi$ are promodules,
					\begin{align}
						A.\phi \leq \phi & \text{ gives the existence of } p' \in \phi \text{ such that } p\supseteq a'p'  \; ,\\
						A.\psi \leq \psi & \text{ gives the existence of } q' \in \psi \text{ and } a'' \in A
						\text{ such that } q \supseteq a''q' \; .
					\end{align}
					Now, from $1 \leq \psi.\phi$, we get that $q'\,p'$ is reflexive i.e. $\star (q' \,p')
					\star$. By the definition of composition we get the existence of an $x \in X$
					such that $\star \, p' \, x \,q'\,\star$. Now, considering $x$ as a map, $x:1 \to X$
					defined as $\star \mapsto x$,
					\begin{align}
						x \, q' \, \star \text{ i.e. } &\star \in q'(x) \text{ gives that }
						q' \supseteq x^o \;,\\
						\star \,p' \, x \text{ i.e. } &x \in p'(\star) \text{ gives that }
						p'\supseteq x \;.
					\end{align}
					Thus, by using inequalities (1),(2) and (3), we get:
					\begin{equation}
						a \supseteq p\,q \supseteq a'\,p'\,q'\,a'' \; .
					\end{equation}
					By definition of $\tilde{a}$, to show $\psi \, \tilde{a} \, x^*$,
					we need that $\psi \leq x^*\,a=x^o.\,A.\,a$. We are now
					going to show that
					$\text{ for any } b\in A$, $\;  x^o\,b\,a \supseteq q'$:
					\[ x^o\,b\,a \supseteq x^o\,b\,a'\,p'\,q' \supseteq x^o\,b\,a'\,x\,q'
					\supseteq x^o\,x\,q' = q' \; .\]
					Where the first inequality comes from (6) by using reflexiveness of $a''$ and then
					left-multiplying by $x^o$. The second inequality comes from (5),
					third one from reflexiveness of $b$ and $a'$,
					and the last one is given by Lemma 3.7.

					In order to show $x^* \, \tilde{a} \, \psi$, by definition of $\tilde{a}$,
					need that $x^o.A=x^* \leq \psi\,a$. Fix $k\in \psi$. We will now show
					$k\,a \supseteq x^o \, a''$:

					\begin{equation}
						a\supseteq a'\,p'\,q'\,a'' \supseteq p'\,q'\,a''
						\supseteq p' \,x^o \,a'' \; .
					\end{equation}

					Where the first inequality is given by (6), second one
					is due to reflexiveness	of $a'$ and the third inequality comes by using (4).
					Left-multiplying (7) with $k$ gives the following:
					\begin{equation}
						ka \supseteq k\,p'\,x^o\,a'' \text{ that is, for any } z\in X,
						\; \; z (k\, a) \star \implies z ( k\,p'\,x^o\,a'') \star \; .
					\end{equation}
					As $\psi$ is a right adjoint to $\phi$, we have $1\leq \psi.\phi$, giving that
					$\star (k \, p' )\star $. So, using the implication in(8), we get that
					$z(k\, a) \star $ implies $z ( x^o\,a'') \star (k\,p') \star$, which in turn
					gives that $z(x^o \, a'')\star$. Hence, we get that $ka \supseteq x^o\, a''$
					\qedhere
			\end{enumerate}
\end{frame}

\section{YL}
\setcounter{equation}{0}
\begin{frame}[fragile]{Showing isomorphism between $[\mathcal{A}^{op},Set](H_A,X)$
					and $X(A)$}
					Let $\mathcal{A} $ be a locally small category.
			Fix an object $A \in \mathcal{A} $ and a presheaf $X$ on $\mathcal{A}$.
					\begin{itemize}
\item<1-> Define $\string ^ :\mathcal{C}(H_A,X) \to X(A) $
					for any $\alpha:H_A \to X,$ as  $\hat{\alpha}:= \alpha_A(1_A)$.
					As $1_A \in Set(A,A)=H_A(A)$,
					definition of $\alpha_A$ gives that $\alpha_A(1_A)\in X(A)$.
\item<2-> Define $\string ~ : X(A) \to \mathcal{C}(H_A,X)$
					for any $ x \in X(A)$ as the natural transformation $\tilde{x} : H_A \to X$ whose
					K-component is the function mapping each morphism $p \in \mathcal{A}(K,A)$
					to $\Big(X(p)\Big)(x)$. That is, $\tilde{x}_K^{}
					(p):=\Big(X(p)\Big)(x)$.
\item<3-> We are going to show that $\tilde{x}$ is a natural transformation.
\item<4->Fix objects $K,L \in \mathcal{A} $ and morphism $q \in \mathcal{A}^{op}(K,L)$.
					$\text{Need to show that the square }
					\begin{tikzcd}
						\mathcal{A}(K,A) \arrow[swap]{d}{\tilde{x}_K} \arrow{r}{ - \circ q}
		& \mathcal{A} (L,A) \arrow{d}{\tilde{x}_L}\\
		X(K) \arrow[swap]{r}{X(q)}
		& X(L)
					\end{tikzcd}\text{ commutes }$.\\

					So, for any $f:K\to A$, need that $\tilde{x}_L(f \circ q)= X(q) \circ \tilde{x}_K(f)$. Using
					the definition of $\tilde{x} $ gives the following.
					\begin{align*}
						LHS &=\tilde{x}_L(f \circ q ) =\Big( X(f \circ q)\Big)(x) \\
						RHS &=X(q) \circ \tilde{x}_K(f) =\Big(X(q)\Big) \big(X(f)(x)\big)=\Big(X(q) \circ X(f)\Big) (x)
					\end{align*}
					And as $X$ is a contravariant functor, $X(f \circ q)= X(q) \circ X(f)$,
					giving that LHS=RHS.
					\end{itemize}
\end{frame}
\begin{frame}[fragile]{Showing isomorphism between $[\mathcal{A}^{op},Set](H_A,X)$
					and $X(A)$}
					\begin{itemize}
					\item<1-> Need to show that $\string ^ $ and $\string ~ $ are mutually inverse.
					\item<2->For any $x \in X(A)$,
							$\hat{\tilde{x}}=\tilde{x}_A (1_A)=\Big(X(1_A) \Big) (x)=1_{X(A)}(x)=x$.
						For any $\alpha \in \mathcal{C} (H_A,X)$ , need to show that $\tilde{\hat{\alpha}}=\alpha$.
							So, it's required that each of their component are equal.
\item<3-> As both $\tilde{\hat{\alpha}}$ and $\alpha$ are natural transformations
							between functors that go to the category \textit{Set}, each of their components is a function.
							So, need to show that for any $f \in \mathcal{A} (K,A)=H_A(K)$,
							$\Big(\tilde{\hat{\alpha}}\Big)_K(f)$=$\alpha_K(f)$.
\item<4-> Using first the definition of $ \string ~$ and then that of $\hat{\alpha}$ gives:
							\begin{align}
								LHS=\tilde{\hat{\alpha}}_K(f)=\Big(X(f)\Big)(\hat{\alpha})=\Big(X(f)\Big)(\alpha_A(1_A))
							\end{align}
							And as $f\in \mathcal{A} (K,A)$, we also have the following.
							\begin{equation}
								RHS=\alpha_K(f)= \alpha_K(1_A \circ f)
							\end{equation}
\item<5-> Because $\alpha$ is a natural transformation, the following square commutes for $1_A$:
							\begin{equation*}\begin{tikzcd}
								\mathcal{A} (A,A) \arrow[swap]{d}{\alpha_A} \arrow{r}{- \circ f}
	& \mathcal{A} (K,A) \arrow{d}{\alpha_K}\\
	X(A) \arrow[swap]{r}{X(f)}
	& X(K)
							\end{tikzcd},\end{equation*}
							which gives that $\alpha_K(1_A \circ f)=\Big(X(f) \Big) \big( \alpha_A (1_A) \big)$.
							Hence, we get that $RHS=LHS$, giving us that $\string ^$ and $\string ~$ define a
							a set isomorphism, as $\alpha_K$ being a 
							function, $RHS$ is a set.
					\end{itemize}
\end{frame}

\begin{frame}[fragile]{Showing natural in $X$}
\begin{enumerate}
\item<1-> By using the very first two Lemmas, it's enough to show that $\string ^$	is natural in $X$ and natural in $A$.
\item<2->
							Fix any $A \in \mathcal{A}$. Need, for
							presheaves $X,Y \in \mathcal{C}$ and
							natural transformation $\beta \in \mathcal{C} (X,Y)$, the following square to commute:
							\begin{equation*}
								\begin{tikzcd}
									\mathcal{C}(H_A,X) \arrow[swap]{d}{ \string ^}
									\arrow{r}{\beta \circ -}
			& \mathcal{C}(H_A,Y) \arrow{d}{\string ^}\\
			X(A) \arrow[swap]{r}{\beta_A}
			& Y(A)
								\end{tikzcd}.
							\end{equation*}
							\item<3->
							So, for any $\alpha:H_A \to X$, we need that
							$\big( \string ^ \circ (\beta \circ \string _ )\big)(\alpha) = \big( \beta_A \circ \string ^
							\big)(\alpha)$.
\item<4-> Using the definition of
							$(\beta \circ \string _ )$ and $\string ^$ gives:
							\begin{align*}
								LHS & = \big( \string ^ \circ (\beta \circ \string _ ) \big)(\alpha) =
								\widehat{\big(  (\beta \circ \string _ )(\alpha)\big)}=
								\widehat{\big(  \beta \circ \alpha\big)}
								=\big(  \beta \circ \alpha\big)_A(1_A) \\
								RHS & = \big( \beta_A \circ \string ^ \big)(\alpha)
								= \beta_A (\widehat{\alpha}) =\big( \beta_A \circ \alpha_A\big)(1_A)
							\end{align*}
							As $\alpha \in \mathcal{C} (H_A,X)$ and $\beta \in \mathcal{C}(X,Y)$
							are morphisms in $\mathcal{C}$, composition in $\mathcal{C}$ gives
							$(\beta \circ \alpha)_A = \beta_A \circ \alpha_A$.
\end{enumerate}
\end{frame}

\begin{frame}[fragile]{Showing natural in $A$}
\begin{enumerate}
\item<1-> Fix any $X \in \mathcal{C} $ Need that
							for objects $A,B \in \mathcal{A} $ and morphism $f\in \mathcal{A} ^{op}(A,B)$,
							the following square commutes:
							\begin{equation*}
								\begin{tikzcd}
									\mathcal{C}(H_A,X) \arrow[swap]{d}{ \string ^}
									\arrow{r}{- \circ H_f}
			& \mathcal{C}(H_B,Y) \arrow{d}{\string ^}\\
			X(A) \arrow[swap]{r}{X(f)}
			& X(B)
								\end{tikzcd},
							\end{equation*}
							Where $H_f$ denotes
							$(f \circ \string _ )$. So, for any $\alpha:H_A \to X$, we need that
							$\big( \string ^ \circ H_f \big)(\alpha)=\Big((X(f)) \circ \string ^ \Big)(\alpha) $.
\item<2-> Using definition of $H_f$ and $\string ^$, we get:
							\begin{align*}
								LHS&=\big( \string ^ \circ H_f \big)(\alpha)= \widehat{\alpha \circ H_f}
								=(\alpha \circ H_f)_B (1_B) = \alpha_B (f \circ 1_B) = \alpha_B(1_A \circ f) \\
								RHS&=\Big((X(f)) \circ \string ^ \Big)(\alpha)
								= (X(f))(\hat{\alpha})= \Big(X(f) \Big)\big(\alpha_A(1_A)\big)
							\end{align*}
\item<3-> By using equality of equations (1) and (2), for
							$f \in \mathcal{A}(B,A)$, we get that
							$\Big(X(f) \Big)\big(\alpha_A(1_A)\big)=\alpha_B( 1_A \circ f) $.
							Hence, $RHS=LHS$. \qedhere
\end{enumerate}
\end{frame}

\section{CT}
\begin{frame}[fragile]{Natural transformations from $H_\star$
					to $H_\star$ are bijections on G.}
					\begin{itemize}
					    \item<1-> Let $G$ be a group. Define category $\mathcal{A}$ with a single object $\star$. And let the morphisms of $\mathcal{A}$ be elements of $G$. Then, $G$ and $\mathcal{A}(\star, \star)$ have the
			same elements
			and rule of composition, so there exists a group isomorphism $\psi:
			\mathcal{A} (\star,\star) \to G$.
\item<2-> As $\mathcal{A} ^{op}$ is a category with a single object,
					each natural transformation $\alpha:
					\begin{tikzcd}[row sep=small, column sep=large]
						\mathcal{A} ^{op} \arrow[r,bend left=50, "H_\star"{name=U, below}]
						\arrow[r, bend right = 50, "H_\star"{name=D, above}]
						& Set
						\arrow[Rightarrow, from=U , to=D, "\alpha"]
					\end{tikzcd}$
					has only one component, that is $\alpha_\star$. Therefore, we can
					identify $\alpha$ with $\alpha_\star$.
\item<3-> Using naturality of $\alpha$, we get that
					\begin{equation*}
						\text{ the square }
						\begin{tikzcd}
							\mathcal{A} (\star,\star) \arrow[swap]{d}{\alpha_\star} \arrow{r}{\string _ \circ f}
			& \mathcal{A} (\star,\star) \arrow{d}{\alpha_\star}\\
			\mathcal{A} (\star,\star) \arrow[swap]{r}{ \string _ \circ f}
			& \mathcal{A} (\star,\star)
						\end{tikzcd}
						\text{ commutes for any } f \in \mathcal{A}(\star,\star).
					\end{equation*}
					
\item<4-> Applying the identity of $\star$ in $\mathcal{A}$ in above square
					gives us the following equation:
					\begin{equation*}
						\big( (\string _ \circ f) \circ \alpha_\star \big)(1_\star) =
						\big(  \alpha_\star \circ (\string _ \circ f)\big)(1_\star)
						\implies \alpha_\star(f)=\alpha_\star(1_\star) \circ f
						\implies \alpha_\star(f)=\alpha_\star(1_\star).f
					\end{equation*}
					
		Thus, every natural transformation $\alpha$ is defined in
					terms of its value at $1_\star$.
					This can be considered as left multiplication by
					$\alpha_\star(1_\star)$ in $G$, which
					we know is a bijection on $G$.
	\item<5-> So far we have shown that
					the collection $[\mathcal{A} ^{op},Set](H_\star,H_\star)$ of
					all $\alpha:H_\star \to H_\star$ is a
					collection of bijections on $G$.
    \end{itemize}
\end{frame}

\begin{frame}{The collection $[\mathcal{A}^{op},Set](H_\star,H_\star)$ is a group.}
					\begin{itemize}
\item<1-> We will show that the collection
					$[\mathcal{A} ^{op},Set](H_\star, H_\star)$
					is a group with respect to composition in the category
					$[\mathcal{A} ^{op},Set]$.
					\item<2->
					As $[\mathcal{A} ^{op},Set]$ is a category, we have that
					the composition is associative. Also, because this collection contains
					morphisms with the same source and destination,
					it is closed under composition. Identity of
					$[ \mathcal{A} ^{op},Set ] (H_\star,H_\star)$ will act as
					the identity for its group structure.
\item<3->
					We will now show closure under inverses.
					Fix any $\gamma:H_\star \to H_\star$.
					
					\item<4-> Since $\gamma_\star(1_\star)$ belongs to
					$ \mathcal{A} (\star,\star)$,
					let us call $\psi(\gamma_\star(1_\star))=h \in G$. Thus,
					there exists $h^{-1}\in G$. 
					
					\item<5-> As $\psi$ is onto,
					there exists $a\in \mathcal{A} (\star,\star)$ such that
					$\psi(a)=h^{-1}$.
					\item<6-> We know that any natural transformation $\alpha$
					is defined in terms of $\alpha_\star(1_\star) \in
					\mathcal{A}(\star,\star)$. Thus, we define
					$\delta:H_\star \to H_\star$ with $\delta_\star(1_\star)=a$.
					Giving us that
					$h^{-1}=\psi \big(\delta_\star(1_\star) \big) $.
						And as $\psi$ is a group isomorphism,
					\[ 1_\star = \psi^{-1}(h.h^{-1})=\psi^{-1}(h)\,. \psi^{-1}(h^{-1})
					=\big(\gamma_\star(1_\star)\big).\big(\delta_\star(1_\star)\big).\]
					\item<7->
This gives us that $\delta$ and $\gamma$ are inverses, because 
for any $ k \in \mathcal{A} (\star,\star)$, \[ 
						\big( \gamma \circ \delta\big)_\star
						(k)
						= \gamma_\star\big(\delta_\star(k)\big)
						=\gamma_\star \big( \delta_\star(1_\star).k \big)
						=\big(\gamma_\star(1_\star)\big).
						\big( \delta_\star(1_\star) \big).k
					=1_\star.k=k\]
				\item<8->
					Thus, the collection $[ \mathcal{A} ^{op}, Set](H_\star,H_\star)$ is a group.
    \end{itemize}
\end{frame}

\begin{frame}{Applying Yoneda Lemma, and showing group isomorphism}
					\begin{itemize}
\item<1-> As the collection of elements of $G$ form a set, $\mathcal{A}(\star,\star)$
					is also a set. Hence, $\mathcal{A}$ is a locally small category. Because
					$\mathcal{A} ^{op}$ has the same number of morphisms as $\mathcal{A}$,
					it is also a locally small category, and we may apply Yoneda Lemma to it.
\item<2-> Taking $A=\star$ and $X=H_\star$, we get:
					\begin{equation*} [ \mathcal{A} ^{op}, Set](H_\star,H_\star)
						\; \hat{\cong} \; H_\star(\star), \end{equation*}
					where the isomorphism $\string ^$ is between sets.
\item<3-> From the proof of Yoneda Lemma, we know that the map
					$\string ^$ acts as $\alpha \mapsto \alpha_\star (1_\star)$.
					Hence, for any $\alpha,\beta:H_\star \to H_\star$,
					\begin{equation*}
						\widehat {\alpha \circ \beta}= (\alpha \circ \beta)_\star(1_\star)
						=(\alpha)_\star \Big( (\beta)_\star (1_\star) \Big)=
						\Big((\alpha)_\star(1_\star) \Big) . \Big( (\beta)_\star (1_\star) \Big)
						=\hat{\alpha} . \hat{\beta}\;,
					\end{equation*}
\item<4-> Using I and II, we get that $[ \mathcal{A} ^{op}, Set](H_\star,H_\star)$
			is a group with all of it's elements being bijections on $G$.
			Thus, it is a subgroup of the symmetric group on G.
\item<5-> Finally, the isomorphism $\string ^$ is between groups,
			with the $LHS$ being the above mentioned subgroup.
			And $RHS$ being $\mathcal{A} (\star, \star)$, which is further
			isomorphic to group $G$:
		\[ G \; \overset{\psi}{\cong} \; \mathcal{A} (\star,\star) \; \hat{\cong} \;
			[ \mathcal{A} ^{op}, Set](H_\star,H_\star)\, \leq Sym(G).\]
			This is	precisely the statement of Cayley's theorem.\qedhere
\end{itemize}
\end{frame}

\section{YE}
\setcounter{equation}{0}
\begin{frame}[fragile]{Functor $H_\bullet$ is full}
    We will show that the functor $H_\bullet$ is
				full and faithful. Fix any objects $X,Y$ in a locally small category $\mathcal{A}$.
    \begin{itemize}
        \item<1-> Fix any objects $X,Y$ in a locally small category $\mathcal{A}$.
To show that $H_\bullet$ is a full, we need to show that for every $\alpha
						\in [\mathcal{A} ^{op},Set](H_X,H_Y)$, there exists a morphism
						$f\in \mathcal{A} (X,Y)$ such that $H_\bullet(f)=\alpha$.
\item<2-> Thus, we need to show that their $K$-components are equal for every $K \in \mathcal{A}$.
Using the definition of $H_\bullet(f)$, this amounts to showing that
						\begin{equation*}  \text{ for any morphism } k \in H_X(K),
							\Big(H_\bullet(f)\Big)_K^{}(k)=\alpha_K(k), \text{ that is }
							f\circ k=\alpha_K(k).
						\end{equation*}
						
\item<3->	Because $\alpha_X$ goes from $H_X(X)$ to $H_Y(X)$, $\alpha_X(1_X)$ is a morphism
						in $\mathcal{A}(X,Y)$. 
\item<4-> We will show that choosing this morphism to be $f$
						will give us the required result, that is $\big( \alpha_X(1_X) \big) \circ
						k=\alpha_K(k)$. 
						
\item<5-> Using the naturality of $\alpha$,
						$ \text{ we get that }
							\begin{tikzcd}
								\mathcal{A} (X,X) \arrow[swap]{d}{\alpha_X} \arrow{r}{\string _ \circ k}
			& \mathcal{A} (K,X) \arrow{d}{\alpha_K}\\
			\mathcal{A} (X,Y) \arrow[swap]{r}{\string _ \circ k}
			& \mathcal{A} (K,Y)
							\end{tikzcd}
						\text{ commutes. }$
						\item<6->
						Thus, for the identity morphism $1_X \in \mathcal{A} (X,X)$, we get the following:
						\[ \Big(H_Y(k) \circ \alpha_X\Big)(1_X)=\Big(\alpha_K \circ H_X(k)\Big)(1_X)
							 \implies
						\alpha_K(1_X) \circ k=\alpha_K(k).\]
\item<6-> Thus, we have that $H_\bullet$ is a full functor.
    \end{itemize}
\end{frame}
\begin{frame}[fragile]{Functor $H_\bullet$ is faithful}
    \begin{itemize}
        \item<1-> Fix any morphisms $f,g$ in $\mathcal{A} (X,Y)$ and suppose	$H_\bullet(f)=H_\bullet(g)$.In order to show $H_\bullet$ is faithful,
						we need to show that $f=g$. 
		\item<2-> As $H_\bullet(f)$ and $H_\bullet(g)$ are equal natural transformations, we have that the action of their $X$-components is equal. 
		\item<3-> Thus,
						in particular, for the identity of $X$, $\big(H_\bullet(f) \big)_X (1_X)
						= \big(H_\bullet(g) \big)_X (1_X)$. 
		\item<4-> Using the definition of $H_\bullet$, we get
						that $f \circ 1_X = g \circ 1_X$. And as both $g$ and $f$ are morphisms
						from $X$, we get that $f=g$.
    \end{itemize}
\end{frame}


\section{ProMod}
\setcounter{equation}{0}
\begin{frame}[allowframebreaks]{Proof: Promod is a 2-category}
\setcounter{equation}{0}
				In order to show that ProMod is a 2-category, need the following:
				\begin{enumerate}
					\item (1-Identities) For each quasi-uniform space $(X,A)$,
						$A:(X,A) \to (X,A)$ a promodule because $A.A=A$ by Lemma 4.5.
					\item (1-Composition) Need composition of promodules to be a promodule.\\
						Let $\phi:(X,A)\to (Y,B)$ and $\psi:(Y,B)\to (Z,C)$ be promodules.
						To show that $\psi.\phi:(X,A) \to (Z,C)$ is a promodule, need it to be a
						prorelation that satisfies the two conditions required to be a promodule:
						\begin{enumerate}
							\item By Lemma 3.4, prorelations are closed under composition.
								Hence, $\psi.\phi$ is a prorelation
							\item Need to show that $\psi.\phi.A \leq \psi.\phi$. So, Fix
								$p \in \psi$ and $q \in \phi$. As $\phi$ is a promodule,
								$\phi.A\leq \phi$ gives that there exists
								$ q' \in \phi \text{ and } a\in A$ such that
								$q'\,a \subseteq q$. Thus,
								$p\,q'\,a \subseteq p\,q$.
							\item Need to show that $C.\psi.\phi \leq \psi.\phi$.Fix $p \in \psi$
								and $q\in \phi$. Because
								$\psi$ is a promodule, $C.\psi \leq \psi$ gives that
								there exists $c\in C$ and $p' \in \psi$ such that
								$c\,p' \subseteq p$. Thus, $c\,p'\,q \subseteq p\,q$
						\end{enumerate}
					\item (2-Identities) As every promodule is contained in itself, always have $\psi \leq \psi$.
						Define this comparison to be the identity 2-cell for $\psi$ and denote it by $\leq_\psi$

					\item (Vertical 2-composition) For promodules $\psi,\phi,\delta:(X,A) \to (Y,B)$,
						if there is is a 2-cell from $\psi$ to $\phi$ and another one from $\phi$
						to $\delta$ i.e. $\psi \leq \phi \leq \delta$, then by transitivity
						of the partial order, $\psi \leq \delta$ i.e. there's a 2-cell from
						$\psi$ to $\delta$.
					\item (Horizontal 2-composition) If there are promodules
						$\psi,\psi':(X,A) \to (Y,B)$ and
						$\phi,\phi':(Y,B) \to (Z,C)$ such that $\psi \leq \psi'$ and $\phi \leq
						\phi'$, need to show that $\psi.\phi \leq \psi'.\phi'$. Fix $p' \in \psi'$ and
						$q' \in \phi'$. As $\psi\leq\psi'$, $\exists p\in \psi: p \subseteq p'$
						and as $\psi\leq\psi'$, $\exists q \in \phi: q \subseteq q'$.
						Thus, $p\,q\subseteq p'\,q'$
					\item (1-Identity) Need to show that for any promodule $\phi:(X,A) \to (Y,B)$,
						$\phi.A=\phi=B.\phi$. By quasi-uniformity of $A$, every $a \in A$, is
						reflexive. Thus, for any $p \in \phi$ and $a \in A$,
						$p=p. \Delta_X \subseteq p \,a$
						giving that $\phi \leq \phi.A$. And as $\phi$ is a promodule,
						$\phi \geq \phi.A$. Hence, by anti-symmetry of the partial order, $\phi=\phi.A$.

						Similarly, By quasi-uniformity of $B$, every $b \in B$, is
						reflexive. Thus, for any $p \in \phi$ and $b\in B$, $p=\Delta_Y.p\subseteq
						b\,p$ giving that $\phi \leq B.\phi$. And as $\phi$ is a promodule,
						$\phi \geq B.\phi$. Hence, $\phi=B.\phi$.
					\item (1-Associativity) As composition of relations is associative, so too is the
						composition of prorelations directly giving that composition of promodules
						i.e. 1-cells is associative.
					\item (Vertical 2-Identity) Let $\leq:\psi \to \phi$ be a 2-cell i.e.
						$\psi \leq \phi$. By our definition of identity 2-cell, $\leq_\psi.\leq_1$
						means precisely that $\psi \leq \psi \leq \phi $, and by transitivity,
						this is equivalent to $\psi \leq \phi$.
						Similarly, $\leq_1.\leq_\phi$ means exactly that $\psi
						\leq \phi \leq \phi $, and this is equivalent to $\psi \leq \phi$.
					\item (Vertical 2-Associativity) Associativity of the partial order on promodules
						directly gives the associativity of composition of 2-cells in ProMod.
					\item (Horizontal 2-Identity) Let $\psi,\phi: (X,A)\to (Y,B)$ be promodules.
						For any 2-cell $\leq:\psi \to \phi $, need to show that the 2-cell given
						by the horizontal composition, $\leq*\leq_A$ is equal to $\leq$, as well as
						equal to $\leq_B *\leq$. So, it's required that $\psi.A \leq \phi.A
						\iff \psi \leq \phi \iff B.\psi \leq B.\phi$. And this holds as a
						direct consequence of (f).
					\item (Horizontal 2-Associativity) As there's a unique 2-cell between any two
						promodules, and composition of promodules is associative,
						horizontal composition of 2-cells is associative.
					\item (2-Identity) For promodules $\psi:(X,A)\to (Y,B)$ and
						$:\phi(Y,B) \to (Z,C)$ need $(\leq_\psi * \leq_\phi)=\leq_{\psi.\phi}$.
						Both sides of the required equality are 2-cells $\leq:\psi.\phi
						\to \psi.\phi$.	Thus, they are equal by the uniqueness of 2-cells between
						any two 1-cells.
					\item (2-Interchange) Let $\psi,\phi,\delta:(X,A) \to (Y,B)$ and
						$\psi',\phi',\delta':(Y,B) \to (Z,C)$ be promodules. For
						2-cells $\leq_1:\psi \to \phi $,$\leq_2:\phi \to \delta $,
						$\leq_a:\psi' \to \phi' $ and $\leq_b:\phi' \to \delta'$,need to show
						$(\leq_b.\leq_a)*(\leq_2.\leq_1)=(\leq_b*\leq_2).(\leq_a*\leq_1)$.
						Both RHS and LHS are 2-cells from $\psi.\psi'$ to $\delta.\delta'$ and are
						hence equal. \qedhere
				\end{enumerate}
\end{frame}

\section{QUnif}
\setcounter{equation}{0}
\begin{frame}[allowframebreaks]{Proof: QUnif is a category}
\setcounter{equation}{0}
				\begin{enumerate}
					\item (Associativity) The composition of functions is associative by definition.
					\item (Identity) For each object $(X,A)$, the identity function
						$\Delta_X:(X,A) \to (X,A)$ is uniformly continuous as
						$\Delta_X.A=A \leq A=A.\Delta_X$.
						\qedhere
				\end{enumerate}
\end{frame}

\section{CoFunc}
\setcounter{equation}{0}
\begin{frame}[allowframebreaks]{Proof: Covariant Functor}
\begin{enumerate}
					\item (Partial-Order) Inclusion of relations acts as the partial order.
					\item (Down-Directed) Fix any $k,k'$ belonging to $B.f$. Thus, there exist
						$b,b'$ in $B$ such that $k=b\,f$ and $k'=b\,f$. Using down-directedness
						of $B$, there exists $c \in B$ such that $c \subseteq b,b'$. Hence,
						by Lemma 3.10, $c\,f \subseteq k,k'$.
					\item (Up-set) Let $k$ belong to $B.f$ and $l:(X,A) \to (Y,B)$ be a
						uniformly continuous function such that $l \supseteq k$.
						Define a relation $b'= \{(f(d),l(d)): d \in Dom(l) \}$.
						By definition, for any $x \in X$ and $z \in Y$ such that $(x,z) \in l$, we get that
						$(f(x),z)\in b'$. And $l \supseteq k=b\, f$ implies $Dom(l)  \supseteq Dom(f)$,
						giving $(x,f(x)) \in f$. Therefore, by definition of composition,
						$(x,z) \in b'.f$. Conversely, suppose $(x,z) \in b'.f$.
						By definition of composition, there exists $f(x)\in Y$ such that
						$(f(x),z) \in b'$. Again using the definition of
						$b'$, we get that $z=l(x)$ i.e. $(x,z)\in l$. Hence, $l=b'\, f$.
						Now we will show that $b'\supseteq b$. Because $b'\, f=l \supseteq k=b\, f$, for any
						$x\in X$ we have that $b'\big(f(x)\big) \supseteq b\big(f(x)\big)$. Thus,
						$b'|_{f(x)} \supseteq b|_{f(x)}$. By down-directedness of $B$,
						$b|_{f(x)} \subset b$ implies $ b(x)|_{f(x)} \in B$.
						Finally, $b' \supseteq b'|_{f(x)} \supseteq b|_{f(x)}$ gives $b' \in B$.
						Hence, $b'.f \in B.f$.
					\item Need to show that $(B.f).A \leq B.f$. So, fix any $b\in B$, we will find
						$b' \in B$ and $a\in A$ such that $b' \, f\,a \subseteq bf$.
						By quasi-uniformity of $B$, there exists $b' \in B$ such that $b'\,b'
						\subseteq b$. Using Lemma 3.10, we get that $b'\,b'\,f \subseteq b\,f$.
						As $f$ is uniformly continuous, $f.A \leq B.f$ gives that there is some
						$a \in A \text{ such that } f\,a \subseteq b'\,f$. Using this in the
						previous inequality, we get $b'\,f\,a \subseteq b'\,b'\,f\subseteq  b\,f$.
					\item Need to show that $B.B.f \leq B.f$. Fix any $b \in B$, we will find
						$b' \in B$ such that $b'\,b'\,f \subseteq b\,f$.
						By quasi-uniformity of $B$, there exists $b \in B$ such that
						$b'\,b' \subseteq  b$. Using Lemma 3.10, we get $b'\,b'\,f \subseteq bf$.
				\end{enumerate}
				Thus, $B.f$ is a promodule. We now proceed to show that $(\string _)_*$ defines a functor.
				\begin{enumerate}
					\item (Composition) Need to show that $(g\circ f)_*=g_*f_*$ i.e. $C.g.f=C.g.B.f$.

						In order to show $C.g.f \leq C.g.B.f$, fix any $b\in B, c\in C$.
						We will show that $c\,g\,f \subseteq c\,g\,b\,f$. As $f$ is uniformly
						continuous, $f.A \leq B.f$ gives that there exists $a \in A$ such that
						$f\,a \subseteq b\,f$. Using Lemma 3.9, we get $(c\,g)f\,a \subseteq
						(c\,g)b\,f$. Now, using reflexiveness of $a$, we get $c\,g\,f \subseteq
						c\,g\,b\,f$.

						Now, to show that $C.g.f \geq C.g.B.f$. Fix any $c \in C$, we will find $c' \in C$ and
						$b \in B$ such that $c\, g\, f\, \supseteq c\, g\, b\, f\, $. By quasi-uniformity of C,
						there exists $c' \in C$ such that $c \supseteq  c'\, c'$. Using Lemma 3.10
						gives that $c\, (g\, f)\supseteq c'\, c'\, (g\, f) $. Because $g$ is uniformly
						continuous, $C.g \geq g.B$ gives us $b\in B$ such that $c'\, g \supseteq b\, g$.
						Using this in the previous inequality gives that $c\, g\, f \supseteq c'\, g\, b\, f$.

					\item(Identity) Let $(X,A)$ be an object of QUnif and
						$1_{(X,A)}:(X,A)\to(X,A)$ be the identity of $(X,A)$. That is,
						$1_{(X,A)}$ is defined as $x\mapsto x$.
						Need to show that $(1_{(X,A)})_*=1_{(X,A)_*}$. Using
						functor's definition, $LHS=(1_{(X,A)})_*=A.(1_{(X,A)})=A.1_{(X,A)}=A$
						and $RHS=1_{(X,A)_*}=1_{(X,A)}$
						Using Proposition 4.8 (f), we get that $A=1_{(X,A)}=RHS$. \qedhere
				\end{enumerate}
\end{frame}

\section{ContraFunc}
\setcounter{equation}{0}
\begin{frame}[allowframebreaks]{Proof: Contravariant Functor}
\setcounter{equation}{0}
				Showing that $f^o .B: (Y,B) \to (X,A)$ is a promodule.\\
				So, need to show $f^o .B$ a prorelation $Y \to X$
				and that $(f^o .B).B \leq f^o .B$ and $A.(f^o .B) \leq f^o .B$ \\
				To show prorelation, \begin{enumerate}
					\item (Partial-order) Inclusion of relations is the
						partial order.
					\item (Down directed) for $k,k' \in f^o .B$, need that $\exists l \in f^o .B
						\text{ such that } l \subseteq k,k'$

						Fix $k,k' \in f^o .B \implies \exists b,b' \in B : k=f^o \circ b \text{ and }
						k' = f^o \circ b'$

						By down-directedness of $B$, there exists $c \in B$ such that
						$ c \subseteq b,b'$, define $l=f^o \circ c$.
						Now, using Lemma 3.9 gives  $l= f^o \circ c \subseteq k,k'$.
					\item (Up-set) for a relation $l:Y \to X$ and $k \in f^o .B$ such that $l \supseteq k$
						, need $l \in f^o .B$

						Let $b\in B$ be such that $k=f^o \circ b$ and define
						$b':=\{(y,y'): y \in Dom(l) \text{ and } y' \in (f^o)^{-1}\big(l(y))\}$\\
						As $l\supseteq k=f^o \circ b$, $Dom(b')=Dom(l)\supseteq Dom(b)$
						\\ and $Ran(l) \supseteq Ran(f^o \circ b)\implies
						\forall y \in Dom(b), Ran(b')=(f^o )^{-1}(l(y)) \supseteq (f^o)^{-1}(f^o \circ b ) = Ran(b)$.\\
						Now, by definition of $b'$, $f^o \circ b' \supseteq l$. To show
						$f^o \circ b \subseteq l$ , \\
						$(x,y)\in f^o \circ b' \implies \exists z \in Y: (x,z)\in b' \text{ and }
						(z,y) \in f^o \implies x \in Dom(l) \text{ and } z \in l(x)$ i.e.
						$(x,z) \in l$.

			\item	To show $(f^o .B).B \leq f^o .B$, need that $\forall b \in B,
				\exists b' \in B : f^o \circ b' \circ b' \subseteq f^o \circ b$,\\
				Fix any $b \in B$, as B is a quasi-uniformity, $\exists b' \in B : b' \circ b' \subseteq b
				\implies f^o \circ b'\circ b' \subseteq f^o \circ b$.

				To show $A.(f^o .B) \leq f^o .B$, need that $\forall b \in B$,
				$\exists b' \in B, a\in A : a \circ f^o \circ b' \subseteq f^o \circ b$.\\
				As $f$ is uniformly continuous, $f.A\leq B.f$ i.e. $\forall b \in B, \exists a \in A
				: f \circ a \subseteq b \circ f
				\implies a= f^o \circ f \circ a \subseteq f^o \circ  b \circ f $   .\\
				Fix any $b \in B, \text{ so, } \exists b' \in B : b'b' \subseteq b$.
				And, for this $b', \exists a : a \subseteq f^ob'f \implies af^ob' \subseteq f^ob'ff^ob'
				\subseteq f^o b'b' \subseteq f^o b \implies af^ob' \subseteq f^o b$.\\
				\end{enumerate}
		Now, need to show that $(\string _)^*$ respects composition and identity.
				\begin{enumerate}
					\item (Composition) let $f,g$ be uniformly continuous,
						$(X,A) \xrightarrow{f} (Y,B) \xrightarrow{g} (Z,C)$
						need that $(g \circ f)^*= f^*.g^* $

						LHS=$(g \circ f)^*=(g \circ f)^o .C=(f^o \circ g^o).C$ and
						RHS=$f^*.g^* =(f^o .B).(g^o .C)$\\
						For equality, showing that LHS$\geq$RHS and LHS$\leq$RHS:

						To show $(f^o \circ g^o).C\geq(f^o .B).(g^o .C)$, need that
						$\forall c \in C, \exists b \in B, c' \in C : f^og^oc
						\supseteq f^obgc'$ \\
						Fix any $c \in C, \text{ so, } \exists c' \in C: c' \circ c' \subseteq c
						\implies f^o g^o c \supseteq f^o g^o (c'c')
						=f^o g^o (c' \Delta_Z c') \supseteq f^o g^o c'(gg^o)c'$ \\
						By uniform continuity of g, for $c'\in C,\exists b\in B: gb\subseteq c'g $
						\\Thus, $f^o g^o c \supseteq f^o g^o (c'g)g^oc' \supseteq
						f^o (g^o g)bg^o c'=f^o bg^o c'$.

						To show $(f^o \circ g^o).C\leq(f^o .B).(g^o .C)$, need that
						$\forall b \in B, c \in C, \exists c' \in C: f^o g^o c \subseteq f^obg^oc $
						\\Fix any $c\in C, b\in B$ will show that $c':=c$ works:\\
						As B is a quasi-uniformity, $\Delta_Y \subseteq b\implies f^o \Delta_Y
						g^o c=f^o g^o c \subseteq f^o b	g^o c=f^o b g^o c'$
					\item(Identity) let $(X,A)\in \text{ QUnif }^{op} $, and
						$1_{(X,A)}:(X,A)\to(X,A)$ as $x\mapsto x$ need that
						$(1_{(X,A)})^*=1_{(X,A)^*}$
						LHS=$(1_{(X,A)})^*=(1_{(X,A)})^o.A=1_{(X,A)}.A=A$. \\
						And as $RHS=1_{(X,A)^*}=1_{(X,A)}$
						Using Proposition 3.2(f), we get that $A=1_{(X,A)}=RHS$. \qedhere
				\end{enumerate}
\end{frame}

\section{ComPro}
\setcounter{equation}{0}
\begin{frame}[allowframebreaks]{Proof: Composition of Prorelations is a prorelation}
\setcounter{equation}{0}
				For prorelations $P:X \to Y$ and $Q:Y \to Z$, need to show that Q.P is a prorelation.
				\begin{enumerate}
					\item (Partial Order) Inclusion of relations gives a partial order.
					\item (Down-Directed) If $k,k' \in Q.P$, then $k=q\,p$ and $k'=q'\,p'$ for some
						$q,q' \in Q$ and $p,p' \in P$. Because $Q$ and $P$ are prorelations,
						and hence down-directed sets there exists, $a \in Q$ such that
						$ a\subseteq q,q'$ and $b \in P$ such that $b \subseteq p,p'$. Thus,
						giving an element, $a\circ b$ of $Q.P$ such that $a \circ b \subseteq k, k'$.
					\item (Up-Set) Let $l:X \to Z$ be a relation, and $k \in Q.P$ such that $l \supseteq k$.
						Define relations $ p:X \to Y $ and $q:Y \to Z$ as,
						$p=\{(x,y): x \in Dom(l) \text{ and } y \in Y\}$ and
						$q=\{(y,z):  y \in Y \text{ and } z \in Ran(l)\}$.
						Because $k\in Q.P$, there exist $q'\in Q$ and $p' \in P$ such that
						$k = q' \circ p'$. Thus by definition of $p$ and $q$, we get that
						$p \supseteq p'$ and $q \supseteq q'$. Hence $p \in P$ and $q \in Q$ because
						$P$ and $Q$ are up-sets, which gives us that $q\circ p \in Q.P$.
						For any $(x,z) \in l$, by definition
						of $p$ and $q$, we get that for every $y \in Y$, $(x,y) \in p$ and
						$(y,z) \in q $. By
						definition of composition, this gives that $(x,z) \in q \circ p$,
						giving that $l \subseteq q \circ p$. And, by definition
						of $q \circ p$ we get that $l \supseteq q \circ p$. Finally giving that
						$l=q \circ p \in Q.P$. \qedhere
				\end{enumerate}
\end{frame}

\section{Topo}
\begin{frame}{Topology induced by a quasi-uniformity}
    A subfamily $\mathbb{B}$ of quasi-uniformity $A$ is called a base for $A$
		if each relation in $A$ contains a relation in $\mathbb{B}$.
			\begin{prop}
				Let $\mathbb{B}$ be the base for quasi-uniformity $A$ on $X$.
				For $x \in X$, define $\mathbb{B}(x)=\{B(x) | B \in \mathbb{B}\}$.
				Then there is a unique topology on $X$ such that for each $x\in X$,
				$\mathbb{B}(x)$ is a base for the neighborhood of $x$ in this topology.
			\end{prop}
			We skip the proof as we have no requirement of it. But refer the interested reader to
\cite{Fletcher_Lindgren_1982} for similar results.
\end{frame}

\end{document}
